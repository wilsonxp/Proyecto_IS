%% LyX 2.0.3 created this file.  For more info, see http://www.lyx.org/.
%% Do not edit unless you really know what you are doing.
\documentclass[10pt,spanish]{article}
\usepackage[utf8x]{inputenc}
\usepackage[letterpaper]{geometry}
\geometry{verbose}
\usepackage{amsmath}
\usepackage{amssymb}

\makeatletter

%%%%%%%%%%%%%%%%%%%%%%%%%%%%%% LyX specific LaTeX commands.
%% Because html converters don't know tabularnewline
\providecommand{\tabularnewline}{\\}

%%%%%%%%%%%%%%%%%%%%%%%%%%%%%% User specified LaTeX commands.

\usepackage{ucs}\usepackage[spanish]{babel}
\usepackage{amsfonts}\usepackage{colortbl}% = = = = = = = = = = = = = = = = = = = = = = = = =
% INICIO EL DOCUMENTO
% = = = = = = = = = = = = = = = = = = = = = = = = =

\makeatother

\usepackage{babel}
\addto\shorthandsspanish{\spanishdeactivate{~<>}}

\begin{document}
\begin{Large} NOTA: Pase algunas de las reglas que ya había y documente
algunos mensajes, obvio faltan varios

\end{Large}


\section{Mensajes}


\subsection{MSG1 Mensaje de confirmación.}

\noindent Tipo: Notificación.

\noindent Estatus: Propuesta.

\noindent Objetivo: Indicar al usuario que la acción solicitada fue
realizada exitosamente.

\noindent Redacción: ARTICULO | ENTIDAD | VALOR ha sido OPERACIÓN
exitosamente.

\noindent Parámetros: El mensaje se muestra con base en lo siguiente
:
\begin{description}
\item [{ARTICULO:}] \noindent Tiene la función de limitar la extensión
de la entidad.
\item [{ENTIDAD:}] \noindent Producto | Ingrediente | Proveedor | Cliente
| Empleado | Reporte
\item [{VALOR:}] \noindent Valor de la entidad a la cual se le realiza
una operación. 
\item [{OPERACIÓN}] \noindent Agregar | Eliminar | Modificar | Generar
\item [{Ejemplo:}] El producto <<Arcoiris>> ha sido eliminado exitosamente.
\end{description}

\subsection{MSG2 Mensaje de error.}

\noindent Tipo: Notificación.

\noindent Estatus: Propuesta.

\noindent Objetivo: Indicar al usuario que ocurrio un error y no se
pudo realizar la acción solicitada.

\noindent Redacción: Error al intentar OPERACIÓN ARTICULO | ENTIDAD
| VALOR 

\noindent Parámetros: El mensaje se muestra con base en lo siguiente
:
\begin{description}
\item [{ARTICULO:}] \noindent Tiene la función de limitar la extensión
de la entidad.
\item [{ENTIDAD:}] \noindent Producto | Ingrediente | Proveedor | Cliente
| Empleado | Reporte
\item [{VALOR:}] \noindent Valor de la entidad a la cual se le realiza
una operación. 
\item [{OPERACIÓN}] \noindent Agregar | Eliminar | Modificar | Generar
\item [{Ejemplo:}] Error al intentar modificar el producto <<Arcoiris>>.
\end{description}

\subsection{MSG3 Datos no encontrados.}

\noindent Tipo: Notificación.

\noindent Estatus: Propuesta.

\noindent Objetivo: Indicar al usuario que los datos solicitados no
arrojarón resultados.

\noindent Redacción: ARTICULO ENTIDAD VALOR no muestra ningún resultado. 

\noindent Parámetros: El mensaje se muestra con base en lo siguiente
:
\begin{description}
\item [{ARTICULO:}] \noindent Tiene la función de limitar la extensión
de la entidad.
\item [{ENTIDAD:}] \noindent Producto | Ingrediente | Proveedor | Cliente
| Empleado | Venta | Compra
\item [{VALOR:}] \noindent Valor de la entidad. 
\item [{Ejemplo:}] Las ventas del 03/Mar/2011 al 03/Abril/2011 no muestran
ningún resultado.
\end{description}

\subsection{MSG4 Error al conectar con la base de datos.}

\noindent Tipo: Notificación.

\noindent Estatus: Propuesta.

\noindent Objetivo: Indicar al usuario que no se obtuvo conexión con
la base de datos.

\noindent Redacción: No se pudo establecer conexión con la base de
datos.


\subsection{MSG5 Error al iniciar sesión.}

\noindent Tipo: Notificación.

\noindent Estatus: Propuesta.

\noindent Objetivo: Notificar al usuario un error al iniciar sesión.

\noindent Redacción: El usuario o la contraseña son incorrectos. 

\subsection{MSG6 Ingredientes insuficientes.}

\noindent Tipo: Notificación.

\noindent Estatus: Propuesta.

\noindent Objetivo: Notificar al usuario que no existen los ingredientes
para la elaboración de un producto.

\noindent Redacción: ARTICULO cantidad ENTIDAD VALOR solicitada es
insuficiente. 

\noindent Parámetros: El mensaje se muestra con base a la BR1 atendiendo
a lo siguiente :
\begin{description}
\item [{ARTICULO:}] \noindent Tiene la función de limitar la extensión
de la entidad.
\item [{ENTIDAD:}] \noindent Limita la unidad de medición.
\item [{VALOR:}] \noindent insuficiente
\item [{Ejemplo:}] La cantidad de litros de leche solicitada es insuficiente.
\end{description}

\subsection{MSG7 Formato incorrecto de los datos.}

\noindent Tipo: Notificación.

\noindent Estatus: Propuesta.

\noindent Objetivo: Notificar al usuario un de formato en los datos
ingresados.

\noindent Redacción: ARTICULO ENTIDAD sólo acepta TIPO. 

\noindent Parámetros: El mensaje se muestra con base a la BR1 atendiendo
a lo siguiente :
\begin{description}
\item [{ARTICULO:}] \noindent Tiene la función de limitar la extensión
de la entidad.
\item [{ENTIDAD:}] \noindent Nombre del campo incorrecto.
\item [{TIPO:}] \noindent Tipo de datos indicado en el diccionario de datos
para cada entidad.
\item [{Ejemplo:}] \noindent El precio sólo acepta números.
\end{description}

\subsection{MSG8 Campos obligatorios}

\noindent Tipo: Notificación.

\noindent Estatus: Propuesta.

\noindent Objetivo: Notificar al usuario que no ingreso un campo obligatorio.

\noindent Redacción: ARTICULO ENTIDAD es un campo obligatorio. 

\noindent Parámetros: El mensaje se muestra atendiendo a lo siguiente
:
\begin{description}
\item [{ARTICULO:}] \noindent Tiene la función de limitar la extensión
de la entidad.
\item [{ENTIDAD:}] \noindent Nombre del campo obligatorio.
\item [{Ejemplo:}] \noindent El campo Nombre es un campo obligatorio.
\end{description}

\subsection{MSG9 Confirmar acción}

\noindent Tipo: Notificación.

\noindent Estatus: Propuesta.

\noindent Objetivo: Solicitar al usuario la conirmación de una acción.

\noindent Redacción: ¿Desea ACCIÓN ARTICULO ENTIDAD?

\noindent Parámetros: El mensaje se muestra atendiendo a lo siguiente
:
\begin{description}
\item [{ACCIÓN:}] \noindent Indica la acción a confirmar.
\item [{ARTICULO:}] \noindent Tiene la función de limitar la extensión
de la entidad.
\item [{ENTIDAD:}] \noindent Nombre del campo obligatorio.
\item [{Ejemplo:}] \noindent ¿Desea eliminar el producto?
\item 
\end{description}

\subsection{MSG10 Registro inexistente.}

\noindent Tipo: Notificación.

\noindent Estatus: Propuesta.

\noindent Objetivo: Indicar al usuario que el registro solicitado no existe.

\noindent Redacción: ARTICULO ENTIDAD VALOR no existe. 

\noindent Parámetros: El mensaje se muestra con base en lo siguiente
:
\begin{description}
\item [{ARTICULO:}] \noindent Tiene la función de limitar la extensión
de la entidad.
\item [{ENTIDAD:}] \noindent Producto | Ingrediente | Proveedor | Cliente
| Empleado | Número de lote
\item [{VALOR:}] \noindent Valor de la entidad. 
\item [{Ejemplo:}] El número de lote L00000  no existe.
\end{description}


\subsection{MSG11 La fecha de inicio es posterior a la fecha de fin}

\noindent Tipo: Notificación.

\noindent Estatus: Propuesta.

\noindent Objetivo: Indicar al usuario que la fecha de inicio es posterior a la fecha de fin.

\noindent Redacción: La fecha de inicio es posterior a la fecha de fin. 


\subsection{MSG12 La fecha indicada es posterior a la fecha del dia de hoy}

\noindent Tipo: Notificación.

\noindent Estatus: Propuesta.

\noindent Objetivo: Indicar al usuario que la fecha indicada es posterior a la fecha del dia de hoy.

\noindent Redacción: La fecha indicada es posterior a la fecha del dia de hoy. 


% = = = = = = = = = = = = = = = = = = = = = = = = = = = = = = = = = = =
% MENSAJES
% = = = = = = = = = = = = = = = = = = = = = = = = = = = = = = = = = = =
	\newpage
	\section{Mensajes del modulo de Producción}
	\subsection{MSGP1 Disponibilidad en el Almacén.}
	\begin{flushleft}
	Tipo: Notificación.
	\par\end{flushleft}
	\begin{flushleft}
	Estatus: Propuesto.
	\par\end{flushleft}
	\begin{flushleft}
	Objetivo: Indicar al usuario si se cuenta o no con espacio disponible en el almacén.
	\par\end{flushleft}
	\begin{flushleft}
	Redacción: ARTICULO ENTIDAD  VALOR disponibilidad para agregar productos.
	\par\end{flushleft}
	\begin{flushleft}
	Parámetros: El mensaje se muestra con base en los siguientes parámetros:
	\par\end{flushleft}
	\begin{description}
	\item [{ARTICULO}] Tiene la función de limitar la extensión de la entidad.
	\item [{ENTIDAD}] Almacén
	\item [{VALOR}] Valor de la entidad a la cual se le realiza una operación.
	\end{description}
	\begin{flushleft}
	Ejemplo: El Almacén tiene disponibilidad para agregar productos, El Almacén NO tiene disponibilidad para agregar productos.
	\par\end{flushleft}
% = = = = = = = = = = = = = = = = = = = = = = = = = = = = = = = = = = =
% MENSAJE 2
% = = = = = = = = = = = = = = = = = = = = = = = = = = = = = = = = = = =
	\subsection{MSGP2 Información de Almacén no disponible.}	
	\begin{flushleft}
	Tipo: Notificación.
	\par\end{flushleft}
	\begin{flushleft}
	Estatus: Propuesto.
	\par\end{flushleft}
	\begin{flushleft}
	Objetivo: Indicar al usuario que no se tiene registró acerca de la disponibilidad del almacén
	\par\end{flushleft}
	\begin{flushleft}
	Redacción: Información ARTICULO ENTIDAD VALOR
	\par\end{flushleft}
	\begin{flushleft}
	Parámetros: El mensaje se muestra con base en los siguientes parámetros:
	\par\end{flushleft}
	\begin{description}
	\item [{ARTICULO}] Tiene la función de limitar la extensión de la entidad.
	\item [{ENTIDAD}] Almacén
	\item [{VALOR}] Valor de la entidad a la cual se le realiza una operación.
	\end{description}
	\begin{flushleft}
	Ejemplo: Información de Almacén no disponible
	\par\end{flushleft}
% = = = = = = = = = = = = = = = = = = = = = = = = = = = = = = = = = = =
% MENSAJE 3
% = = = = = = = = = = = = = = = = = = = = = = = = = = = = = = = = = = =
	\subsection{MSGP3 No existen pedidos pendientes.}	
	\begin{flushleft}
	Tipo: Notificación.
	\par\end{flushleft}
	\begin{flushleft}
	Estatus: Propuesto.
	\par\end{flushleft}
	\begin{flushleft}
	Objetivo: Indicar al usuario que no se tiene registró de pedidos pendientes.
	\par\end{flushleft}
	\begin{flushleft}
	Redacción: No se encontraron ENTIDAD VALOR en espera.
	\par\end{flushleft}
	\begin{flushleft}
	Parámetros: El mensaje se muestra con base en los siguientes parámetros:
	\par\end{flushleft}
	\begin{description}
	\item [{ENTIDAD}] Pedidos
	\item [{VALOR}] Valor de la entidad a la cual se le realiza una operación.
	\end{description}
	\begin{flushleft}
	Ejemplo: No se encontraron pedidos pendientes en espera.\newline
	No se encontraron pedidos para cambio en espera
	\par\end{flushleft}	


\section{Reglas de Negocio}


\subsection{BR1 Elaboración de lotes}

\noindent Tipo: Restricción.

\noindent Nivel: Obligatorio. Debe cumplirse siempre.

\noindent Versión: 1.0

\noindent Estado: Propuesta.

\noindent Descripción: Para poder elaborar un producto las cantidades
de cada ingrediente indicadas en la receta deben existir en el almacen
y deben provenir de un mismo proveedor,esto con el fin de facilitar
el control de calidad.

\noindent Ejemplo: 

\noindent %
\begin{tabular}{|c|}
\hline 
\multicolumn{1}{|c|}{Se necesitan 10 litros de leche}\tabularnewline
\hline 
\end{tabular}%
\begin{tabular}{|c|}
\hline 
\multicolumn{1}{|c|}{Se tienen 8 litros del proveedor A y 2 litros del proveedor B}\tabularnewline
\hline 
\end{tabular}

\noindent El producto no podría elaborarse porque no existe la cantidad
de materia prima suficiente proveniente de un mismo proveedor.


\subsection{BR2 Forma de pago}

\noindent Tipo: Definición

\noindent Nivel: Obligatorio

\noindent Versión:1.0

\noindent Estado: Propuesta

\noindent Descripción: Se debe indicar en toda venta realizada la
forma de pago, y en caso de ser necesario, los ultimos cuatro dgitos
del numero de cuenta/ tarjeta por cuestiones fiscales.


\subsection{BR3 Límite de almacén}

\noindent Tipo: Restrictiva.

\noindent Nivel: Obligatorio

\noindent Versión:1.0

\noindent Estado: Propuesta

\noindent Descripción: La cantidad de galletas que se pueden almacenar
en el almacen estan limitas a 100,000 cajas. Si se llega a esta límite
debe detenerse la producción.


\subsection{BR4 Cliente prioritario}

\noindent Tipo: Definición.

\noindent Nivel: Obligatorio

\noindent Versión:1.0

\noindent Estado: Propuesta

\noindent Descripción: Si un pedido tiene el caracter de compensación
de producto (por alguna falla reportada por el cliente o problema
que se presente durante o despues de su elaboración), este debe tener
prioridad para su entrega, sobre cualquier otra entrega a efectuarse.


\subsection{BR5 Formato de los reportes}

\noindent Tipo: Definición.

\noindent Nivel: Obligatorio

\noindent Versión:1.0

\noindent Estado: Propuesta

\noindent Descripción: Todo reporte debe indicar con el membrete de
la empresa, deberá indicar el área de la que proviene, fecha de elaboración
, periodo que abarca y nombre de quien lo elaboró.


\subsection{BR6 Líneas de producción.}

\noindent Tipo: Restricción.

\noindent Nivel: Obligatorio

\noindent Versión:1.0

\noindent Estado: Propuesta

\noindent Descripción: Existen unicamentes tres líneas de producción,
las cuales se utilizan alternadamente.


% = = = = = = = = = = = = = = = = = = = = = = = = = = = = = = = = = = =
% REGLA DE NEGOCIO 1
% = = = = = = = = = = = = = = = = = = = = = = = = = = = = = = = = = = =
	\subsection{BRP1 Pedidos en espera}
	\begin{flushleft}
	Regla: BRP1 Pedidos en espera
	\par\end{flushleft}
	\begin{flushleft}
	Tipo: Definición
	\par\end{flushleft}
	\begin{flushleft}
	Nivel: Obligatorio. Debe cumplirse siempre.
	\par\end{flushleft}
	\begin{flushleft}
	Versión: 1.0
	\par\end{flushleft}
	\begin{flushleft}
	Estado: Propuesta
	\par\end{flushleft}
	\begin{flushleft}
	Descripción: Los pedidos en espera se pueden clasificar principalmente en 2 tipos:\newline
	\begin{itemize}
	\item Pedidos Pendientes
	\item Pedidos para cambio
	\end{itemize}
	\par\end{flushleft}
	\begin{flushleft}
	Siendo los primeros, en relación a producción de nuevos lotes y los segundos pedidos a los cuales se debe hacer un cambio de producto debido a un error de producción y/o control de calidad.
	\end{flushleft}
% = = = = = = = = = = = = = = = = = = = = = = = = = = = = = = = = = = =
% REGLA DE NEGOCIO 2
% = = = = = = = = = = = = = = = = = = = = = = = = = = = = = = = = = = =
	\subsection{BRP2 Contenido del pedido}
	\begin{flushleft}
	Regla: BRP2 Contenido del pedido
	\par\end{flushleft}
	\begin{flushleft}
	Tipo: Restricción
	\par\end{flushleft}
	\begin{flushleft}
	Nivel: Obligatorio. Debe cumplirse siempre.
	\par\end{flushleft}
	\begin{flushleft}
	Versión: 1.0
	\par\end{flushleft}
	\begin{flushleft}
	Estado: Propuesta
	\par\end{flushleft}
	\begin{flushleft}
	Descripción: El registró de los pedidos principalmente debe estar formado principalmente por los siguientes campos:\newline
	\begin{itemize}
	\item Numero de pedido
	\item Nombre del cliente
	\item Descripción del pedido
	\end{itemize}
	\par\end{flushleft}
% = = = = = = = = = = = = = = = = = = = = = = = = = = = = = = = = = = =
% REGLA DE NEGOCIO 3
% = = = = = = = = = = = = = = = = = = = = = = = = = = = = = = = = = = =
	\subsection{BRP3 Búsqueda de ingredientes}
	\begin{flushleft}
	Regla: BRP3 Búsqueda de ingredientes
	\par\end{flushleft}
	\begin{flushleft}
	Tipo: Restricción
	\par\end{flushleft}
	\begin{flushleft}
	Nivel: Obligatorio. Debe cumplirse siempre.
	\par\end{flushleft}
	\begin{flushleft}
	Versión: 1.0
	\par\end{flushleft}
	\begin{flushleft}
	Estado: Propuesta
	\par\end{flushleft}
	\begin{flushleft}
	Descripción: Tomando en cuenta que para realizar la búsqueda el ingrediente debe estar previamente registrada en el catalogo correspondiente, 
	el resultado de la búsqueda por ingrediente debe devolver los siguientes datos:
	\begin{itemize}
	\item Identificador de producto
	\item Nombre del producto
	\item Proveedor
	\item Disponibilidad	
	\end{itemize}
	\par\end{flushleft}
% = = = = = = = = = = = = = = = = = = = = = = = = = = = = = = = = = = =
% REGLA DE NEGOCIO 4
% = = = = = = = = = = = = = = = = = = = = = = = = = = = = = = = = = = =
	\subsection{BRP4 Información de Lote}
	\begin{flushleft}
	Regla: BRP4 Información de Lote
	\par\end{flushleft}
	\begin{flushleft}
	Tipo: Descripción.
	\par\end{flushleft}
	\begin{flushleft}
	Nivel: Obligatorio. Debe cumplirse siempre.
	\par\end{flushleft}
	\begin{flushleft}
	Versión: 1.0
	\par\end{flushleft}
	\begin{flushleft}
	Estado: Propuesta
	\par\end{flushleft}
	\begin{flushleft}
	Descripción: Cuando registramos un lote nos interesa conocer información descriptiva del lote, para lograr esto se toma en cuenta los siguientes campos:
	\begin{itemize}
	\item Numero Identificador de lote[Asignado únicamente por el sistema]
	\item Producto Asociado al numero de Lote
	\item Fecha de Elaboración
	\item Linea de Producción asociada
	\end{itemize}
	\par\end{flushleft}
	% = = = = = = = = = = = = = = = = = = = = = = = = =
	%	FIN DEL DOCUMENTO
	% = = = = = = = = = = = = = = = = = = = = = = = = =

\end{document}

% = = = = = = = = = = = = = = = = = = = = = = = = =
%	FIN DEL DOCUMENTO
% = = = = = = = = = = = = = = = = = = = = = = = = =

\end{document}
