\documentclass[10pt,letterpaper]{article}
\usepackage[utf8x]{inputenc}
\usepackage{ucs}
\usepackage[spanish]{babel}
\usepackage{amsmath}
\usepackage{amsfonts}
\usepackage{amssymb}
\usepackage{graphicx}
\usepackage{colortbl}
\usepackage[left=2cm,right=2cm,top=2cm,bottom=2cm]{geometry}
% = = = = = = = = = = = = = = = = = = = = = = = = =
% INICIO EL DOCUMENTO
% = = = = = = = = = = = = = = = = = = = = = = = = =
\begin{document}
\begin{Large}
NOTA: AGREGAR MARGENES A LA TABLA, AGREGAR IMAGENES EN LAS TRAYECTORIAS, ALINEAR MENSAJES Y BR
\end{Large}

\section{Documentacion de Casos de Uso}
\subsection{CUX\#.\#: Nombre del caso de uso}
\textbf{\large Resumen}\\
Este CU tiene como objetivo permitir al usuario blablabla blablabla blablabla blablabla blablabla blablabla
 blablabla blablabla blablabla blablabla blablabla.\\

% = = = = = = = = = = = = = = = = = = = = = = = = =
%	INICIO DE TABLA
% = = = = = = = = = = = = = = = = = = = = = = = = = 
\begin{table}[!ht]
\begin{center}

\begin{tabular}{|c|c|c|}
\hline
\multicolumn{2}{|c|}{Caso de Uso:} & CU\#.\#: Nombre del Caso de Uso\\
\hline
\multicolumn{3}{|>{\columncolor[gray]{0.7}}c|}{Resumen de Atributos}\\
\hline
\multicolumn{2}{|c|}{Autor} & El nombre de quien lo hizo.\\
\hline
\multicolumn{2}{|c|}{Actor} & Usuario\\
\hline
\multicolumn{2}{|c|}{Propósito} & Permitir al usuario realizar la acción.\\
\hline
\multicolumn{2}{|c|}{Entradas} & Para realizar ACCIÓN el USUARIO deberá capturar etc etc etc:\\
\hline
\multicolumn{2}{|c|}{Salidas} & Se mostrara en pantalla el mensaje MSG\# cuando la operación.\\
\hline
\multicolumn{2}{|c|}{Pre-condiciones} & Que existan bla bla bla bla bla.\\
\hline
\multicolumn{2}{|c|}{Pos-condiciones} & Se realizara bla bla bla bla bla.\\
\hline
\multicolumn{2}{|c|}{Errores} & Se mostrara el mensaje bla bla bla.\\
\hline
\multicolumn{2}{|c|}{Tipo} & Primario | Secundario | etc.\\
\hline
\multicolumn{2}{|c|}{Fuente} & Basado en el funcionamiento etc.\\
\hline
\end{tabular}

\end{center}
\caption{Caso de Uso \#: Nombre del CU}
\label{tab:CasosdeUso:nombredecasodeuso}
\end{table}
% = = = = = = = = = = = = = = = = = = = = = = = = =
%	FIN DE TABLA
% = = = = = = = = = = = = = = = = = = = = = = = = =
% = = = = = = = = = = = = = = = = = = = = = = = = =
%	INICIO DE TRAYECTORIA
% = = = = = = = = = = = = = = = = = = = = = = = = = 
\textbf{\large Trayectorias del CU}\\
\textit{Trayectoria Principal}
\begin{enumerate}
  \item ACTOR solicita registrar un idioma presionando el boton [UNBOTON] de la pantalla IU\# NombredePantalla.
  \item SISTEMA busca los elementos de los catalogos nombredecatalogo.
  \item SISTEMA realiza otra accion mediante el uso de la pantalla IU\# NombredePantalla. [Trayectoria A].
  \item ACTOR ingresa los datos del idioma.
  \item ACTOR selecciona la opcion 'No' en ¿Cuenta con certificacion?. [Trayectoria B].
  \item SISTEMA realiza otra accion.
  \item ACTOR realiza otra accion.
  \item SISTEMA realiza otra accion.\\
  \ldots \ldots Fin del Caso de Uso\\
\end{enumerate}

\textit{Trayectoria Alternativa A}\\
Condición: Especificar condición.\\
\begin{description}
\item [A-1]. SISTEMA muestra la pantalla IU\# Nombredepantalla indicando etc.
\item [A-2]. Fin del Caso de Uso\\
\end{description}
\newpage
\textit{Trayectoria Alternativa B}\\
Condición: Especificar condición.\\
\begin{description}
\item [B-1]. ACTOR selecciona si en la opción.
\item [B-2]. SISTEMA muestra un formulario y solicita al usuario la información complementaria.
\item [B-3]. ACTOR realiza otra acción.
\item [B-N]. ACTOR realiza otra acción.
\item [- - - -]Fin de la trayectoria.
\end{description}

\section{Mensajes}
\subsection{MSG1 Nombre del Mensaje}
\begin{flushleft}
Tipo:	Notificación.\\
Estatus:	Aprobado.\\
Objetivo:	Indicar al usuario que la acción solicitada sea realizada exitosamente.\\
Redacción:	ARTICULO ENTIDAD VALOR ha sido OPERACIÓN exitosamente.\\
Parámetros:	El mensaje se muestra con base en los siguientes 1:\\
\begin{description}
\item[ARTICULO] Añadir descripción.
\item[ENTIDAD] Añadir descripción.
\item[VALOR] Añadir descripción.
\item[OPERACIÓN] Añadir descripción.
\end{description}
\end{flushleft}

\section{Reglas de Negocio}
\subsection{BR\# Nombre de la RN}
\begin{flushleft}
Regla: BR1 Información Correcta.\\
Tipo:	Restricción.\\
Nivel:	Obligatorio. Debe cumplirse siempre.\\
Versión:	1.0\\
Estado:		Aprobado.\\
Descripción:	La información que se proporcione al CVU marcada como obligatoria debe pertenecer al tipo de dato especificado \ldots.
\end{flushleft}
% = = = = = = = = = = = = = = = = = = = = = = = = =
%	FIN DEL DOCUMENTO
% = = = = = = = = = = = = = = = = = = = = = = = = =
\end{document}
