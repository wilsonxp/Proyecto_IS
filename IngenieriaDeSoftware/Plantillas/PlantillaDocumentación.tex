
\documentclass[10pt,spanish]{article}
\usepackage[latin1]{inputenc}
\usepackage[letterpaper]{geometry}
\geometry{verbose}
\usepackage{amsmath}
\usepackage{amssymb}
\usepackage{graphicx}

\makeatletter

\providecommand{\tabularnewline}{\\}

\usepackage{ucs}\usepackage[spanish]{babel}
\usepackage{amsfonts}\usepackage{colortbl}% = = = = = = = = = = = = = = = = = = = = = = = = =
% INICIO EL DOCUMENTO
% = = = = = = = = = = = = = = = = = = = = = = = = =


\makeatother

\usepackage{babel}
\addto\shorthandsspanish{\spanishdeactivate{~<>}}

\begin{document}
	\begin{Large} NOTA: Esta es la plantilla para documentar, la tabla
	de la trayectoria principal se copia en las alternativas, solo es
	copiar y pegar dependiendo de si se requiere algo del sistema o algo
	de un actori

	\end{Large}


	\section{Documentación de Casos de Uso}


		\subsection{CUX\#.\#: Nombre del caso de uso}

		\textbf{\large Resumen}{\large }\\
		{\large{} Este CU tiene como objetivo permitir al usuario blablabla
		blablabla blablabla blablabla blablabla blablabla blablabla blablabla
		blablabla blablabla blablabla.}\\
		{\large \par}

		% = = = = = = = = = = = = = = = = = = = = = = = = =
		%	INICIO DE TABLA
		% = = = = = = = = = = = = = = = = = = = = = = = = = 
		\begin{table}[!ht]
		\begin{centering}
		\begin{tabular}{|c||c|l|}
		\hline 
		\multicolumn{2}{|c|}{Caso de Uso:} & CU\#.\#: Nombre del Caso de Uso\tabularnewline
		\hline 
		\multicolumn{3}{|>{\columncolor[gray]{0.7}}c}{Resumen de Atributos}\tabularnewline
		\hline 
		\multicolumn{2}{|c|}{Autor} & El nombre de quien lo hizo.\tabularnewline
		\hline 
		\multicolumn{2}{|c|}{Actor} & Usuario\tabularnewline
		\hline 
		\multicolumn{2}{|c|}{Propósito} & Permitir al usuario realizar la acción.\tabularnewline
		\hline 
		\multicolumn{2}{|c|}{Entradas} & Para realizar ACCIÓN el USUARIO deberá capturar etc etc etc:\tabularnewline
		\hline 
		\multicolumn{2}{|c|}{Salidas} & Se mostrara en pantalla el mensaje MSG\# cuando la operación.\tabularnewline
		\hline 
		\multicolumn{2}{|c|}{Pre-condiciones} & Que existan bla bla bla bla bla.\tabularnewline
		\hline 
		\multicolumn{2}{|c|}{Pos-condiciones} & Se realizara bla bla bla bla bla.\tabularnewline
		\hline 
		\multicolumn{2}{|c|}{Errores} & Se mostrara el mensaje bla bla bla.\tabularnewline
		\hline 
		\multicolumn{2}{|c|}{Tipo} & Primario | Secundario | etc.\tabularnewline
		\hline 
		\multicolumn{2}{|c|}{Fuente} & Basado en el funcionamiento etc.\tabularnewline
		\hline 
		\end{tabular}
		\par\end{centering}

	\caption{Caso de Uso \#: Nombre del CU}


	\label{tab:CasosdeUso:nombredecasodeuso} 
	\end{table}


	% = = = = = = = = = = = = = = = = = = = = = = = = =
	%	FIN DE TABLA
	% = = = = = = = = = = = = = = = = = = = = = = = = =
	% = = = = = = = = = = = = = = = = = = = = = = = = =
	%	INICIO DE TRAYECTORIA
	% = = = = = = = = = = = = = = = = = = = = = = = = = 
	\textbf{\large Trayectorias del CU}{\large \par}

	\textit{\large Trayectoria Principal}{\large{} }{\large \par}

	\begin{tabular}{ccl}
	 &  & \tabularnewline
	1. & \includegraphics{actor} & Solicita accion tal presionando boton{[}botonX{]} de la pantalla\#.\#\tabularnewline
	2. & \includegraphics{sistema} & Despliega formulario\tabularnewline
	3. & \includegraphics{sistema} & \tabularnewline
	4. & \includegraphics{actor} & Hace otra accion\tabularnewline
	 &  & .... Fin del caso de uso\tabularnewline
	\end{tabular}

	\textit{Trayectoria Alternativa A} 

	Condición: Especificar condición.

	\newpage{}\textit{Trayectoria Alternativa B} 

	Condición: Especificar condición.


	\section{Mensajes}


	\subsection{MSG1 Nombre del Mensaje}

	\begin{flushleft}
	Tipo: Notificación.
	\par\end{flushleft}

	\begin{flushleft}
	Estatus: Aprobado.
	\par\end{flushleft}

	\begin{flushleft}
	Objetivo: Indicar al usuario que la acción solicitada sea realizada
	exitosamente.
	\par\end{flushleft}

	\begin{flushleft}
	Redacción: ARTICULO ENTIDAD VALOR ha sido OPERACIÓN exitosamente.
	\par\end{flushleft}

	\begin{flushleft}
	Parámetros: El mensaje se muestra con base en los siguientes 1:
	\par\end{flushleft}
	\begin{description}
	\item [{ARTICULO}] Añadir descripción. 
	\item [{ENTIDAD}] Añadir descripción. 
	\item [{VALOR}] Añadir descripción. 
	\item [{OPERACIÓN}] Añadir descripción. 
	\end{description}

	\section{Reglas de Negocio}


	\subsection{BR\# Nombre de la RN}

	\begin{flushleft}
	Regla: BR1 Información Correcta.
	\par\end{flushleft}

	\begin{flushleft}
	Tipo: Restricción.
	\par\end{flushleft}

	\begin{flushleft}
	Nivel: Obligatorio. Debe cumplirse siempre.
	\par\end{flushleft}

	\begin{flushleft}
	Versión: 1.0
	\par\end{flushleft}

	\begin{flushleft}
	Estado: Aprobado.
	\par\end{flushleft}

	\begin{flushleft}
	Descripción: La información que se proporcione al CVU marcada como
	obligatoria debe pertenecer al tipo de dato especificado \ldots{}. 
	\par\end{flushleft}

	% = = = = = = = = = = = = = = = = = = = = = = = = =
	%	FIN DEL DOCUMENTO
	% = = = = = = = = = = = = = = = = = = = = = = = = =

\end{document}
