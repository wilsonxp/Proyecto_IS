%% LyX 2.0.3 created this file.  For more info, see http://www.lyx.org/.
%% Do not edit unless you really know what you are doing.
\documentclass[10pt,spanish]{article}
\usepackage[utf8x]{inputenc}
\usepackage[letterpaper]{geometry}
\geometry{verbose}
\usepackage{amsmath}
\usepackage{amssymb}
\usepackage{graphicx}

\makeatletter

%%%%%%%%%%%%%%%%%%%%%%%%%%%%%% LyX specific LaTeX commands.
%% Because html converters don't know tabularnewline
\providecommand{\tabularnewline}{\\}

%%%%%%%%%%%%%%%%%%%%%%%%%%%%%% User specified LaTeX commands.

\usepackage{ucs}\usepackage[spanish]{babel}
\usepackage{amsfonts}\usepackage{colortbl}% = = = = = = = = = = = = = = = = = = = = = = = = =
% INICIO EL DOCUMENTO
% = = = = = = = = = = = = = = = = = = = = = = = = =



\usepackage{babel}
\addto\shorthandsspanish{\spanishdeactivate{~<>}}





\usepackage{babel}
\addto\shorthandsspanish{\spanishdeactivate{~<>}}

\makeatother

\usepackage{babel}
\addto\shorthandsspanish{\spanishdeactivate{~<>}}

\begin{document}

\section{Documentación de Casos de Uso}


\subsection{CUG1.0: Gestionar Notificaciones}

\textbf{\large Resumen}{\large }\\
{\large{} Este Caso de uso tiene como objetivo permitir al usuario
enviar, recibir y eliminar notificaciones a otras áreas de la empresa.}\\


% = = = = = = = = = = = = = = = = = = = = = = = = =
%	INICIO DE TABLA
% = = = = = = = = = = = = = = = = = = = = = = = = = 
\begin{table}[!ht]
\begin{centering}
\begin{tabular}{|c||c|l|}
\hline 
\multicolumn{2}{|c|}{Caso de Uso:} & CUG1.0: Gestionar Notificaciones\tabularnewline
\hline 
\multicolumn{3}{|>{\columncolor[gray]{0.7}}c}{Resumen de Atributos}\tabularnewline
\hline 
\multicolumn{2}{|c|}{Autor} & \multicolumn{1}{p{10cm}||}{Arcos Reyes Larissa}\tabularnewline
\hline 
\multicolumn{2}{|c|}{Actor} & \multicolumn{1}{p{10cm}||}{Todos los usuarios del sistema.}\tabularnewline
\hline 
\multicolumn{2}{|c|}{Propósito} & \multicolumn{1}{p{10cm}||}{a}\tabularnewline
\hline 
\multicolumn{2}{|c|}{Entradas} & \multicolumn{1}{p{10cm}||}{a}\tabularnewline
\hline 
\multicolumn{2}{|c|}{Salidas} & \multicolumn{1}{p{10cm}||}{a}\tabularnewline
\hline 
\multicolumn{2}{|c|}{Pre-condiciones} & \multicolumn{1}{p{10cm}||}{a}\tabularnewline
\hline 
\multicolumn{2}{|c|}{Pos-condiciones} & \multicolumn{1}{p{10cm}||}{a}\tabularnewline
\hline 
\multicolumn{2}{|c|}{Errores} & \multicolumn{1}{p{10cm}||}{}\tabularnewline
\hline 
\multicolumn{2}{|c|}{Tipo} & \multicolumn{1}{p{10cm}||}{Primario | Secundario | etc.}\tabularnewline
\hline 
\multicolumn{2}{|c|}{Fuente} & \multicolumn{1}{p{10cm}||}{Basado en el funcionamiento etc.}\tabularnewline
\hline 
\end{tabular}
\par\end{centering}

\caption{Caso de Uso 1.0: Gestionar Notificaciones}


\label{tab:CasosdeUso:nombredecasodeuso} 
\end{table}


% = = = = = = = = = = = = = = = = = = = = = = = = =
%	FIN DE TABLA
% = = = = = = = = = = = = = = = = = = = = = = = = =
% = = = = = = = = = = = = = = = = = = = = = = = = =
%	INICIO DE TRAYECTORIA
% = = = = = = = = = = = = = = = = = = = = = = = = = 


\textbf{\large Trayectorias del CU}{\large \par}

\textit{\large Trayectoria Principal}{\large {} }{\large \par}

\begin{tabular}{ccl}
 &  & \tabularnewline
1.  & \includegraphics{actor}  & \multicolumn{1}{p{12cm}}{ Presiona el icono {[}notificaciones{]} en la pantalla PG1}\tabularnewline
2.  & \includegraphics{sistema}  & \multicolumn{1}{p{12cm}}{Carga los datos de las notificaciones correspondientes al usuario
{[}Trayectoria Alternativa A{]}}\tabularnewline
3.  & \includegraphics{sistema}  & \multicolumn{1}{p{12cm}}{Despliega la pantalla PG2}\tabularnewline
 &  & \multicolumn{1}{p{12cm}}{.... Fin del caso de uso}\tabularnewline
\end{tabular}

\textit{Trayectoria Alternativa A}

Condición: No se pudieron cargar los datos de las notificaciones anteriores

\begin{tabular}{ccl}
 &  & \tabularnewline
1.  & \includegraphics{sistema}  & \multicolumn{1}{p{12cm}}{ Muestra el mensaje de error MSG4 'No se pudo establecer conexión
con la base de datos'}\tabularnewline
2.  & \includegraphics{sistema}  & \multicolumn{1}{p{12cm}}{Regresa a la pantalla PG1}\tabularnewline
 &  & \multicolumn{1}{p{12cm}}{.... Fin del caso de uso}\tabularnewline
\end{tabular}

\subsection{CUG4.0: Gestionar Ventas}

\textbf{\large Resumen}{\large }\\
{\large{} Este Caso de uso tiene como objetivo permitir al usuario
registrar, modificar, buscar y cancelar ventas.}\\


% = = = = = = = = = = = = = = = = = = = = = = = = =
%	INICIO DE TABLA
% = = = = = = = = = = = = = = = = = = = = = = = = = 
% Se utiliza para que no se salga la tabla de la hoja, porque sino se autodimensiona
%\multicolumn{1}{p{10cm}||}{Contenido}
%
\begin{table}[!ht]
\begin{centering}
\begin{tabular}{|c||c|l|}
\hline 
\multicolumn{2}{|c|}{Caso de Uso:} & CUG4.0: Gestionar Ventas\tabularnewline
\hline 
\multicolumn{3}{|>{\columncolor[gray]{0.7}}c}{Resumen de Atributos}\tabularnewline
\hline 
\multicolumn{2}{|c|}{Autor} & \multicolumn{1}{p{10cm}||}{Cabrera Alvarez Estefany Viridiana}\tabularnewline
\hline 
\multicolumn{2}{|c|}{Actor} & \multicolumn{1}{p{10cm}||}{Usuario de Ventas, Administrador}\tabularnewline
\hline 
\multicolumn{2}{|c|}{Propósito} & \multicolumn{1}{p{10cm}||}{Acceder a opciones de crear, modificar,consultar o buscar venta.
}\tabularnewline
\hline 
\multicolumn{2}{|c|}{Entradas} & \multicolumn{1}{p{10cm}||}{Ninguna}\tabularnewline
\hline 
\multicolumn{2}{|c|}{Salidas} & \multicolumn{1}{p{10cm}||}{Interfaz Correspondiente a gestionar Ventas}\tabularnewline
\hline 
\multicolumn{2}{|c|}{Pre-condiciones} & \multicolumn{1}{p{10cm}||}{El usuario de ventas debe estar autentificado}\tabularnewline
\hline 
\multicolumn{2}{|c|}{Pos-condiciones} & \multicolumn{1}{p{10cm}||}{La base de datos debe estar disponible}\tabularnewline
\hline 
\multicolumn{2}{|c|}{Errores} & \multicolumn{1}{p{10cm}||}{MSG4 Error al conectar con la base de datos.}\tabularnewline
\hline 
\multicolumn{2}{|c|}{Tipo} & \multicolumn{1}{p{10cm}||}{Primario}\tabularnewline
\hline 
\multicolumn{2}{|c|}{Fuente} & \multicolumn{1}{p{10cm}||}{Basado en el funcionamiento etc.}\tabularnewline
\hline 
\end{tabular}
\par\end{centering}

\caption{Caso de Uso 4.0: Gestionar Ventas}
\label{tab:CasosdeUso:nombredecasodeuso} 
\end{table}


% = = = = = = = = = = = = = = = = = = = = = = = = =
%	FIN DE TABLA
% = = = = = = = = = = = = = = = = = = = = = = = = =
% = = = = = = = = = = = = = = = = = = = = = = = = =
%	INICIO DE TRAYECTORIA
% = = = = = = = = = = = = = = = = = = = = = = = = = 


\textbf{\large Trayectorias del CU}{\large \par}

\textit{\large Trayectoria Principal}{\large {} }{\large \par}

%Solo hay que cambiar el nombre de la imagen dependiendo de si es actor o sistema

\begin{tabular}{ccl}
 &  & \tabularnewline
1.  & \includegraphics{actor}  & \multicolumn{1}{p{12cm}}{ Presiona el botón {[}Ventas{]} en la pantalla PV1}\tabularnewline
2.  & \includegraphics{sistema}  & \multicolumn{1}{p{12cm}}{Carga los datos de las ventas registradas en la Base de datos. 		
{[}Trayectoria Alternativa A{]}}\tabularnewline
3.  & \includegraphics{sistema}  & \multicolumn{1}{p{12cm}}{Despliega la pantalla PV2}\tabularnewline
 &  & \multicolumn{1}{p{12cm}}{.... Fin del caso de uso}\tabularnewline
\end{tabular}

\textit{Trayectoria Alternativa A}

Condición: No se pudieron cargar los datos de las Ventas anteriores

\begin{tabular}{ccl}
 &  & \tabularnewline
1.  & \includegraphics{sistema}  & \multicolumn{1}{p{12cm}}{ Muestra el mensaje de error MSG4 'No se pudo establecer conexión
con la base de datos'}\tabularnewline
2.  & \includegraphics{sistema}  & \multicolumn{1}{p{12cm}}{Regresa a la pantalla PV1}\tabularnewline
 &  & \multicolumn{1}{p{12cm}}{.... Fin del caso de uso}\tabularnewline
\end{tabular}

\newpage{}
%====================
%Nuevo caso de uso
%====================

\subsubsection{CUS4.1:Agregar Venta}

\textbf{\large Resumen}{\large }\\
{\large{} Este Caso de uso tiene como objetivo permitir al usuario
registrar, modificar, buscar y cancelar ventas.}\\


% = = = = = = = = = = = = = = = = = = = = = = = = =
%	INICIO DE TABLA
% = = = = = = = = = = = = = = = = = = = = = = = = = 
% Se utiliza para que no se salga la tabla de la hoja, porque sino se autodimensiona
%\multicolumn{1}{p{10cm}||}{Contenido}
%
\begin{table}[!ht]
\begin{centering}
\begin{tabular}{|c||c|l|}
\hline 
\multicolumn{2}{|c|}{Caso de Uso:} & CUG4.1: Agregar Venta\tabularnewline
\hline 
\multicolumn{3}{|>{\columncolor[gray]{0.7}}c}{Resumen de Atributos}\tabularnewline
\hline 
\multicolumn{2}{|c|}{Autor} & \multicolumn{1}{p{10cm}||}{Cabrera Alvarez Estefany Viridiana}\tabularnewline
\hline 
\multicolumn{2}{|c|}{Actor} & \multicolumn{1}{p{10cm}||}{Usuario de Ventas, Administrador}\tabularnewline
\hline 
\multicolumn{2}{|c|}{Propósito} & \multicolumn{1}{p{10cm}||}{Registrar una venta en el sistema}\tabularnewline
\hline 
\multicolumn{2}{|c|}{Entradas} & \multicolumn{1}{p{10cm}||}{Ninguna**}\tabularnewline
\hline 
\multicolumn{2}{|c|}{Salidas} & \multicolumn{1}{p{10cm}||}{**}\tabularnewline
\hline 
\multicolumn{2}{|c|}{Pre-condiciones} & \multicolumn{1}{p{10cm}||}{El usuario de ventas debe estar autentificado, El cliente no debe existir}\tabularnewline
\hline 
\multicolumn{2}{|c|}{Pos-condiciones} & \multicolumn{1}{p{10cm}||}{**}\tabularnewline
\hline 
\multicolumn{2}{|c|}{Errores} & \multicolumn{1}{p{10cm}||}{MSG2 Mensaje de error, MSG4 Error al conectar con la base de datos.}\tabularnewline
\hline 
\multicolumn{2}{|c|}{Tipo} & \multicolumn{1}{p{10cm}||}{Secundario}\tabularnewline
\hline 
\multicolumn{2}{|c|}{Fuente} & \multicolumn{1}{p{10cm}||}{Basado en el funcionamiento etc.}\tabularnewline
\hline 
\end{tabular}
\par\end{centering}

\caption{Caso de Uso 4.0: Gestionar Ventas}
\label{tab:CasosdeUso:nombredecasodeuso} 
\end{table}


% = = = = = = = = = = = = = = = = = = = = = = = = =
%	FIN DE TABLA
% = = = = = = = = = = = = = = = = = = = = = = = = =
% = = = = = = = = = = = = = = = = = = = = = = = = =
%	INICIO DE TRAYECTORIA
% = = = = = = = = = = = = = = = = = = = = = = = = = 


\textbf{\large Trayectorias del CU}{\large \par}

\textit{\large Trayectoria Principal}{\large {} }{\large \par}

%Solo hay que cambiar el nombre de la imagen dependiendo de si es actor o sistema

\begin{tabular}{ccl}
 &  & \tabularnewline
1.  & \includegraphics{actor}  & \multicolumn{1}{p{12cm}}{ Presiona el ícono {[}Alta Venta{]} en la pantalla PV2}\tabularnewline
2.  & \includegraphics{sistema}  & \multicolumn{1}{p{12cm}}{Despliega la pantalla PV3 con el formulario a llenar.}\tabularnewline
3.  & \includegraphics{actor}  & \multicolumn{1}{p{12cm}}{Llena los datos del formulario}\tabularnewline
4.  & \includegraphics{actor}  & \multicolumn{1}{p{12cm}}{Da clic en el botón de aceptar{[}Trayectoria Alternativa A{]}}\tabularnewline
5.  & \includegraphics{sistema}  & \multicolumn{1}{p{12cm}}{Verifica si los campos se llenaron correctamente{[}Trayectoria Alternativa B{]}}\tabularnewline
 &  & \multicolumn{1}{p{12cm}}{.... Fin del caso de uso}\tabularnewline
\end{tabular}

\textit{Trayectoria Alternativa A}

Condición: No se pudieron cargar los datos de las Ventas anteriores

\begin{tabular}{ccl}
 &  & \tabularnewline
1.  & \includegraphics{sistema}  & \multicolumn{1}{p{12cm}}{ Muestra el mensaje de error MSG4 'No se pudo establecer conexión
con la base de datos'}\tabularnewline
2.  & \includegraphics{sistema}  & \multicolumn{1}{p{12cm}}{Regresa a la pantalla PV1}\tabularnewline
 &  & \multicolumn{1}{p{12cm}}{.... Fin del caso de uso}\tabularnewline
\end{tabular}

\newpage{}


\section{Mensajes}


\subsection{MSG1 Mensaje de confirmación.}

\noindent Tipo: Notificación.

\noindent Estatus: Propuesta.

\noindent Objetivo: Indicar al usuario que la acción solicitada fue
realizada exitosamente.

\noindent Redacción: ARTICULO | ENTIDAD | VALOR ha sido OPERACIÓN
exitosamente.

\noindent Parámetros: El mensaje se muestra con base en lo siguiente
: 
\begin{description}
\item [{ARTICULO:}] Tiene la función de limitar la extensión de la entidad. 
\item [{ENTIDAD:}] Producto | Ingrediente | Proveedor | Cliente | Empleado
| Reporte 
\item [{VALOR:}] \noindent Valor de la entidad a la cual se le realiza
una operación. 
\item [{OPERACIÓN}] \noindent Agregar | Eliminar | Modificar | Generar 
\item [{Ejemplo:}] \noindent El producto <<Arcoiris>> ha sido eliminado
exitosamente. 
\end{description}

\subsection{MSG2 Mensaje de error.}

\noindent Tipo: Notificación.

\noindent Estatus: Propuesta.

\noindent Objetivo: Indicar al usuario que ocurrio un error y no se
pudo realizar la acción solicitada.

\noindent Redacción: Error al intentar OPERACIÓN ARTICULO | ENTIDAD
| VALOR

\noindent Parámetros: El mensaje se muestra con base en lo siguiente
: 
\begin{description}
\item [{ARTICULO:}] Tiene la función de limitar la extensión de la entidad. 
\item [{ENTIDAD:}] Producto | Ingrediente | Proveedor | Cliente | Empleado
| Reporte 
\item [{VALOR:}] \noindent Valor de la entidad a la cual se le realiza
una operación. 
\item [{OPERACIÓN}] \noindent Agregar | Eliminar | Modificar | Generar 
\item [{Ejemplo:}] \noindent Error al intentar modificar el producto <<Arcoiris>>. 
\end{description}

\subsection{MSG3 Datos no encontrados.}

\noindent Tipo: Notificación.

\noindent Estatus: Propuesta.

\noindent Objetivo: Indicar al usuario que los datos solicitados no
arrojarón resultados.

\noindent Redacción: ARTICULO ENTIDAD VALOR no muestra ningún resultado.

\noindent Parámetros: El mensaje se muestra con base en lo siguiente
: 
\begin{description}
\item [{ARTICULO:}] Tiene la función de limitar la extensión de la entidad. 
\item [{ENTIDAD:}] Producto | Ingrediente | Proveedor | Cliente | Empleado
| Venta | Compra 
\item [{VALOR:}] \noindent Valor de la entidad. 
\item [{Ejemplo:}] \noindent Las ventas del 03/Mar/2011 al 03/Abril/2011
no muestran ningún resultado. 
\end{description}

\subsection{MSG4 Error al conectar con la base de datos.}

\noindent Tipo: Notificación.

\noindent Estatus: Propuesta.

\noindent Objetivo: Indicar al usuario que no se obtuvo conexión con
la base de datos.

\noindent Redacción: No se pudo establecer conexión con la base de
datos.


\subsection{MSG5 Error al iniciar sesión.}

\noindent Tipo: Notificación.

\noindent Estatus: Propuesta.

\noindent Objetivo: Notificar al usuario un error al iniciar sesión.

\noindent Redacción: ARTICULO ENTIDAD VALOR incorrecto.

\noindent Parámetros: El mensaje se muestra con base en lo siguiente
: 
\begin{description}
\item [{ARTICULO:}] Tiene la función de limitar la extensión de la entidad. 
\item [{ENTIDAD:}] Usuario | Contraseña 
\item [{VALOR:}] \noindent Valor de la entidad. 
\item [{Ejemplo:}] \noindent La contraseña es incorrecta. 
\end{description}

\subsection{MSG6 Ingredientes insuficientes.}

\noindent Tipo: Notificación.

\noindent Estatus: Propuesta.

\noindent Objetivo: Notificar al usuario un error al iniciar sesión.

\noindent Redacción: ARTICULO cantidad ENTIDAD VALOR solicitada es
insuficiente.

\noindent Parámetros: El mensaje se muestra con base a la BR1 atendiendo
a lo siguiente : 
\begin{description}
\item [{ARTICULO:}] Tiene la función de limitar la extensión de la entidad. 
\item [{ENTIDAD:}] Limita la unidad de medición. 
\item [{VALOR:}] \noindent insuficiente 
\item [{Ejemplo:}] \noindent La cantidad de litros de leche solicitada
es insuficiente. 
\end{description}

\subsection{MSG7 Formato incorrecto de los datos.}

\noindent Tipo: Notificación.

\noindent Estatus: Propuesta.

\noindent Objetivo: Notificar al usuario un error al iniciar sesión.

\noindent Redacción: ARTICULO ENTIDAD sólo acepta TIPO.

\noindent Parámetros: El mensaje se muestra con base a la BR1 atendiendo
a lo siguiente : 
\begin{description}
\item [{ARTICULO:}] Tiene la función de limitar la extensión de la entidad. 
\item [{ENTIDAD:}] Nombre del campo incorrecto. 
\item [{TIPO:}] \noindent Tipo de datos indicado en el diccionario de datos
para cada entidad. 
\item [{Ejemplo:}] \noindent El precio sólo acepta números. 
\end{description}

\section{Reglas de Negocio}


\subsection{BR1 Elaboración de lotes}

\noindent Tipo: Restricción.

\noindent Nivel: Obligatorio. Debe cumplirse siempre.

\noindent Versión: 1.0

\noindent Estado: Propuesta.

\noindent Descripción: Para poder elaborar un producto las cantidades
de cada ingrediente indicadas en la receta deben existir en el almacen
y deben provenir de un mismo proveedor,esto con el fin de facilitar
el control de calidad.

\noindent Ejemplo:

\noindent %
\begin{tabular}{|c|}
\hline 
\multicolumn{1}{|c|}{Se necesitan 10 litros de leche}\tabularnewline
\hline 
\end{tabular}%
\begin{tabular}{|c|}
\hline 
\multicolumn{1}{|c|}{Se tienen 8 litros del proveedor A y 2 litros del proveedor B}\tabularnewline
\hline 
\end{tabular}

\noindent El producto no podría elaborarse porque no existe la cantidad
de materia prima suficiente proveniente de un mismo proveedor.


\subsection{BR2 Forma de pago}

\noindent Tipo: Definición

\noindent Nivel: Obligatorio

\noindent Versión:1.0

\noindent Estado: Propuesta

\noindent Descripción: Se debe indicar en toda venta realizada la
forma de pago, y en caso de ser necesario, los ultimos cuatro dgitos
del numero de cuenta/ tarjeta por cuestiones fiscales.


\subsection{BR3 Límite de almacén}

\noindent Tipo: Restrictiva.

\noindent Nivel: Obligatorio

\noindent Versión:1.0

\noindent Estado: Propuesta

\noindent Descripción: La cantidad de galletas que se pueden almacenar
en el almacen estan limitas a 100,000 cajas. Si se llega a esta límite
debe detenerse la producción.


\subsection{BR4 Cliente prioritario}

\noindent Tipo: Definición.

\noindent Nivel: Obligatorio

\noindent Versión:1.0

\noindent Estado: Propuesta

\noindent Descripción: Si un pedido tiene el caracter de compensación
de producto (por alguna falla reportada por el cliente o problema
que se presente durante o despues de su elaboración), este debe tener
prioridad para su entrega, sobre cualquier otra entrega a efectuarse.


\subsection{BR5 Formato de los reportes}

\noindent Tipo: Definición.

\noindent Nivel: Obligatorio

\noindent Versión:1.0

\noindent Estado: Propuesta

\noindent Descripción: Todo reporte debe indicar con el membrete de
la empresa, deberá indicar el área de la que proviene, fecha de elaboración
, periodo que abarca y nombre de quien lo elaboró.

% = = = = = = = = = = = = = = = = = = = = = = = = =
%	FIN DEL DOCUMENTO
% = = = = = = = = = = = = = = = = = = = = = = = = =

\end{document}
