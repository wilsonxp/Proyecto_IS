
\documentclass[10pt,spanish]{article}
\usepackage[utf8x]{inputenc}
\usepackage[letterpaper]{geometry}
\geometry{verbose}
\usepackage{array}
\usepackage{longtable}
\usepackage{amsmath}
\usepackage{amssymb}

\makeatletter

\usepackage{array}

\providecommand{\tabularnewline}{\\}

\usepackage{ucs}\usepackage[spanish]{babel}
\usepackage{amsfonts}\usepackage{colortbl}
% = = = = = = = = = = = = = = = = = = = = = = = = =
% INICIO EL DOCUMENTO
% = = = = = = = = = = = = = = = = = = = = = = = = =
\usepackage{babel}
\addto\shorthandsspanish{\spanishdeactivate{~<>}}

\begin{document}

\section{Revisión de Trayectorias: Modulo de Compras}

\subsection{CUCO01: Gestionar Proveedores}
Numero de Errores: 9\tabularnewline
Resumen: Explica bien que realiza el caso de uso.

% = = = = = = = = = = = = = = = = = = = = = = = = =
%	INICIO DE TABLA
% = = = = = = = = = = = = = = = = = = = = = = = = = 


\begin{center}
\begin{longtable}{|rp{8cm}|}
\hline 
\textbf{Caso de uso:}  & CUCO01: Gestionar Proveedores\tabularnewline
\hline 
\multicolumn{2}{|>{\columncolor[gray]{0.7}}c}{Resumen de Atributos}\tabularnewline
\hline 
Realizador por:  & Ricardo Jimenez Navarro\tabularnewline
\hline 
Revisado por:  & Diaz Rodriguez Kevin Alan\tabularnewline
\hline 
Próposito:  & Sin Observaciones.\tabularnewline
\hline 
Entradas:  & Debe ir 'Sesion Iniciada', no la propuesta.\tabularnewline
\hline 
Salidas:  & Sin Observaciones.\tabularnewline
\hline 
Pre-Condiciones  & Sin Observaciones.\tabularnewline
\hline 
Pos-Condiciones  & No se definio esta Pos-Condicion  
\begin{itemize}
\item La base de datos debe estar disponible.\end{itemize}
\tabularnewline
\hline 
Errores:  & No se definio este mensaje de error. (Puede existir, ya que al inicio cargan los 5 primeros proveedores). 
\begin{itemize}
\item MJ3: No hay conexion a la base de datos.\end{itemize}
\tabularnewline
\hline  
\end{longtable}
\par\end{center}

% = = = = = = = = = = = = = = = = = = = = = = = = =
%	FIN DE TABLA
% = = = = = = = = = = = = = = = = = = = = = = = = =
% = = = = = = = = = = = = = = = = = = = = = = = = =
%	INICIO DE TRAYECTORIA
% = = = = = = = = = = = = = = = = = = = = = = = = =


\textit{\large Trayectoria Principal}{\large {} }{\large \par}

Observaciones trayectoria principal. %
\begin{longtable}{rp{8cm}}
No.  & Observación\tabularnewline
3. & El boton de la pantalla PC1 dice Agregar Proveedor y el CU dice Registrar Proveedor.\tabularnewline
....  &  Debe decir 'fin del caso de uso', NO 'fin de trayectoria'.\tabularnewline
\end{longtable}

\textit{Trayectoria Alternativa A}

Condición: La condicion no es la adecuada.

\begin{longtable}{rp{8cm}}
No.  & Observación\tabularnewline
1.  & Este paso debe ser la condicion de la trayectoria y comenzar con el paso 2.\tabularnewline
\end{longtable}

\textit{Trayectoria Alternativa B}

Condición: La condicion no es la adecuada.

\begin{longtable}{rp{8cm}}
No.  & Observación\tabularnewline
1.  & Este paso debe ser la condicion de la trayectoria y comenzar con el paso 2.\tabularnewline
\end{longtable}% = = = = = = = = = = = = = = = = = = = = = = = = =
%	FIN DE TRAYECTORIA
% = = = = = = = = = = = = = = = = = = = = = = = = =
\newpage{}


\subsection{CUCO01.1: Agregar Proveedores}
Numero de Errores: 11\tabularnewline
Resumen: Debe cambiar 'Registrar Proveedores' a ' Agregar Proveedores'.

% = = = = = = = = = = = = = = = = = = = = = = = = =
%	INICIO DE TABLA
% = = = = = = = = = = = = = = = = = = = = = = = = = 


\begin{center}
\begin{longtable}{|rp{8cm}|}
\hline 
\textbf{Caso de uso:}  & CUCO01.1: Agregar Proveedores\tabularnewline
\hline 
\multicolumn{2}{|>{\columncolor[gray]{0.7}}c}{Resumen de Atributos}\tabularnewline
\hline 
Realizador por:  & Ricardo Jimenez Navarro\tabularnewline
\hline 
Revisado por:  & Diaz Rodriguez Kevin Alan\tabularnewline
\hline 
Próposito:  & Sin Observaciones.\tabularnewline
\hline 
Entradas:  & Seria mejor solo poner 'LLenar formulario', en lugar de todos los puntos.\tabularnewline
\hline 
Salidas:  & Sin Observaciones.\tabularnewline
\hline 
Pre-Condiciones  & No debe ir la Pre-Condicion propuesta.
\begin{itemize}
\item El actor debe estar autentifcado como Administrador de Compras.(No debe de ir ya que al entrar a esta opcion el actor previamente se autentifico.
\item No se definio esta Pre-Condicion: 'El proveedor no debe existir', ya que
podria intentar registrarlo dos veces.\end{itemize}
\tabularnewline
\hline 
Pos-Condiciones  & No se definio esta Pos-Condicion  
\begin{itemize}
\item La base de datos debe estar disponible.\end{itemize}
\tabularnewline
\hline 
Errores:  & En errores se debe de poner el codigo de cada
mensaje de error, esto solo para su mejor entendimiento.\tabularnewline
\hline  
\end{longtable}
\par\end{center}

% = = = = = = = = = = = = = = = = = = = = = = = = =
%	FIN DE TABLA
% = = = = = = = = = = = = = = = = = = = = = = = = =
% = = = = = = = = = = = = = = = = = = = = = = = = =
%	INICIO DE TRAYECTORIA
% = = = = = = = = = = = = = = = = = = = = = = = = =


\textit{\large Trayectoria Principal}{\large {} }{\large \par}

Observaciones trayectoria principal. %
\begin{longtable}{rp{8cm}}
No.  & Observación\tabularnewline
1. & Ya se realizo en el CUCO01.\tabularnewline
2. & Ya se realizo en el CUCO01.\tabularnewline
3. & Debe comenzar en este paso, ya que es tu continuacion (extends) del caso CUCO01.\tabularnewline
\end{longtable}

\textit{Trayectoria Alternativa C}

Condición: Sin Observaciones.

\begin{longtable}{rp{8cm}}
No.  & Observación\tabularnewline
2.  & Este paso es igual al paso 3.\tabularnewline
3.  & Este paso es igual al paso 2.\tabularnewline
\end{longtable}

\textit{Trayectoria Alternativa A,B,D}
: Sin Observaciones. 
\newpage{}


\subsection{CUCO01.2: Modificar Proveedores}
Numero de Errores: 10\tabularnewline
Resumen: Explica bien que realiza el caso de uso.
% = = = = = = = = = = = = = = = = = = = = = = = = =
%	INICIO DE TABLA
% = = = = = = = = = = = = = = = = = = = = = = = = = 


\begin{center}
\begin{longtable}{|rp{8cm}|}
\hline 
\textbf{Caso de uso:}  & CUCO01.2: Modificar Proveedores\tabularnewline
\hline 
\multicolumn{2}{|>{\columncolor[gray]{0.7}}c}{Resumen de Atributos}\tabularnewline
\hline 
Realizador por:  & Ricardo Jimenez Navarro\tabularnewline
\hline 
Revisado por:  & Diaz Rodriguez Kevin Alan\tabularnewline
\hline 
Próposito:  & Sin Observaciones.\tabularnewline
\hline 
Entradas:  & Solo debe ir la primera entrada'El actor teclea el parametro de busqueda del proveedor', la segunda entrada ya se realiza en la trayectoria principal.\tabularnewline
\hline 
Salidas:  & Sin Observaciones.\tabularnewline
\hline 
Pre-Condiciones  & Solo debe ir la Pre-Condicion 2.
\begin{itemize}
\item El proveedor del cual se quiere modifcar la informacion debe
existir en el sistema.\end{itemize}
\tabularnewline
\hline 
Pos-Condiciones  & No se definio esta Pos-Condicion  
\begin{itemize}
\item La base de datos debe estar disponible.\end{itemize}
\tabularnewline
\hline 
Errores:  & 
\begin{itemize}
\item En errores se debe de poner el codigo de cada
mensaje de error, esto solo para su mejor entendimiento.
\item No se definio el mensaje de error MJ5 'No hay informacion'.\end{itemize}
\tabularnewline
\hline
\end{longtable}
\par\end{center}

% = = = = = = = = = = = = = = = = = = = = = = = = =
%	FIN DE TABLA
% = = = = = = = = = = = = = = = = = = = = = = = = =
% = = = = = = = = = = = = = = = = = = = = = = = = =
%	INICIO DE TRAYECTORIA
% = = = = = = = = = = = = = = = = = = = = = = = = =


\textit{\large Trayectoria Principal}{\large {} }{\large \par}

Observaciones trayectoria principal. %
\begin{longtable}{rp{8cm}}
No.  & Observación\tabularnewline
1. & Debe decir el codigo de la pantalla en este caso PC0. Debe contener la Trayectoria alternativa D.\tabularnewline
5. & La Trayectoria alternativa E no debe ir, ya que previamente lo busco en el paso 2, y si llego al paso 5 quiere decir que si lo encontro. \tabularnewline
6. & ¿Cual es la PANTALLA2?.\tabularnewline
\end{longtable}

\textit{Trayectoria Alternativa E}

Condición: Error de redaccion.\tabularnewline

\textit{Trayectoria Alternativa A,B,C,D}: Sin Observaciones. 

\newpage{}


\subsection{CUCO01.3: Eliminar Proveedores}
Numero de Errores: 8\tabularnewline
Resumen: Explica bien que realiza el caso de uso.
% = = = = = = = = = = = = = = = = = = = = = = = = =
%	INICIO DE TABLA
% = = = = = = = = = = = = = = = = = = = = = = = = = 


\begin{center}
\begin{longtable}{|rp{8cm}|}
\hline 
\textbf{Caso de uso:}  & CUCO01.3: Eliminar Proveedores\tabularnewline
\hline 
\multicolumn{2}{|>{\columncolor[gray]{0.7}}c}{Resumen de Atributos}\tabularnewline
\hline 
Realizador por:  & Ricardo Jimenez Navarro\tabularnewline
\hline 
Revisado por:  & Diaz Rodriguez Kevin Alan\tabularnewline
\hline 
Próposito:  & Sin Observaciones.\tabularnewline
\hline 
Entradas:  & Sin Observaciones.\tabularnewline
\hline 
Salidas:  & Sin Observaciones.\tabularnewline
\hline 
Pre-Condiciones  & Solo debe ir la Pre-Condicion 2.
\begin{itemize}
\item El proveedor que se quiere eliminar debe existir en el sistema.\end{itemize}
\tabularnewline
\hline 
Pos-Condiciones  & No se definio esta Pos-Condicion  
\begin{itemize}
\item La base de datos debe estar disponible.\end{itemize}
\tabularnewline
\hline 
Errores:  & 
\begin{itemize}
\item En errores se debe de poner el codigo de cada
mensaje de error, esto solo para su mejor entendimiento.
\item No se definio el mensaje de error MJ5 'No hay informacion'.\end{itemize}
\tabularnewline
\hline
\end{longtable}
\par\end{center}

% = = = = = = = = = = = = = = = = = = = = = = = = =
%	FIN DE TABLA
% = = = = = = = = = = = = = = = = = = = = = = = = =
% = = = = = = = = = = = = = = = = = = = = = = = = =
%	INICIO DE TRAYECTORIA
% = = = = = = = = = = = = = = = = = = = = = = = = =


\textit{\large Trayectoria Principal}{\large {} }{\large \par}

Observaciones trayectoria principal. %
\begin{longtable}{rp{8cm}}
No.  & Observación\tabularnewline
1. & Debe decir el codigo de la pantalla en este caso PC0. Debe contener la Trayectoria alternativa B.\tabularnewline
5. & La Trayectoria alternativa C no debe ir, ya que previamente lo busco en el paso 2, y si llego al paso 5 quiere decir que si lo encontro. \tabularnewline
\end{longtable}

\textit{Trayectoria Alternativa C}

Condición: Error de redaccion.\tabularnewline

\textit{Trayectoria Alternativa A,B}

Sin Observaciones.
\end{document}
