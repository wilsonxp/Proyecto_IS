
\documentclass[10pt,spanish]{article}
\usepackage[utf8x]{inputenc}
\usepackage[letterpaper]{geometry}
\geometry{verbose}
\usepackage{array}
\usepackage{longtable}
\usepackage{amsmath}
\usepackage{amssymb}

\makeatletter

\usepackage{array}

\providecommand{\tabularnewline}{\\}

\usepackage{ucs}\usepackage[spanish]{babel}
\usepackage{amsfonts}\usepackage{colortbl}
% = = = = = = = = = = = = = = = = = = = = = = = = =
% INICIO EL DOCUMENTO
% = = = = = = = = = = = = = = = = = = = = = = = = =
\usepackage{babel}
\addto\shorthandsspanish{\spanishdeactivate{~<>}}

\begin{document}


\section{Revisión de Trayectorias}

Resumen: El resumen es entendible y describe la acción que del CU.


\subsection{CUAD3.2: Cambiar contraseña}

% = = = = = = = = = = = = = = = = = = = = = = = = =
%	INICIO DE TABLA
% = = = = = = = = = = = = = = = = = = = = = = = = = 


\begin{center}
\begin{longtable}{|rp{8cm}|}
\hline 
\textbf{Caso de uso:}  & CUAD3.2: Cambiar contraseña\tabularnewline
\hline 
\multicolumn{2}{|>{\columncolor[gray]{0.7}}c}{Resumen de Atributos}\tabularnewline
\hline 
Realizador por:  & Irvin Ortiz Ochoa\tabularnewline
\hline 
Revisado por:  & Consuelo Mayén Christian Armando\tabularnewline
\hline 
Próposito:  & El propósito es simple y se comprende perfectamente.\tabularnewline
\hline 
Entradas:  & Anoteaciones de las entradas
\begin{itemize}
\item No especifican a qué se refieren con "login y password". 
\item  Login no podria considerarse como una endrada.\end{itemize}
\tabularnewline
\hline 
Salidas:  & Anotaciones de las salidas 
\begin{itemize}
\item No tengo anotaciones acerca de las salidas. 
\end{itemize}
\tabularnewline
\hline 
Pre-Condiciones  & Anotaciones de las precondiciones 
\begin{itemize}
\item Lo único que cambiaria sería que dijera que el usuario se ha identificado correctamente.\end{itemize}
\tabularnewline
\hline 
Pos-Condiciones  & Anotaciones de las poscondiciones 
\begin{itemize}
\item No hay anotaciones acerca de las poscondiciones.\end{itemize}
\tabularnewline
\hline 
Errores:  & Anotaciones de los errores 
\begin{itemize}
\item No tienen definido ningún tipo de error.\end{itemize}
\tabularnewline
\hline 
\end{longtable}
\par\end{center}

% = = = = = = = = = = = = = = = = = = = = = = = = =
%	FIN DE TABLA
% = = = = = = = = = = = = = = = = = = = = = = = = =
% = = = = = = = = = = = = = = = = = = = = = = = = =
%	INICIO DE TRAYECTORIA
% = = = = = = = = = = = = = = = = = = = = = = = = =

\newpage

\textit{\large Trayectoria Principal}{\large {} }{\large \par}

Observaciones trayectoria principal. %
\begin{longtable}{rp{8cm}}
No.  & Observación\tabularnewline
1.  &No hay observaciones para la trayectoria principal. \tabularnewline
\end{longtable}

\textit{Trayectoria Alternativa A}

Condición: ¿Esta bien la condición?

\begin{longtable}{rp{8cm}}
No.  & Observación\tabularnewline
1.  & Suena un poco redundante que el sistema te muestre la pantalla de "Cancelar operación".
\end{longtable}% = = = = = = = = = = = = = = = = = = = = = = = = =
%	FIN DE TRAYECTORIA
% = = = = = = = = = = = = = = = = = = = = = = = = =
\newpage{} 
\end{document}
