
\documentclass[10pt,spanish]{article}
\usepackage[utf8x]{inputenc}
\usepackage[letterpaper]{geometry}
\geometry{verbose}
\usepackage{array}
\usepackage{longtable}
\usepackage{amsmath}
\usepackage{amssymb}

\makeatletter

\usepackage{array}

\providecommand{\tabularnewline}{\\}

\usepackage{ucs}\usepackage[spanish]{babel}
\usepackage{amsfonts}\usepackage{colortbl}
% = = = = = = = = = = = = = = = = = = = = = = = = =
% INICIO EL DOCUMENTO
% = = = = = = = = = = = = = = = = = = = = = = = = =
\usepackage{babel}
\addto\shorthandsspanish{\spanishdeactivate{~<>}}

\begin{document}


\section{Revisión de Trayectorias}

Resumen: Deberia ser más especifico con el caso de uso, ya que es una función que no describen cómo es que va a funcionar el Email.


\subsection{CUAD3.1: Recordar contraseña}

% = = = = = = = = = = = = = = = = = = = = = = = = =
%	INICIO DE TABLA
% = = = = = = = = = = = = = = = = = = = = = = = = = 


\begin{center}
\begin{longtable}{|rp{8cm}|}
\hline 
\textbf{Caso de uso:}  &Recordar contrase~na\tabularnewline
\hline 
\multicolumn{2}{|>{\columncolor[gray]{0.7}}c}{Resumen de Atributos}\tabularnewline
\hline 
Realizador por:  & Adriana Navarro Maga~na\tabularnewline
\hline 
Revisado por:  & Consuelo Mayén Christian Armando\tabularnewline
\hline 
Próposito:  & El propósito está claro y bien explicado.\tabularnewline
\hline 
Entradas:  & Anotaciones de las entradas 
\begin{itemize}
\item Podrían definir como entrada el nombre del usuario, porque "login" no es un valor de entrada. \end{itemize}
\tabularnewline
\hline 
Salidas:  & Anotaciones de las salidas 
\begin{itemize}
\item No tengo anotaciones que hacerle las salidas.\end{itemize}
\tabularnewline
\hline 
Pre-Condiciones  & Anotaciones de las precondiciones 
\begin{itemize}
\item Podria decirse que una precondición es que el usuario exista, para que tenga una contraseña que recordar. \end{itemize}
\tabularnewline
\hline 
Pos-Condiciones  & Anotaciones de las poscondiciones 
\begin{itemize}
\item No tengo anotaciones que hacer a las poscondicion. \end{itemize}
\tabularnewline
\hline 
Errores:  & Anotaciones de los errores 
\begin{itemize}
\item No tienen ningún apartado donde defina los errores. \end{itemize}
\tabularnewline
\hline 
\end{longtable}
\par\end{center}

% = = = = = = = = = = = = = = = = = = = = = = = = =
%	FIN DE TABLA
% = = = = = = = = = = = = = = = = = = = = = = = = =
% = = = = = = = = = = = = = = = = = = = = = = = = =
%	INICIO DE TRAYECTORIA
% = = = = = = = = = = = = = = = = = = = = = = = = =

\newpage

\textit{\large Trayectoria Principal}{\large {} }{\large \par}

Observaciones trayectoria principal.%
\begin{longtable}{rp{8cm}}
No.  & Observación\tabularnewline
1.  &En el caso de uso anterior definen que se entra al CU por errores en ingreso de datos por parte del usuarios, pero aquí definen que se entra a travez de uno botón.\tabularnewline
8.  &Creo que aquí están definiendo que se valide después de idenficiar al usuario. Creo que deberían validar los datos de entrada antes que nada.\tabularnewline
\end{longtable}

\textit{Trayectoria Alternativa A}

Condición: Está correctamente explicada la condición.

\begin{longtable}{rp{8cm}}
No.  & Observación\tabularnewline
1.  & No tengo anotaciones para la trayectoria A.\tabularnewline

\end{longtable}% = = = = = = = = = = = = = = = = = = = = = = = = =
%	FIN DE TRAYECTORIA
% = = = = = = = = = = = = = = = = = = = = = = = = =

\textit{Trayectoria Alternativa B}

Condición: Tiene un error al decir "La contraseña  es inválidos".

\begin{longtable}{rp{8cm}}
No.  & Observación\tabularnewline
1.  & Nunca definen las razones por las que la contraseña podria ser invalida.\tabularnewline

\end{longtable}% = = = = = = = = = = = = = = = = = = = = = = = = =
%	FIN DE TRAYECTORIA
% = = = = = = = = = = = = = = = = = = = = = = = = =

\textit{Trayectoria Alternativa C}

Condición: Tiene un error al decir "El usuario  es inválidos".

\begin{longtable}{rp{8cm}}
No.  & Observación\tabularnewline
1.  &Ésta trayectoria no debería existir, pues el usuario ya debería estar identificado al querer restarurar su contraseña.\tabularnewline

\end{longtable}% = = = = = = = = = = = = = = = = = = = = = = = = =
%	FIN DE TRAYECTORIA
% = = = = = = = = = = = = = = = = = = = = = = = = =
\newpage{} 
\end{document}
