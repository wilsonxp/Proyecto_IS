
\documentclass[10pt,spanish]{article}
\usepackage[utf8x]{inputenc}
\usepackage[letterpaper]{geometry}
\geometry{verbose}
\usepackage{array}
\usepackage{longtable}
\usepackage{amsmath}
\usepackage{amssymb}

\makeatletter

\usepackage{array}

\providecommand{\tabularnewline}{\\}

\usepackage{ucs}\usepackage[spanish]{babel}
\usepackage{amsfonts}\usepackage{colortbl}
% = = = = = = = = = = = = = = = = = = = = = = = = =
% INICIO EL DOCUMENTO
% = = = = = = = = = = = = = = = = = = = = = = = = =
\usepackage{babel}
\addto\shorthandsspanish{\spanishdeactivate{~<>}}

\begin{document}

\section{Revisión de Trayectorias}

Resumen: El resumen es sencillo y describe el funcionamiento del CU.


\subsection{CUAD3: Entrar al sistema}

% = = = = = = = = = = = = = = = = = = = = = = = = =
%	INICIO DE TABLA
% = = = = = = = = = = = = = = = = = = = = = = = = = 


\begin{center}
\begin{longtable}{|rp{8cm}|}
\hline 
\textbf{Caso de uso:}  & Entrar al sistema\tabularnewline
\hline 
\multicolumn{2}{|>{\columncolor[gray]{0.7}}c}{Resumen de Atributos}\tabularnewline
\hline 
Realizador por:  & Irvin Ortiz Ochoa\tabularnewline
\hline 
Revisado por:  & Consuelo Mayén Christian Armando\tabularnewline
\hline 
Próposito:  & El propósito está claro, aunque creo que debería explicarlo un poco más.\tabularnewline
\hline 
Entradas:  & Anotaciones de las entradas 
\begin{itemize}
\item Puesto que son entradas sencillas, están bien explicadas.
\end{itemize}
\tabularnewline
\hline 
Salidas:  & Anotaciones de las salidas 
\begin{itemize}
\item Puede decirse que "Menu de Sesión iniciada" no es una salida, sino un resultado del inicio de sesión. 
\end{itemize}
\tabularnewline
\hline 
Pre-Condiciones  & Anotaciones de las precondiciones 
\begin{itemize}
\item Pues no hay precondiciones, pero tal vez la precondición sería que se entre al sistema.\end{itemize}
\tabularnewline
\hline 
Pos-Condiciones  & Anotaciones de las poscondiciones 
\begin{itemize}
\item No tienen definida ninguna poscondición. \end{itemize}
\tabularnewline
\hline 
Errores:  & Anotaciones de los errores 
\begin{itemize}
\item No tienen definido ningún error. \end{itemize}
\tabularnewline
\hline 
\end{longtable}
\par\end{center}

% = = = = = = = = = = = = = = = = = = = = = = = = =
%	FIN DE TABLA
% = = = = = = = = = = = = = = = = = = = = = = = = =
% = = = = = = = = = = = = = = = = = = = = = = = = =
%	INICIO DE TRAYECTORIA
% = = = = = = = = = = = = = = = = = = = = = = = = =


\textit{\large Trayectoria Principal}{\large {} }{\large \par}

Observaciones trayectoria principal. 
\begin{longtable}{rp{8cm}}
No.  & Observación\tabularnewline
1.  & No hay más anotación que decir que en el número 6 donde se validan los datos
ingresados, deberia ir más arriba (Entre 3 y 4), pues de nada sirve que valide los datos
si ya el usuario ingresó al sistema.
\end{longtable}

\textit{Trayectoria Alternativa A}

Condición: No tiene condición, tal vez denerían ingresar una, porque no se sabe cuándo se entra a esa trayectoria.

\begin{longtable}{rp{8cm}}
No.  & Observación\tabularnewline
1.  & Deberían darle más oportunidades al usuario de equivocarse con su contraseña,
pues como veo, ahí dice que al primer error llama a CU de "Recordar contraseña".\tabularnewline
2.  & El mensaje de error deberia ser otro. El mensaje de "Dato obligatorio vacío no va al caso"\tabularnewline

\end{longtable}% = = = = = = = = = = = = = = = = = = = = = = = = =
%	FIN DE TRAYECTORIA
% = = = = = = = = = = = = = = = = = = = = = = = = =

\textit{Trayectoria Alternativa B}

Condición: No tiene condición, tal vez denerían ingresar una, porque no se sabe cuándo se entra a esa trayectoria.

\begin{longtable}{rp{8cm}}
No.  & Observación\tabularnewline
1.  & No hay anotaciones para ésta trayectoria.\tabularnewline

\end{longtable}% = = = = = = = = = = = = = = = = = = = = = = = = =
%	FIN DE TRAYECTORIA
% = = = = = = = = = = = = = = = = = = = = = = = = =
\newpage{} 
\end{document}
