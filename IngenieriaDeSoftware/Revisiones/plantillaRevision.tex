
\documentclass[10pt,spanish]{article}
\usepackage[utf8x]{inputenc}
\usepackage[letterpaper]{geometry}
\geometry{verbose}
\usepackage{array}
\usepackage{longtable}
\usepackage{amsmath}
\usepackage{amssymb}

\makeatletter

\usepackage{array}

\providecommand{\tabularnewline}{\\}

\usepackage{ucs}\usepackage[spanish]{babel}
\usepackage{amsfonts}\usepackage{colortbl}
% = = = = = = = = = = = = = = = = = = = = = = = = =
% INICIO EL DOCUMENTO
% = = = = = = = = = = = = = = = = = = = = = = = = =
\usepackage{babel}
\addto\shorthandsspanish{\spanishdeactivate{~<>}}

\begin{document}
\begin{Large} NOTA: Esta es la plantilla para documentar los errores
de las trayectorias del otro equipo.

\end{Large}


\section{Revisión de Trayectorias}

Resumen: se entiende que hace el cu leyendo el resumen o si esta bien
redactado


\subsection{CUX\#.\#: Nombre del caso de uso}

% = = = = = = = = = = = = = = = = = = = = = = = = =
%	INICIO DE TABLA
% = = = = = = = = = = = = = = = = = = = = = = = = = 


\begin{center}
\begin{longtable}{|rp{8cm}|}
\hline 
\textbf{Caso de uso:}  & Nombre del CU\tabularnewline
\hline 
\multicolumn{2}{|>{\columncolor[gray]{0.7}}c}{Resumen de Atributos}\tabularnewline
\hline 
Realizador por:  & Nombre del otro equipo\tabularnewline
\hline 
Revisado por:  & Nombre de nuestro equipo\tabularnewline
\hline 
Próposito:  & Observaciones de si el proposito es claro o si se entiende que hace
el cu\tabularnewline
\hline 
Entradas:  & Anotaciones de las entradas 
\begin{itemize}
\item Anotacion 1 
\item anotacion 2\end{itemize}
\tabularnewline
\hline 
Salidas:  & Anotaciones de las salidas 
\begin{itemize}
\item Anotacion 1 
\item anotacion 2\end{itemize}
\tabularnewline
\hline 
Pre-Condiciones  & Anotaciones de las precondiciones 
\begin{itemize}
\item Anotacion 1\end{itemize}
\tabularnewline
\hline 
Pos-Condiciones  & Anotaciones de las poscondiciones 
\begin{itemize}
\item Anotacion 1\end{itemize}
\tabularnewline
\hline 
Errores:  & Anotaciones de los errores 
\begin{itemize}
\item Anotacion 1\end{itemize}
\tabularnewline
\hline 
\end{longtable}
\par\end{center}

% = = = = = = = = = = = = = = = = = = = = = = = = =
%	FIN DE TABLA
% = = = = = = = = = = = = = = = = = = = = = = = = =
% = = = = = = = = = = = = = = = = = = = = = = = = =
%	INICIO DE TRAYECTORIA
% = = = = = = = = = = = = = = = = = = = = = = = = =


\textit{\large Trayectoria Principal}{\large {} }{\large \par}

Observaciones trayectoria principal. ( no es necesario escribir toda
la trayectoria solo las obsersavaciones y el numero donde se hace%
\begin{longtable}{rp{8cm}}
No.  & Observación\tabularnewline
1.  & Falta validacioón de base de tados\tabularnewline
8.  & La pantalla no coincide\tabularnewline
11.  & Falta boton\tabularnewline
13.  & Este paso es inecesario\tabularnewline
\end{longtable}

\textit{Trayectoria Alternativa A}

Condición: ¿Esta bien la condición?

\begin{longtable}{rp{8cm}}
No.  & Observación\tabularnewline
1.  & Falta validacioón de base de tados\tabularnewline
8.  & La pantalla no coincide\tabularnewline
11.  & Falta boton\tabularnewline
13.  & Este paso es inecesario\tabularnewline
\end{longtable}% = = = = = = = = = = = = = = = = = = = = = = = = =
%	FIN DE TRAYECTORIA
% = = = = = = = = = = = = = = = = = = = = = = = = =
\newpage{} 
\end{document}
