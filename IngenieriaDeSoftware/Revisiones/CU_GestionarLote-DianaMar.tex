
\documentclass[10pt,spanish]{article}
\usepackage[utf8x]{inputenc}
\usepackage[letterpaper]{geometry}
\geometry{verbose}
\usepackage{array}
\usepackage{longtable}
\usepackage{amsmath}
\usepackage{amssymb}

\makeatletter

\usepackage{array}

\providecommand{\tabularnewline}{\\}

\usepackage{ucs}\usepackage[spanish]{babel}
\usepackage{amsfonts}\usepackage{colortbl}
% = = = = = = = = = = = = = = = = = = = = = = = = =
% INICIO EL DOCUMENTO
% = = = = = = = = = = = = = = = = = = = = = = = = =
\usepackage{babel}
\addto\shorthandsspanish{\spanishdeactivate{~<>}}

\begin{document}


\section{Revisi�n de la trayectoria del CU CUPRO3.0 Gesti�n de lote de producto (Crear, modificar y eliminar)}

\subsection{CUPR03.1: Crear nuevo lote de producto}


Resumen: Se entiende la redaccci�n del caso de uso.



% = = = = = = = = = = = = = = = = = = = = = = = = =
%    INICIO DE TABLA
% = = = = = = = = = = = = = = = = = = = = = = = = = 


\begin{center}
\begin{longtable}{|rp{8cm}|}
\hline 
\textbf{Caso de uso:}  & Crear nuevo lote de producto\tabularnewline
\hline 
\multicolumn{2}{|>{\columncolor[gray]{0.7}}c}{Resumen de Atributos}\tabularnewline
\hline 
Realizado por:  & Luis Alfredo S�nchez Angeles\tabularnewline
\hline 
Revisado por:  & Diana Mar Escoto\tabularnewline
\hline 
Pr�posito:  & El prop�sito es dar de alta un nuevo lote con todas sus caracteristicas, no solo dar de alta un nuevo ID\tabularnewline
\hline 
Entradas:  & Anotaciones de las entradas 
Sin observaciones
\tabularnewline
\hline 
Salidas:  & Anotaciones de las salidas 
Sin observaciones
\tabularnewline
\hline 
Pre-Condiciones  & Anotaciones de las precondiciones 
\begin{itemize}
\item Que el usuario de produccion este identificado en el sistema\end{itemize}
\tabularnewline
\hline 
Post-Condiciones  & Anotaciones de las poscondiciones 
\begin{itemize}
\item No tienen post-condiciones\end{itemize}
\tabularnewline
\hline 
Errores:  & Anotaciones de los errores 
\begin{itemize}
\item Sin observaciones\end{itemize}
\tabularnewline
\hline 
\end{longtable}
\par\end{center}

% = = = = = = = = = = = = = = = = = = = = = = = = =
%	FIN DE TABLA
% = = = = = = = = = = = = = = = = = = = = = = = = =
% = = = = = = = = = = = = = = = = = = = = = = = = =
%	INICIO DE TRAYECTORIA
% = = = = = = = = = = = = = = = = = = = = = = = = =


\textit{\large Trayectoria Principal}{\large {} }{\large \par}

Observaciones trayectoria principal.

\begin{longtable}{rp{8cm}}
No.  & Observaci�n\tabularnewline
-  &Colocar la imagen o la palabra "sistema" para las trayectorias en las que el sistema realiza acciones. \tabularnewline
-.  & Enumerar las pantallas\tabularnewline
4.  & La opci�n "Cancelar" no debe aparecer en la trayectoria principal\tabularnewline
\end{longtable}
\textit{Trayectoria Alternativa A}

Condici�n: No se especifican adecuadamente las condiciones

\begin{longtable}{rp{8cm}}
No.  & Observaci�n\tabularnewline
A2.  & Revisar el item mal colocado para la trayectoria A3\tabularnewline


\end{longtable}

\textit{Trayectoria Alternativa B}

Condici�n: No se especifican adecuadamente las condiciones

\begin{longtable}{rp{8cm}}
No.  & Observaci�n\tabularnewline
B4.  &Especificar que trayectoria a la que regresa (Regresa al paso 4 de la trayectoria principal) \tabularnewline


\end{longtable}

\textit{Trayectoria Alternativa C}

Condici�n: No se especifican adecuadamente las condiciones

\begin{longtable}{rp{8cm}}
No.  & Observaci�n\tabularnewline
-.  & La trayectoria no termina ni regresa a la trayectoria principal\tabularnewline


\end{longtable}

\textit{Trayectoria Alternativa D}

Condici�n: No se especifican adecuadamente las condiciones

\begin{longtable}{rp{8cm}}
No.  & Observaci�n\tabularnewline
D2.  &Especificar que trayectoria a la que regresa (Regresa al paso 4 de la trayectoria principal) \tabularnewline


\end{longtable}
% = = = = = = = = = = = = = = = = = = = = = = = = =
%	FIN DE TRAYECTORIA
% = = = = = = = = = = = = = = = = = = = = = = = = =
\newpage{} 

% = = = = = = = = = = = = = = = = = = = = = = = = =
%    NUEVO CU
% = = = = = = = = = = = = = = = = = = = = = = = = =



\subsection{Revision del CUPRO3.2 Modificar un lote de producto}

Resumen: Se entiende el caso de uso sin embargo el resumen esta un poco mal redactado.



% = = = = = = = = = = = = = = = = = = = = = = = = =
%    INICIO DE TABLA
% = = = = = = = = = = = = = = = = = = = = = = = = = 


\begin{center}
\begin{longtable}{|rp{8cm}|}
\hline 
\textbf{Caso de uso:}  & Modificar un lote de producto\tabularnewline
\hline 
\multicolumn{2}{|>{\columncolor[gray]{0.7}}c}{Resumen de Atributos}\tabularnewline
\hline 
Realizado por:  & Luis Alfredo S�nchez Angeles\tabularnewline
\hline 
Revisado por:  & Diana Mar Escoto\tabularnewline
\hline 
Pr�posito:  & Sin observaciones\tabularnewline
\hline 
Entradas:  & Anotaciones de las entradas 
Sin observaciones
\tabularnewline
\hline 
Salidas:  & Anotaciones de las salidas 
Sin observaciones
\tabularnewline
\hline 
Pre-Condiciones  & Anotaciones de las precondiciones 
\begin{itemize}
\item Sin observaciones\end{itemize}
\tabularnewline
\hline 
Post-Condiciones  & Anotaciones de las poscondiciones 
\begin{itemize}
\item No tienen post-condiciones\end{itemize}
\tabularnewline
\hline 
Errores:  & Anotaciones de los errores 
\begin{itemize}
\item Falta agregar algunos mensajes de error que aparecen en la trayectoria\end{itemize}
\tabularnewline
\hline 
\end{longtable}
\par\end{center}

% = = = = = = = = = = = = = = = = = = = = = = = = =
%    FIN DE TABLA
% = = = = = = = = = = = = = = = = = = = = = = = = =
% = = = = = = = = = = = = = = = = = = = = = = = = =
%	INICIO DE TRAYECTORIA
% = = = = = = = = = = = = = = = = = = = = = = = = =


\textit{\large Trayectoria Principal}{\large {} }{\large \par}

Observaciones trayectoria principal.

\begin{longtable}{rp{8cm}}
No.  & Observaci�n\tabularnewline
-  &Falta indicar en que momento de la trayec \tabularnewline
5.  & La opci�n "Cancelar" no debe aparecer en la trayectoria principal\tabularnewline

\end{longtable}
\textit{Trayectoria Alternativa A, B y C}

Condici�n: No se especifican adecuadamente las condiciones

\textit{Trayectoria Alternativa D}

Condici�n: No se especifican adecuadamente las condiciones

\begin{longtable}{rp{8cm}}
No.  & Observaci�n\tabularnewline
C1.  &Esta linea no pertenece a este caso de uso. \tabularnewline


\end{longtable}


\end{longtable}
% = = = = = = = = = = = = = = = = = = = = = = = = =
%	FIN DE TRAYECTORIA
% = = = = = = = = = = = = = = = = = = = = = = = = =
\newpage{}

\textit{Trayectoria Alternativa C}

Condici�n: No se especifican adecuadamente las condiciones

\begin{longtable}{rp{8cm}}
No.  & Observaci�n\tabularnewline
-.  & La trayectoria no termina ni regresa a la trayectoria principal\tabularnewline


\end{longtable}

\textit{Trayectoria Alternativa D}

Condici�n: No se especifican adecuadamente las condiciones

\begin{longtable}{rp{8cm}}
No.  & Observaci�n\tabularnewline
D2.  &Especificar que trayectoria a la que regresa (Regresa al paso 4 de la trayectoria principal) \tabularnewline


\end{longtable}
% = = = = = = = = = = = = = = = = = = = = = = = = =
%    FIN DE TRAYECTORIA
% = = = = = = = = = = = = = = = = = = = = = = = = =
\newpage{} 

% = = = = = = = = = = = = = = = = = = = = = = = = =
%    NUEVO CU
% = = = = = = = = = = = = = = = = = = = = = = = = =



\subsection{Revision del CUPRO3.3 Eliminar un lote de productos}

Resumen: Se entiende el caso de uso pero es necesario revisar la redacci�n.



% = = = = = = = = = = = = = = = = = = = = = = = = =
%    INICIO DE TABLA
% = = = = = = = = = = = = = = = = = = = = = = = = = 


\begin{center}
\begin{longtable}{|rp{8cm}|}
\hline 
\textbf{Caso de uso:}  & Eliminar un lote de productos\tabularnewline
\hline 
\multicolumn{2}{|>{\columncolor[gray]{0.7}}c}{Resumen de Atributos}\tabularnewline
\hline 
Realizado por:  & Luis Alfredo S�nchez Angeles\tabularnewline
\hline 
Revisado por:  & Diana Mar Escoto\tabularnewline
\hline 
Pr�posito:  & Sin observaciones\tabularnewline
\hline 
Entradas:  & Anotaciones de las entradas 
Sin observaciones
\tabularnewline
\hline 
Salidas:  & Anotaciones de las salidas 
Sin observaciones
\tabularnewline
\hline 
Pre-Condiciones  & Anotaciones de las precondiciones 
\begin{itemize}
\item Sin observaciones\end{itemize}
\tabularnewline
\hline 
Post-Condiciones  & Anotaciones de las poscondiciones 
\begin{itemize}
\item No tienen post-condiciones\end{itemize}
\tabularnewline
\hline 
Errores:  & Anotaciones de los errores 
\begin{itemize}
\item Falta agregar el mensaje de error que aparece en la trayectoria\end{itemize}
\tabularnewline
\hline 
\end{longtable}
\par\end{center}

% = = = = = = = = = = = = = = = = = = = = = = = = =
%    FIN DE TABLA
% = = = = = = = = = = = = = = = = = = = = = = = = =
% = = = = = = = = = = = = = = = = = = = = = = = = =
%	INICIO DE TRAYECTORIA
% = = = = = = = = = = = = = = = = = = = = = = = = =


\textit{\large Trayectoria Principal}{\large {} }{\large \par}

Observaciones trayectoria principal.

\begin{longtable}{rp{8cm}}
No.  & Observaci�n\tabularnewline
-  &Falta indicar el mensaje de operaci�n exitosa al eliminar el lote \tabularnewline
\end{longtable}

% = = = = = = = = = = = = = = = = = = = = = = = = =
%	FIN DE TRAYECTORIA
% = = = = = = = = = = = = = = = = = = = = = = = = =

\end{document}
