
\documentclass[10pt,spanish]{article}
\usepackage[utf8x]{inputenc}
\usepackage[letterpaper]{geometry}
\geometry{verbose}
\usepackage{array}
\usepackage{longtable}
\usepackage{amsmath}
\usepackage{amssymb}

\makeatletter

\usepackage{array}

\providecommand{\tabularnewline}{\\}

\usepackage{ucs}\usepackage[spanish]{babel}
\usepackage{amsfonts}\usepackage{colortbl}
% = = = = = = = = = = = = = = = = = = = = = = = = =
% INICIO EL DOCUMENTO
% = = = = = = = = = = = = = = = = = = = = = = = = =
\usepackage{babel}
\addto\shorthandsspanish{\spanishdeactivate{~<>}}

\begin{document}


\section{CUCO1: Servicio al cliente}

¿El caso de uso lo atiende, analiza la queja y reporta o solo registra la queja?

\subsection{CUCO1: Servicio al cliente}

% = = = = = = = = = = = = = = = = = = = = = = = = =
%	INICIO DE TABLA
% = = = = = = = = = = = = = = = = = = = = = = = = = 


\begin{center}
\begin{longtable}{|rp{8cm}|}
\hline 
\textbf{Caso de uso:}  & CUCO1: Servicio al cliente\tabularnewline
\hline 
\multicolumn{2}{|>{\columncolor[gray]{0.7}}c}{Resumen de Atributos}\tabularnewline
\hline 
Realizador por:  & Cienfuegos Najera Alan Raziel\tabularnewline
\hline 
Revisado por:  & Pérez Alvarez Juan Carlos\tabularnewline
\hline 
Próposito:  & ¿El caso de uso atiende a los clientes o registra el problema solamente?\tabularnewline
\hline 
Entradas:  & Faltan especificar algunas entradas
\tabularnewline
\hline 
Salidas:  & Sin observaciones
\tabularnewline
\hline 
Pre-Condiciones  & Sin observaciones
\tabularnewline
\hline 
Pos-Condiciones  & ¿No hay?
\tabularnewline
\hline 
Errores:  & Sin observaciones
\tabularnewline
\hline 
\end{longtable}
\par\end{center}

% = = = = = = = = = = = = = = = = = = = = = = = = =
%	FIN DE TABLA
% = = = = = = = = = = = = = = = = = = = = = = = = =
% = = = = = = = = = = = = = = = = = = = = = = = = =
%	INICIO DE TRAYECTORIA
% = = = = = = = = = = = = = = = = = = = = = = = = =


\textit{\large Trayectoria Principal}{\large {} }{\large \par}

\begin{longtable}{rp{8cm}}
No.  & Observación\tabularnewline
1.  & ¿En que pantalla esta ese botón?\tabularnewline
2.  & ¿Qué pantalla es? ¿Cuál es la acción PCA - CORREGIR?\tabularnewline
4.  & No hay botón Enviar\tabularnewline
6.  & Valida... ¿Y?\tabularnewline
\end{longtable}

\textit{Trayectoria Alternativa A}

\begin{longtable}{rp{8cm}}
No.  & Observación\tabularnewline
1.  & No hay pantalla PCA01\tabularnewline
\end{longtable}

Total de observaciones de este CU: 8
% = = = = = = = = = = = = = = = = = = = = = = = = =
%	FIN DE TRAYECTORIA
% = = = = = = = = = = = = = = = = = = = = = = = = =

\newpage{} 
\section{CUCO1.1: Análisis de fallos}

La redacción no va enfocada a lo que el CU hace.

\subsection{CUCO1.1: Análisis de fallos}

% = = = = = = = = = = = = = = = = = = = = = = = = =
%	INICIO DE TABLA
% = = = = = = = = = = = = = = = = = = = = = = = = = 


\begin{center}
\begin{longtable}{|rp{8cm}|}
\hline 
\textbf{Caso de uso:}  & CUCO1.1: Análisis de fallos\tabularnewline
\hline 
\multicolumn{2}{|>{\columncolor[gray]{0.7}}c}{Resumen de Atributos}\tabularnewline
\hline 
Realizador por:  & Efrén Hernandez Juárez\tabularnewline
\hline 
Revisado por:  & Pérez Alvarez Juan Carlos\tabularnewline
\hline 
Próposito:  & ¿El CU analiza o solo registra?\tabularnewline
\hline 
Entradas:  & Sin observaciones
\tabularnewline
\hline 
Salidas:  & Sin observaciones
\tabularnewline
\hline 
Pre-Condiciones  & ¿No es necesario iniciar sesión?
\tabularnewline
\hline 
Pos-Condiciones  & ¿No hay?
\tabularnewline
\hline 
Errores:  & ¿Qué codigos de mensaje son?
\tabularnewline
\hline 
\end{longtable}
\par\end{center}

% = = = = = = = = = = = = = = = = = = = = = = = = =
%	FIN DE TABLA
% = = = = = = = = = = = = = = = = = = = = = = = = =
% = = = = = = = = = = = = = = = = = = = = = = = = =
%	INICIO DE TRAYECTORIA
% = = = = = = = = = = = = = = = = = = = = = = = = =


\textit{\large Trayectoria Principal}{\large {} }{\large \par}

\begin{longtable}{rp{8cm}}
No.  & Observación\tabularnewline
1.  & ¿Dónde esta el botón? (Código de pantalla)\tabularnewline
2.  & No hay pantalla PCA017\tabularnewline
7.  & ¿Dónde lo ingresa? \tabularnewline
9.  & ¿A quién o a donde la envía?\tabularnewline
\end{longtable}

\textit{Trayectoria Alternativa A}

\begin{longtable}{rp{8cm}}
No.  & Observación\tabularnewline
3.  & No hay pantalla PCA02\tabularnewline
\end{longtable}

\textit{Trayectoria Alternativa B}

\begin{longtable}{rp{8cm}}
No.  & Observación\tabularnewline
1.  & ¿Cómo se determina si esta mal el lote?\tabularnewline
2. & ¿Cómo se le solicita?\tabularnewline
\end{longtable}

\textit{Trayectoria Alternativa C}

\begin{longtable}{rp{8cm}}
No.  & Observación\tabularnewline
2. & ¿Cómo se le solicita?\tabularnewline
\end{longtable}

Total de observaciones de este CU: 13
% = = = = = = = = = = = = = = = = = = = = = = = = =
%	FIN DE TRAYECTORIA
% = = = = = = = = = = = = = = = = = = = = = = = = =
\newpage{} 

\section{CUCO04: Generar Reporte}

Sin observaciones del resumen.

\subsection{CUCO04: Generar Reporte}

% = = = = = = = = = = = = = = = = = = = = = = = = =
%	INICIO DE TABLA
% = = = = = = = = = = = = = = = = = = = = = = = = = 


\begin{center}
\begin{longtable}{|rp{8cm}|}
\hline 
\textbf{Caso de uso:}  & CUCO04: Generar Reporte\tabularnewline
\hline 
\multicolumn{2}{|>{\columncolor[gray]{0.7}}c}{Resumen de Atributos}\tabularnewline
\hline 
Realizador por:  & Ricardo Jiménez Navarro\tabularnewline
\hline 
Revisado por:  & Pérez Alvarez Juan Carlos\tabularnewline
\hline 
Próposito:  & Sin observaciones\tabularnewline
\hline 
Entradas:  & Sin observaciones
\tabularnewline
\hline 
Salidas:  & Códigos de mensajes
\tabularnewline
\hline 
Pre-Condiciones  & Sin observaciones
\tabularnewline
\hline 
Pos-Condiciones  & ¿No hay?
\tabularnewline
\hline 
Errores:  & Códigos de mensajes
\tabularnewline
\hline 
\end{longtable}
\par\end{center}

% = = = = = = = = = = = = = = = = = = = = = = = = =
%	FIN DE TABLA
% = = = = = = = = = = = = = = = = = = = = = = = = =
% = = = = = = = = = = = = = = = = = = = = = = = = =
%	INICIO DE TRAYECTORIA
% = = = = = = = = = = = = = = = = = = = = = = = = =


\textit{\large Trayectoria Principal}{\large {} }{\large \par}

\begin{longtable}{rp{8cm}}
No.  & Observación\tabularnewline
1.  & ¿De donde la selecciona?\tabularnewline
2.  & ¿Cuál es PANTALLA1?\tabularnewline
5.  & No hay botón ACEPTA \tabularnewline
8.  & ¿Cómo lo genera?¿Lo guarda o lo muestra?\tabularnewline
\end{longtable}

\textit{Trayectoria Alternativa A}

\begin{longtable}{rp{8cm}}
No.  & Observación\tabularnewline
3.  &  No hay PANTALLA GENERAR REPORTE\tabularnewline
\end{longtable}

\textit{Trayectoria Alternativa C}

\begin{longtable}{rp{8cm}}
No.  & Observación\tabularnewline
2. & La pantalla dice: Periodo especifico. Falta definir que solicita las fechas y sus validaciones.\tabularnewline
\end{longtable}

Total de observaciones de este CU: 8
% = = = = = = = = = = = = = = = = = = = = = = = = =
%	FIN DE TRAYECTORIA
% = = = = = = = = = = = = = = = = = = = = = = = = =
\newpage{} 


\section{CUVE3. Servicio al cliente}

El nombre del CU no coincide con el de la tabla. 

\subsection{CUVE3. Servicio al cliente}

% = = = = = = = = = = = = = = = = = = = = = = = = =
%	INICIO DE TABLA
% = = = = = = = = = = = = = = = = = = = = = = = = = 


\begin{center}
\begin{longtable}{|rp{8cm}|}
\hline 
\textbf{Caso de uso:}  & CCUVE3. Servicio al cliente \tabularnewline
\hline 
\multicolumn{2}{|>{\columncolor[gray]{0.7}}c}{Resumen de Atributos}\tabularnewline
\hline 
Realizador por:  & Carlos Antonio Macías Duarte\tabularnewline
\hline 
Revisado por:  & Pérez Alvarez Juan Carlos\tabularnewline
\hline 
Próposito:  & "La ventas"\tabularnewline
\hline 
Entradas:  & Faltan algunos acentos. Mayúsculas donde no van.
\tabularnewline
\hline 
Salidas:  & Sin observaciones
\tabularnewline
\hline 
Pre-Condiciones  & Mayúsculas y minúsculas donde no van
\tabularnewline
\hline 
Pos-Condiciones  & ¿No hay?
\tabularnewline
\hline 
Errores:  & Sin observaciones
\tabularnewline
\hline 
\end{longtable}
\par\end{center}

% = = = = = = = = = = = = = = = = = = = = = = = = =
%	FIN DE TABLA
% = = = = = = = = = = = = = = = = = = = = = = = = =
% = = = = = = = = = = = = = = = = = = = = = = = = =
%	INICIO DE TRAYECTORIA
% = = = = = = = = = = = = = = = = = = = = = = = = =


\textit{\large Trayectoria Principal}{\large {} }{\large \par}

\begin{longtable}{rp{8cm}}
No.  & Observación\tabularnewline
1.  & ¿Dónde y como lo solicita? Mayúsculas y minúsculas (en este y casi todos los puntos de las trayectorias)\tabularnewline
5.  & No hay botón Aceptar\tabularnewline
10.  & "Reportess"\tabularnewline
\end{longtable}

\textit{Trayectoria Alternativa A}

\begin{longtable}{rp{8cm}}
No.  & Observación\tabularnewline
4.  &  No hay botón Aceptar\tabularnewline
\end{longtable}

\textit{Trayectoria Alternativa B}

\begin{longtable}{rp{8cm}}
No.  & Observación\tabularnewline
1. & ¿No ya lo verificó de donde viene esta trayectoria (A.5)?\tabularnewline
\end{longtable}

\textit{Trayectoria Alternativa C}

\begin{longtable}{rp{8cm}}
No.  & Observación\tabularnewline
1. & Misma observación que en B.1\tabularnewline
\end{longtable}

\textit{Trayectoria Alternativa D}

\begin{longtable}{rp{8cm}}
No.  & Observación\tabularnewline
1. & Misma observación que en B.1\tabularnewline
\end{longtable}

\textit{Trayectoria Alternativa E}

\begin{longtable}{rp{8cm}}
No.  & Observación\tabularnewline
1. & Misma observación que en B.1\tabularnewline
2. & ¿Cómo sabe el sistema que es un dato inválido?\tabularnewline
\end{longtable}

\textit{Trayectoria Alternativa F}

\begin{longtable}{rp{8cm}}
No.  & Observación\tabularnewline
1. & "preciona"\tabularnewline
\end{longtable}

\textit{Trayectoria Alternativa G}

\begin{longtable}{rp{8cm}}
No.  & Observación\tabularnewline
1. & Misma observación que en B.1\tabularnewline
\end{longtable}

\textit{Trayectoria Alternativa H}

\begin{longtable}{rp{8cm}}
No.  & Observación\tabularnewline
1. & Misma observación que en B.1\tabularnewline
2. & ¿Cómo es un valor válido? \tabularnewline
\end{longtable}


Total de observaciones de este CU: 18
% = = = = = = = = = = = = = = = = = = = = = = = = =
%	FIN DE TRAYECTORIA
% = = = = = = = = = = = = = = = = = = = = = = = = =
\newpage{} 

\end{document}




