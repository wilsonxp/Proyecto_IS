
\documentclass[10pt,spanish]{article}
\usepackage[utf8x]{inputenc}
\usepackage[letterpaper]{geometry}
\geometry{verbose}
\usepackage{array}
\usepackage{longtable}
\usepackage{amsmath}
\usepackage{amssymb}

\makeatletter

\usepackage{array}

\providecommand{\tabularnewline}{\\}

\usepackage{ucs}\usepackage[spanish]{babel}
\usepackage{amsfonts}\usepackage{colortbl}
% = = = = = = = = = = = = = = = = = = = = = = = = =
% INICIO EL DOCUMENTO
% = = = = = = = = = = = = = = = = = = = = = = = = =
\usepackage{babel}
\addto\shorthandsspanish{\spanishdeactivate{~<>}}

\begin{document}
\section{Modulo de Producción}


\subsection{CUPR1: Gestionar Receta de Galletas}

Resumen: la redaccion con la palabra CONTROL no es correcta, ya que da a entender que validara que personas tendran acceso al modulo.


% = = = = = = = = = = = = = = = = = = = = = = = = =
%	INICIO DE TABLA
% = = = = = = = = = = = = = = = = = = = = = = = = = 


\begin{center}
\begin{longtable}{|rp{8cm}|}
\hline 
\textbf{Caso de uso:}  & Gestionar Receta de Galletas\tabularnewline
\hline 
\multicolumn{2}{|>{\columncolor[gray]{0.7}}c}{Resumen de Atributos}\tabularnewline
\hline 
Realizador por:  & Cristian Hipolito Tenorio\tabularnewline
\hline 
Revisado por:  & Eduardo Ventura Cruz\tabularnewline
\hline 
Próposito:  & 
¿elimina, actualiza o realiza ambas?\tabularnewline
\hline 
Entradas:  &   
\begin{itemize}
\item los menus no son entradas 
\end{itemize}
\tabularnewline
\hline 
Salidas:  & 
\begin{itemize}
\item Sin observaciones. 
\end{itemize}
\tabularnewline
\hline 
Pre-Condiciones:  &  
\begin{itemize}
\item Los usuarios del modulo deben estar autentificados\end{itemize}
\tabularnewline
\hline 
Pos-Condiciones:  & \begin{itemize}
\item no hay poscondiciones? \end{itemize}
\tabularnewline
\hline 
Errores:  &  
\begin{itemize}
\item El error no se utiliza en la trayectoria\end{itemize}
\tabularnewline
\hline 
\end{longtable}
\par\end{center}

% = = = = = = = = = = = = = = = = = = = = = = = = =
%	FIN DE TABLA
% = = = = = = = = = = = = = = = = = = = = = = = = =
% = = = = = = = = = = = = = = = = = = = = = = = = =
%	INICIO DE TRAYECTORIA
% = = = = = = = = = = = = = = = = = = = = = = = = =


\textit{\large Trayectoria Principal}{\large {} }{\large \par}
 
\begin{longtable}{rp{8cm}}
No.  & Observación\tabularnewline
1.  & No hay pantallas correspondientes.\tabularnewline
2.  & La pantalla no coincide\tabularnewline
\end{longtable}

\textit{Trayectoria Alternativa A}

Condición: No se definen las condiciones por las cuales se llega a las trayectorias alternativas.

\begin{longtable}{rp{8cm}}
No.  & Observación\tabularnewline
1. & No hay pantalla  \tabularnewline
\end{longtable}% = = = = = = = = = = = = = = = = = = = = = = = = =
%	FIN DE TRAYECTORIA
% = = = = = = = = = = = = = = = = = = = = = = = = =
Total de observaciones de este CU:8

\newpage{}
\subsection{CUPR1.1: Crear Nueva Receta}

Resumen: Sin observaciones.

% = = = = = = = = = = = = = = = = = = = = = = = = =
%	INICIO DE TABLA
% = = = = = = = = = = = = = = = = = = = = = = = = = 


\begin{center}
\begin{longtable}{|rp{8cm}|}
\hline 
\textbf{Caso de uso:}  & CUPR1.1 Crear nueva receta\tabularnewline
\hline 
\multicolumn{2}{|>{\columncolor[gray]{0.7}}c}{Resumen de Atributos}\tabularnewline
\hline 
Realizador por:  & Cristian Hipolito Tenorio\tabularnewline
\hline 
Revisado por:  & Eduardo Ventura Cruz\tabularnewline
\hline 
Próposito:  & 
Sin observaciones\tabularnewline
\hline 
Entradas:  &   
\begin{itemize}
\item Falta especificar las entradas de los datos. 
\end{itemize}
\tabularnewline
\hline 
Salidas:  & 
\begin{itemize}
\item Sin observaciones
\end{itemize}
\tabularnewline
\hline 
Pre-Condiciones:  &  
\begin{itemize}
\item Los usuarios del modulo deben estar autentificados
\item ¿Es necesario que existe la materia prima en producción?
\end{itemize}
\tabularnewline
\hline 
Pos-Condiciones:  &  
\begin{itemize}
\item ¿No hay poscondiciones?\end{itemize}
\tabularnewline
\hline 
Errores:  & Anotaciones de los errores 
\begin{itemize}
\item El error no se utiliza en la trayectoria\end{itemize}
\tabularnewline
\hline 
\end{longtable}
\par\end{center}

% = = = = = = = = = = = = = = = = = = = = = = = = =
%	FIN DE TABLA
% = = = = = = = = = = = = = = = = = = = = = = = = =
% = = = = = = = = = = = = = = = = = = = = = = = = =
%	INICIO DE TRAYECTORIA
% = = = = = = = = = = = = = = = = = = = = = = = = =


\textit{\large Trayectoria Principal}{\large {} }{\large \par}
 
\begin{longtable}{rp{8cm}}
No.  & Observaciones\tabularnewline
1.  & No se especifica la nomenclatura de las pantallas.\tabularnewline
2.  & NO hay pantallas.\tabularnewline

\end{longtable}

\textit{Trayectoria Alternativa A}

Condición: No se definen las condiciones por las cuales se llega a las trayectorias alternativas.

\begin{longtable}{rp{8cm}}
No.  & Observación\tabularnewline
1. & Esa es la condicion de la trayectoria. \tabularnewline
\end{longtable}% = = = = = = = = = = = = = = = = = = = = = = = = =
\textit{Trayectoria Alternativa B}

\begin{longtable}{rp{8cm}}
No.  & Observación\tabularnewline
1. & ¿solo valida campos vacios?   \tabularnewline
\end{longtable}% = = = = = = = = = = = = = = = = = = = = = = = = =

%	FIN DE TRAYECTORIA
% = = = = = = = = = = = = = = = = = = = = = = = = =
Total de observaciones de este CU:9
\newpage{} 
 
\subsection{CUPR1.2: Modificar Receta}

Resumen: No es necesario especificar el usuario.

% = = = = = = = = = = = = = = = = = = = = = = = = =
%	INICIO DE TABLA
% = = = = = = = = = = = = = = = = = = = = = = = = = 

\begin{center}
\begin{longtable}{|rp{8cm}|}
\hline 
\textbf{Caso de uso:}  & CUPR1.2 Modificar Receta\tabularnewline
\hline 
\multicolumn{2}{|>{\columncolor[gray]{0.7}}c}{Resumen de Atributos}\tabularnewline
\hline 
Realizador por:  & Cristian Hipolito Tenorio\tabularnewline
\hline 
Revisado por:  & Eduardo Ventura Cruz\tabularnewline
\hline 
Próposito:  & 
Sin observaciones\tabularnewline
\hline 
Entradas:  &   
\begin{itemize}
\item Falta especificar las entradas de los datos. 
\end{itemize}
\tabularnewline
\hline 
Salidas:  & 
\begin{itemize}
\item Sin observaciones
\end{itemize}
\tabularnewline
\hline 
Pre-Condiciones:  &  
\begin{itemize}
\item Los usuario del modulo deben estar auntentificados
\end{itemize}
\tabularnewline
\hline 
Pos-Condiciones:  &  
\begin{itemize}
\item ¿No hay poscondiciones?\end{itemize}
\tabularnewline
\hline 
Errores:  & 
\begin{itemize}
\item No se especifica en que parte se utiliza el error\end{itemize}
\tabularnewline
\hline 
\end{longtable}
\par\end{center}

% = = = = = = = = = = = = = = = = = = = = = = = = =
%	FIN DE TABLA
% = = = = = = = = = = = = = = = = = = = = = = = = =
% = = = = = = = = = = = = = = = = = = = = = = = = =
%	INICIO DE TRAYECTORIA
% = = = = = = = = = = = = = = = = = = = = = = = = =


\textit{\large Trayectoria Principal}{\large {} }{\large \par}
 
\begin{longtable}{rp{8cm}}
No.  & Observaciones\tabularnewline
1.  & No se especifica la nomenclatura de las pantallas.\tabularnewline
2.  & NO hay pantallas.\tabularnewline
5.  & ¿Solo valida campos vacios? .\tabularnewline

\end{longtable}

\textit{Trayectoria Alternativa A}

Condición: No se definen las condiciones por las cuales se llega a las trayectorias alternativas.

\begin{longtable}{rp{8cm}}
No.  & Observación\tabularnewline
1. & Esa es la condicion de la trayectoria.  \tabularnewline
\end{longtable}% = = = = = = = = = = = = = = = = = = = = = = = = =
\textit{Trayectoria Alternativa B}

\begin{longtable}{rp{8cm}}
No.  & Observación\tabularnewline
1. & ¿solo valida campos vacios?   \tabularnewline
\end{longtable}% = = = = = = = = = = = = = = = = = = = = = = = = =

%	FIN DE TRAYECTORIA
% = = = = = = = = = = = = = = = = = = = = = = = = =
Total de observaciones de este CU:8


\newpage{} 
 
\subsection{CUPR1.3: Eliminar Receta}

Resumen: No es necesario especificar el usuario.

% = = = = = = = = = = = = = = = = = = = = = = = = =
%	INICIO DE TABLA
% = = = = = = = = = = = = = = = = = = = = = = = = = 

\begin{center}
\begin{longtable}{|rp{8cm}|}
\hline 
\textbf{Caso de uso:}  & CUPR1.3 Eliminar Receta\tabularnewline
\hline 
\multicolumn{2}{|>{\columncolor[gray]{0.7}}c}{Resumen de Atributos}\tabularnewline
\hline 
Realizador por:  & Cristian Hipolito Tenorio\tabularnewline
\hline 
Revisado por:  & Eduardo Ventura Cruz\tabularnewline
\hline 
Próposito:  & 
Sin observaciones\tabularnewline
\hline 
Entradas:  &   
\begin{itemize}
\item Los menus no son entradas. 
\end{itemize}
\tabularnewline
\hline 
Salidas:  & 
\begin{itemize}
\item Sin observaciones
\end{itemize}
\tabularnewline
\hline 
Pre-Condiciones:  &  
\begin{itemize}
\item Los usuario del modulo deben estar auntentificados
\end{itemize}
\tabularnewline
\hline 
Pos-Condiciones:  &  
\begin{itemize}
\item ¿No hay poscondiciones?\end{itemize}
\tabularnewline
\hline 
Errores:  & 
\begin{itemize}
\item No se especifican los errores utilizados en la trayectoria\end{itemize}
\tabularnewline
\hline 
\end{longtable}
\par\end{center}

% = = = = = = = = = = = = = = = = = = = = = = = = =
%	FIN DE TABLA
% = = = = = = = = = = = = = = = = = = = = = = = = =
% = = = = = = = = = = = = = = = = = = = = = = = = =
%	INICIO DE TRAYECTORIA
% = = = = = = = = = = = = = = = = = = = = = = = = =


\textit{\large Trayectoria Principal}{\large {} }{\large \par}
 
\begin{longtable}{rp{8cm}}
No.  & Observaciones\tabularnewline
1.  & No se especifica la nomenclatura de las pantallas.\tabularnewline
2.  & NO hay pantallas.\tabularnewline
\end{longtable}

\textit{Trayectoria Alternativa A}

Condición: No se definen las condiciones por las cuales se llega a las trayectorias alternativas.

\begin{longtable}{rp{8cm}}
No.  & Observación\tabularnewline
1. & Esa es la condicion de la trayectoria. \tabularnewline
\end{longtable}% = = = = = = = = = = = = = = = = = = = = = = = = =

%	FIN DE TRAYECTORIA
% = = = = = = = = = = = = = = = = = = = = = = = = =
Total de observaciones de este CU:8


\end{document}
