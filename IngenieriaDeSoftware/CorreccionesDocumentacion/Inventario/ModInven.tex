%% LyX 2.0.3 created this file.  For more info, see http://www.lyx.org/.
%% Do not edit unless you really know what you are doing.
\documentclass[10pt,spanish]{article}
\usepackage[utf8x]{inputenc}
\usepackage[letterpaper]{geometry}
\geometry{verbose}
\usepackage{amsmath}
\usepackage{amssymb}
\usepackage{graphicx}

\makeatletter

%%%%%%%%%%%%%%%%%%%%%%%%%%%%%% LyX specific LaTeX commands.
%% Because html converters don't know tabularnewline
\providecommand{\tabularnewline}{\\}

%%%%%%%%%%%%%%%%%%%%%%%%%%%%%% User specified LaTeX commands.

\usepackage{ucs}\usepackage[spanish]{babel}
\usepackage{amsfonts}\usepackage{colortbl}% = = = = = = = = = = = = = = = = = = = = = = = = =
% INICIO EL DOCUMENTO
% = = = = = = = = = = = = = = = = = = = = = = = = =



\usepackage{babel}
\addto\shorthandsspanish{\spanishdeactivate{~<>}}





\usepackage{babel}
\addto\shorthandsspanish{\spanishdeactivate{~<>}}



\usepackage{babel}
\addto\shorthandsspanish{\spanishdeactivate{~<>}}





\usepackage{babel}
\addto\shorthandsspanish{\spanishdeactivate{~<>}}





\usepackage{babel}
\addto\shorthandsspanish{\spanishdeactivate{~<>}}





\usepackage{babel}
\addto\shorthandsspanish{\spanishdeactivate{~<>}}





\usepackage{babel}
\addto\shorthandsspanish{\spanishdeactivate{~<>}}





\usepackage{babel}
\addto\shorthandsspanish{\spanishdeactivate{~<>}}





\usepackage{babel}
\addto\shorthandsspanish{\spanishdeactivate{~<>}}



\usepackage{babel}
\addto\shorthandsspanish{\spanishdeactivate{~<>}}





\usepackage{babel}
\addto\shorthandsspanish{\spanishdeactivate{~<>}}





\usepackage{babel}
\addto\shorthandsspanish{\spanishdeactivate{~<>}}





\usepackage{babel}
\addto\shorthandsspanish{\spanishdeactivate{~<>}}





\usepackage{babel}
\addto\shorthandsspanish{\spanishdeactivate{~<>}}





\usepackage{babel}
\addto\shorthandsspanish{\spanishdeactivate{~<>}}





\usepackage{babel}
\addto\shorthandsspanish{\spanishdeactivate{~<>}}

\makeatother

\usepackage{babel}
\addto\shorthandsspanish{\spanishdeactivate{~<>}}

\begin{document}
\tableofcontents{}

\pagebreak{}


\section{Módulo Inventarios.}


\subsection{CUI1.0: Generar Lista de compras}

\textbf{\large Resumen}\\
 {\large {} {} {} {} Este Caso de uso tiene como objetivo el generar
una lista de las materias primas que estan por agotarse y permitir
de esta manera adquirir mas}\\


% = = = = = = = = = = = = = = = = = = = = = = = = =
%    INICIO DE TABLA
% = = = = = = = = = = = = = = = = = = = = = = = = = 
\begin{table}[!ht]
\begin{centering}
\begin{tabular}{|c||c|l|}
\hline 
\multicolumn{2}{|c|}{Caso de Uso:} & CUI1.0: Generar Lista de Compras\tabularnewline
\hline 
\multicolumn{3}{|>{\columncolor[gray]{0.7}}c}{Resumen de Atributos}\tabularnewline
\hline 
\multicolumn{2}{|c|}{Autor} & \multicolumn{1}{p{10cm}||}{Mar Escoto Diana}\tabularnewline
\hline 
\multicolumn{2}{|c|}{Actor} & \multicolumn{1}{p{10cm}||}{Personal de Inventarios}\tabularnewline
\hline 
\multicolumn{2}{|c|}{Propósito} & \multicolumn{1}{p{10cm}||}{Generar una lista con las materias primas que estan por agotarse}\tabularnewline
\hline 
\multicolumn{2}{|c|}{Entradas} & \multicolumn{1}{p{10cm}||}{}\tabularnewline
\hline 
\multicolumn{2}{|c|}{Salidas} & \multicolumn{1}{p{10cm}||}{Reporte enviado al modulo de compras}\tabularnewline
\hline 
\multicolumn{2}{|c|}{Pre-condiciones} & \multicolumn{1}{p{10cm}||}{Que el usuario este dado de alta en el área de inventario}\tabularnewline
\hline 
\multicolumn{2}{|c|}{Pos-condiciones} & \multicolumn{1}{p{10cm}||}{Enviar un reporte al módulo de compras}\tabularnewline
\hline 
\multicolumn{2}{|c|}{Errores} & \multicolumn{1}{p{10cm}||}{MSG4 Error al conectar con la base de datos. MSG2 'El reporte no se
realizo correctamente}\tabularnewline
\hline 
\multicolumn{2}{|c|}{Tipo} & \multicolumn{1}{p{10cm}||}{Primario}\tabularnewline
\hline 
\multicolumn{2}{|c|}{Fuente} & \multicolumn{1}{p{10cm}||}{}\tabularnewline
\hline 
\end{tabular}
\par\end{centering}

\caption{Caso de Uso 1.0: Generar Lista de Compras}


\label{CUI1.0} 
\end{table}


% = = = = = = = = = = = = = = = = = = = = = = = = =
%	FIN DE TABLA
% = = = = = = = = = = = = = = = = = = = = = = = = =
% = = = = = = = = = = = = = = = = = = = = = = = = =
%	INICIO DE TRAYECTORIA
% = = = = = = = = = = = = = = = = = = = = = = = = = 


\textbf{\large Trayectorias del CU}{\large \par}

\textit{\large Trayectoria Principal}{\large {} }{\large \par}

\begin{tabular}{ccl}
 &  & \tabularnewline
1.  & \includegraphics{actor}  & \multicolumn{1}{p{12cm}}{ Presiona el icono {[}Consultar Existencias{]} en la pantalla PI3}\tabularnewline
2.  & \includegraphics{sistema}  & \multicolumn{1}{p{12cm}}{Carga los datos de las materias primas y la cantidad de existencia
{[}Trayectoria Alternativa A{]}}\tabularnewline
3.  & \includegraphics{sistema}  & \multicolumn{1}{p{12cm}}{Despliega la pantalla PI3.1}\tabularnewline
4.  & \includegraphics{actor}  & \multicolumn{1}{p{12cm}}{Selecciona las materias primas que estan por terminarse para agregar
al reporte}\tabularnewline
5.  & \includegraphics{actor}  & \multicolumn{1}{p{12cm}}{Selecciona el icono {[}Realizar reporte{]}}\tabularnewline
6.  & \includegraphics{sistema}  & \multicolumn{1}{p{12cm}}{El sistema envía un mensaje de reporte enviado exitosamente MSG1 {[}Trayectoria
Alternativa B{]}}\tabularnewline
 &  & \multicolumn{1}{p{12cm}}{.... Fin del caso de uso}\tabularnewline
\end{tabular}\\
 \textit{Trayectoria Alternativa A}

Condición: rNo pudo cargar los datos solicitados

\begin{tabular}{ccl}
 &  & \tabularnewline
1.  & \includegraphics{sistema}  & \multicolumn{1}{p{12cm}}{ Muestra el mensaje de error MSG4 'No se pudo establecer conexión
con la base de datos'}\tabularnewline
2.  & \includegraphics{sistema}  & \multicolumn{1}{p{12cm}}{Regresa a la pantalla PI3}\tabularnewline
 &  & \multicolumn{1}{p{12cm}}{.... Fin del caso de uso}\tabularnewline
\end{tabular}\\
 \textit{Trayectoria Alternativa B}

Condición: El reporte no se realiza exitosamente

\begin{tabular}{ccl}
 &  & \tabularnewline
1.  & \includegraphics{sistema}  & \multicolumn{1}{p{12cm}}{ Muestra el mensaje de error MSG2 'El reporte no se realizo correctamente'}\tabularnewline
2.  & \includegraphics{sistema}  & \multicolumn{1}{p{12cm}}{Regresa a la pantalla PI3}\tabularnewline
 &  & \multicolumn{1}{p{12cm}}{.... Fin del caso de uso}\tabularnewline
\end{tabular}

\newpage

\subsection{CUI2.0: Consultar Existencia de productos}

\textbf{\large Resumen}\\
 {\large {} {} {} {} {} {}El caso de uso se refiere a la posibilidad
de que el usuario de inventarior supervise la existencia de algun
producto.}\\


% = = = = = = = = = = = = = = = = = = = = = = = = =
%	INICIO DE TABLA
% = = = = = = = = = = = = = = = = = = = = = = = = = 
\begin{table}[!ht]
\begin{centering}
\begin{tabular}{|c||c|l|}
\hline 
\multicolumn{2}{|c|}{Caso de Uso:} & CUI2.0: Existencias de productos\tabularnewline
\hline 
\multicolumn{3}{|>{\columncolor[gray]{0.7}}c}{Resumen de Atributos}\tabularnewline
\hline 
\multicolumn{2}{|c|}{Autor} & \multicolumn{1}{p{10cm}||}{Consuelo Mayén Christian Armando}\tabularnewline
\hline 
\multicolumn{2}{|c|}{Actor} & \multicolumn{1}{p{10cm}||}{El personal capacitado para el manejo de inventarios.}\tabularnewline
\hline 
\multicolumn{2}{|c|}{Propósito} & \multicolumn{1}{p{10cm}||}{Que el usuario pueda consultar la disponibilidad de algún producto en el inventario.}\tabularnewline
\hline 
\multicolumn{2}{|c|}{Entradas} & \multicolumn{1}{p{10cm}||}{Algún criterio de búsqueda}\tabularnewline
\hline 
\multicolumn{2}{|c|}{Salidas} & \multicolumn{1}{p{10cm}||}{Reporte con la información requerida. }\tabularnewline
\hline 
\multicolumn{2}{|c|}{Pre-condiciones} & \multicolumn{1}{p{10cm}||}{Estar identificado como un usuario encargado de inventarios.}\tabularnewline
\hline 
\multicolumn{2}{|c|}{Pos-condiciones} & \multicolumn{1}{p{10cm}||}{Que los datos de la base de datos se hayan consultado satisfactoriamente}\tabularnewline
\hline 
\multicolumn{2}{|c|}{Errores} & \multicolumn{1}{p{10cm}||}{Error al ingresar a la base de datos.}\tabularnewline
\hline 
\multicolumn{2}{|c|}{Tipo} & \multicolumn{1}{p{10cm}||}{Primario}\tabularnewline
\hline 
\multicolumn{2}{|c|}{Fuente} & \multicolumn{1}{p{10cm}||}{}\tabularnewline
\hline 
\end{tabular}
\par\end{centering}


\caption{Caso de Uso2.0: Existencias de productos}


\label{CUI4.0} 
\end{table}


% = = = = = = = = = = = = = = = = = = = = = = = = =
%	FIN DE TABLA
% = = = = = = = = = = = = = = = = = = = = = = = = =
% = = = = = = = = = = = = = = = = = = = = = = = = =
%	INICIO DE TRAYECTORIA
% = = = = = = = = = = = = = = = = = = = = = = = = = 


\textbf{\large Trayectorias del CU}{\large \par}

\textit{\large Trayectoria Principal}{\large {} }{\large \par}

%Solo hay que cambiar el nombre de la imagen dependiendo de si es actor o sistema


\begin{tabular}{ccl}
 &  & \tabularnewline
1.  & \includegraphics{actor}  & \multicolumn{1}{p{12cm}}{ Presiona el ícono {[}Gestión Lotes{]} en la pantalla PI1}\tabularnewline
2.  & \includegraphics{actor}  & \multicolumn{1}{p{12cm}}{ Presiona el ícono {[}Consulta producto{]} en la pantalla PI3}\tabularnewline
3.  & \includegraphics{sistema}  & \multicolumn{1}{p{12cm}}{Despliega la pantalla PI6 con el boton buscar.}\tabularnewline
4.  & \includegraphics{sistema}  & \multicolumn{1}{p{12cm}}{Recibe el reporte y lo muestra en pantalla, especificando asi la información
necesaria que se ha requerido.}\tabularnewline
 &  & \multicolumn{1}{p{12cm}}{.... Fin del caso de uso}\tabularnewline
\end{tabular}

\newpage{}

\textit{Trayectoria Alternativa A}

Condición: No se pudo acceder a los registros de la BD.

\begin{tabular}{ccl}
 &  & \tabularnewline
1.  & \includegraphics{sistema}  & \multicolumn{1}{p{12cm}}{ Muestra el mensaje de error MSG4 'No se pudo establecer conexión
con la base de datos'}\tabularnewline
2.  & \includegraphics{sistema}  & \multicolumn{1}{p{12cm}}{Regresa a la pantalla PI1}\tabularnewline
 &  & \multicolumn{1}{p{12cm}}{.... Fin del caso de uso}\tabularnewline
\end{tabular}

\textit{Trayectoria Alternativa B}

Condición: Los registros tienen mal formato.

\begin{tabular}{ccl}
 &  & \tabularnewline
1.  & \includegraphics{sistema}  & \multicolumn{1}{p{12cm}}{ Muestra el mensaje de error MSG7 'Formato de los datos es incorrecto'}\tabularnewline
2.  & \includegraphics{sistema}  & \multicolumn{1}{p{12cm}}{Regresa a la pantalla PI3}\tabularnewline
 &  & \multicolumn{1}{p{12cm}}{.... Fin del caso de uso}\tabularnewline
\end{tabular}

\textit{Trayectoria Alternativa C}

Condición: Datos no encontrados en la consulta que acabamos de realizar.

\begin{tabular}{ccl}
 &  & \tabularnewline
1.  & \includegraphics{sistema}  & \multicolumn{1}{p{12cm}}{ Muestra el mensaje de error MSG3 Datos no encontrados'}\tabularnewline
2.  & \includegraphics{sistema}  & \multicolumn{1}{p{12cm}}{Regresa a la pantalla PI3}\tabularnewline
 &  & \multicolumn{1}{p{12cm}}{.... Fin del caso de uso}\tabularnewline
\end{tabular}


\newpage



\subsection{CUI3.0: Reportes Inventario}

\textbf{\large Resumen}{\large }\\
{\large{} {} {} {} {} Llevar a cabo el llenado y envio de los reportes
para notificar a las demás instancias algún contratiempo y/o pedido.}\\


% = = = = = = = = = = = = = = = = = = = = = = = = =
%	INICIO DE TABLA
% = = = = = = = = = = = = = = = = = = = = = = = = = 
\begin{table}[!ht]
\begin{centering}
\begin{tabular}{|c||c|l|}
\hline 
\multicolumn{2}{|c|}{Caso de Uso:} & CUI3.0: Reportes Inventario\tabularnewline
\hline 
\multicolumn{3}{|>{\columncolor[gray]{0.7}}c}{Resumen de Atributos}\tabularnewline
\hline 
\multicolumn{2}{|c|}{Autor} & \multicolumn{1}{p{10cm}||}{Consuelo Mayén Christian Armando}\tabularnewline
\hline 
\multicolumn{2}{|c|}{Actor} & \multicolumn{1}{p{10cm}||}{Personal de Inventarios.}\tabularnewline
\hline 
\multicolumn{2}{|c|}{Propósito} & \multicolumn{1}{p{10cm}||}{Notificar a las demás instancias acerca de alguna anomalía o pedido.}\tabularnewline
\hline 
\multicolumn{2}{|c|}{Entradas} & \multicolumn{1}{p{10cm}||}{Datos del formulario.}\tabularnewline
\hline 
\multicolumn{2}{|c|}{Salidas} & \multicolumn{1}{p{10cm}||}{Mensaje enviado y notificación a la instancia pertinende.}\tabularnewline
\hline 
\multicolumn{2}{|c|}{Pre-condiciones} & \multicolumn{1}{p{10cm}||}{Estar registrado como personal de inventario calificado.}\tabularnewline
\hline 
\multicolumn{2}{|c|}{Pos-condiciones} & \multicolumn{1}{p{10cm}||}{Que el reporte se haya concluido satisfactoriamente}\tabularnewline
\hline 
\multicolumn{2}{|c|}{Errores} & \multicolumn{1}{p{10cm}||}{Errores al notificar a las demás instancias de la empresa.}\tabularnewline
\hline 
\multicolumn{2}{|c|}{Tipo} & \multicolumn{1}{p{10cm}||}{Primario}\tabularnewline
\hline 
\multicolumn{2}{|c|}{Fuente} & \multicolumn{1}{p{10cm}||}{Basado en el funcionamiento etc.}\tabularnewline
\hline 
\end{tabular}
\par\end{centering}

\caption{Caso de Uso 3.0: Reportes Inventario}


\label{CUI5.0} 
\end{table}


% = = = = = = = = = = = = = = = = = = = = = = = = =
%	FIN DE TABLA
% = = = = = = = = = = = = = = = = = = = = = = = = =
% = = = = = = = = = = = = = = = = = = = = = = = = =
%	INICIO DE TRAYECTORIA
% = = = = = = = = = = = = = = = = = = = = = = = = = 


\textbf{\large Trayectorias del CU}{\large \par}

\textit{\large Trayectoria Principal}{\large {} }{\large \par}

\begin{tabular}{ccl}
 &  & \tabularnewline
1.  & \includegraphics{actor}  & \multicolumn{1}{p{12cm}}{ Presiona el ícono {[}Gestión Lotes{]} en la pantalla PI1}\tabularnewline
2.  & \includegraphics{actor}  & \multicolumn{1}{p{12cm}}{ Presiona el ícono {[}Consulta producto{]} en la pantalla PI3}\tabularnewline
3.  & \includegraphics{actor}  & \multicolumn{1}{p{12cm}}{Llenar el formulario con toda la información que se necesita.}\tabularnewline
4.  & \includegraphics{sistema}  & \multicolumn{1}{p{12cm}}{Se envia el reporte para que el sistema mande las notificaciones a
la entidadd seleccionada.}\tabularnewline
5.  & \includegraphics{sistema}  & \multicolumn{1}{p{12cm}}{Se presenta un mensaje de aprovación MSG1.}\tabularnewline
 &  & \multicolumn{1}{p{12cm}}{.... Fin del caso de uso}\tabularnewline
\end{tabular}

\newpage{}

\subsection{CUI4: Gestionar Materias Primas}

\textbf{\large Resumen}{\large }\\
{\large{} Este Caso de uso tiene como objetivo permitir al usuario registrar, modificar y buscar registros de materias primas}\\


% = = = = = = = = = = = = = = = = = = = = = = = = =
%	INICIO DE TABLA
% = = = = = = = = = = = = = = = = = = = = = = = = = 
% Se utiliza para que no se salga la tabla de la hoja, porque sino se autodimensiona
%\multicolumn{1}{p{10cm}||}{Contenido}
%
\begin{table}[!ht]
\begin{centering}
\begin{tabular}{|c||c|l|}
\hline 
\multicolumn{2}{|c|}{Caso de Uso:} & CUI4.0: Gestionar Materias Primas\tabularnewline
\hline 
\multicolumn{3}{|>{\columncolor[gray]{0.7}}c}{Resumen de Atributos}\tabularnewline
\hline 
\multicolumn{2}{|c|}{Autor} & \multicolumn{1}{p{10cm}||}{Mar Escoto Diana}\tabularnewline
\hline 
\multicolumn{2}{|c|}{Actor} & \multicolumn{1}{p{10cm}||}{Personal de Inventarios}\tabularnewline
\hline 
\multicolumn{2}{|c|}{Propósito} & \multicolumn{1}{p{10cm}||}{Acceder a opciones de crear, modificar,consultar o buscar de materias primas.
}\tabularnewline
\hline 
\multicolumn{2}{|c|}{Entradas} & \multicolumn{1}{p{10cm}||}{}\tabularnewline
\hline 
\multicolumn{2}{|c|}{Salidas} & \multicolumn{1}{p{10cm}||}{Modificación de los registros de materia prima en la base de datos}\tabularnewline
\hline 
\multicolumn{2}{|c|}{Pre-condiciones} & \multicolumn{1}{p{10cm}||}{Que el usuario este dado de alta en el área de inventario.}\tabularnewline
\hline 
\multicolumn{2}{|c|}{Pos-condiciones} & \multicolumn{1}{p{10cm}||}{La base de datos debe estar disponible}\tabularnewline
\hline 
\multicolumn{2}{|c|}{Errores} & \multicolumn{1}{p{10cm}||}{MSG4 Error al conectar con la base de datos.}\tabularnewline
\hline 
\multicolumn{2}{|c|}{Tipo} & \multicolumn{1}{p{10cm}||}{Primario}\tabularnewline
\hline 
\multicolumn{2}{|c|}{Fuente} & \multicolumn{1}{p{10cm}||}{Basado en el funcionamiento etc.}\tabularnewline
\hline 
\end{tabular}
\par\end{centering}

\caption{Caso de Uso I4.0: Gestionar Materias Primas}
\label{tab:CasosdeUso:nombredecasodeuso} 
\end{table}


% = = = = = = = = = = = = = = = = = = = = = = = = =
%	FIN DE TABLA
% = = = = = = = = = = = = = = = = = = = = = = = = =
% = = = = = = = = = = = = = = = = = = = = = = = = =
%	INICIO DE TRAYECTORIA
% = = = = = = = = = = = = = = = = = = = = = = = = = 


\textbf{\large Trayectorias del CU}{\large \par}

\textit{\large Trayectoria Principal}{\large {} }{\large \par}

%Solo hay que cambiar el nombre de la imagen dependiendo de si es actor o sistema

\begin{tabular}{ccl}
 &  & \tabularnewline
1.  & \includegraphics{actor}  & \multicolumn{1}{p{12cm}}{ Presiona el botón {[}Gestión de Materias Primas{]} en la pantalla PI}\tabularnewline
2.  & \includegraphics{sistema}  & \multicolumn{1}{p{12cm}}{Carga los datos de las materias primas registradas en la Base de datos. 		
{[}Trayectoria Alternativa A{]}}\tabularnewline
3.  & \includegraphics{sistema}  & \multicolumn{1}{p{12cm}}{Despliega la pantalla PI1}\tabularnewline
 &  & \multicolumn{1}{p{12cm}}{.... Fin del caso de uso}\tabularnewline
\end{tabular}

\textit{Trayectoria Alternativa A}

Condición: No se pudieron cargar los datos de las Materias Primas

\begin{tabular}{ccl}
 &  & \tabularnewline
1.  & \includegraphics{sistema}  & \multicolumn{1}{p{12cm}}{ Muestra el mensaje de error MSG4 'No se pudo establecer conexión
con la base de datos'}\tabularnewline
2.  & \includegraphics{sistema}  & \multicolumn{1}{p{12cm}}{Regresa a la pantalla PI}\tabularnewline
 &  & \multicolumn{1}{p{12cm}}{.... Fin del caso de uso}\tabularnewline
\end{tabular}

\newpage{}
%====================
%Nuevo caso de uso
%====================

\subsubsection{CUI4.1:Agregar Materia Prima}

\textbf{\large Resumen}{\large }\\
{\large{} Este Caso de uso tiene como objetivo permitir al usuario registrar una nueva materia prima}\\


% = = = = = = = = = = = = = = = = = = = = = = = = =
%	INICIO DE TABLA
% = = = = = = = = = = = = = = = = = = = = = = = = = 
% Se utiliza para que no se salga la tabla de la hoja, porque sino se autodimensiona
%\multicolumn{1}{p{10cm}||}{Contenido}
%
\begin{table}[!ht]
\begin{centering}
\begin{tabular}{|c||c|l|}
\hline 
\multicolumn{2}{|c|}{Caso de Uso:} & CUI4.1.1: Agregar Materia Prima\tabularnewline
\hline 
\multicolumn{3}{|>{\columncolor[gray]{0.7}}c}{Resumen de Atributos}\tabularnewline
\hline 
\multicolumn{2}{|c|}{Autor} & \multicolumn{1}{p{10cm}||}{Mar Escoto Diana}\tabularnewline
\hline 
\multicolumn{2}{|c|}{Actor} & \multicolumn{1}{p{10cm}||}{Personal de Inventario}\tabularnewline
\hline 
\multicolumn{2}{|c|}{Propósito} & \multicolumn{1}{p{10cm}||}{Registrar una materia prima en el sistema}\tabularnewline
\hline 
\multicolumn{2}{|c|}{Entradas} & \multicolumn{1}{p{10cm}||}{}\tabularnewline
\hline 
\multicolumn{2}{|c|}{Salidas} & \multicolumn{1}{p{10cm}||}{Nueva materia prima agregada a la base de datos}\tabularnewline
\hline 
\multicolumn{2}{|c|}{Pre-condiciones} & \multicolumn{1}{p{10cm}||}{Que el usuario este dado de alta en el área de inventario,. Que la materia prima no este registrada anteriormente}\tabularnewline
\hline 
\multicolumn{2}{|c|}{Pos-condiciones} & \multicolumn{1}{p{10cm}||}{La base de datos debe de estar disponible}\tabularnewline
\hline 
\multicolumn{2}{|c|}{Errores} & \multicolumn{1}{p{10cm}||}{ MSG4 Error al conectar con la base de datos. MSG8 'Falta llenar campos obliglatorios'. MSG7 'Formato Incorrecto de los datos'}\tabularnewline
\hline 
\multicolumn{2}{|c|}{Tipo} & \multicolumn{1}{p{10cm}||}{Secundario}\tabularnewline
\hline 
\multicolumn{2}{|c|}{Fuente} & \multicolumn{1}{p{10cm}||}{Basado en el funcionamiento etc.}\tabularnewline
\hline 
\end{tabular}
\par\end{centering}

\caption{Caso de Uso I4.1: Agregar Materia Prima}
\label{tab:CasosdeUso:nombredecasodeuso} 
\end{table}


% = = = = = = = = = = = = = = = = = = = = = = = = =
%	FIN DE TABLA
% = = = = = = = = = = = = = = = = = = = = = = = = =
% = = = = = = = = = = = = = = = = = = = = = = = = =
%	INICIO DE TRAYECTORIA
% = = = = = = = = = = = = = = = = = = = = = = = = = 


\textbf{\large Trayectorias del CU}{\large \par}

\textit{\large Trayectoria Principal}{\large {} }{\large \par}



\begin{tabular}{ccl}
 &  & \tabularnewline
1.  & \includegraphics{actor}  & \multicolumn{1}{p{12cm}}{ Presiona el ícono {[}Agregar Materia Prima{]} en la pantalla PI2}\tabularnewline
2.  & \includegraphics{sistema}  & \multicolumn{1}{p{12cm}}{Despliega la pantalla PI1.1 con el formulario a llenar.}\tabularnewline
3.  & \includegraphics{actor}  & \multicolumn{1}{p{12cm}}{Llena los datos del formulario}\tabularnewline
4.  & \includegraphics{actor}  & \multicolumn{1}{p{12cm}}{Da clic en el botón de aceptar{[}Trayectoria Alternativa A{]}}\tabularnewline
5.  & \includegraphics{sistema}  & \multicolumn{1}{p{12cm}}{Verifica si los campos obligatorios se encuentran con datos{[}Trayectoria Alternativa B{]}, verifica si los datos se llenaron correctamente{[}Trayectoria Alternativa C{]}}\tabularnewline
6. &\includegraphics{sistema} & \multicolumn{1}{p{12cm}}{Envía mensaje de confirmación MSG1}\tabularnewline
 &  & \multicolumn{1}{p{12cm}}{.... Fin del caso de uso}\tabularnewline
\end{tabular}

\textit{Trayectoria Alternativa A}

Condición: No se pudieron cargar los datos ingresados a la base de datos

\begin{tabular}{ccl}
 &  & \tabularnewline
1.  & \includegraphics{sistema}  & \multicolumn{1}{p{12cm}}{ Muestra el mensaje de error MSG4 'No se pudo establecer conexión
con la base de datos'}\tabularnewline
2.  & \includegraphics{sistema}  & \multicolumn{1}{p{12cm}}{Regresa a la pantalla PI2}\tabularnewline
 &  & \multicolumn{1}{p{12cm}}{.... Fin del caso de uso}\tabularnewline
\end{tabular}
\\
\textit{Trayectoria Alternativa B}

Condición: Faltan campos obligatorios por llenar

\begin{tabular}{ccl}
 &  & \tabularnewline
1.  & \includegraphics{sistema}  & \multicolumn{1}{p{12cm}}{ Muestra el mensaje de error MSG8 'Falta llenar campos obliglatorios'}\tabularnewline
2.  & \includegraphics{sistema}  & \multicolumn{1}{p{12cm}}{Regresa a la pantalla PI2}\tabularnewline
 &  & \multicolumn{1}{p{12cm}}{.... Fin del caso de uso}\tabularnewline
\end{tabular}

\textit{Trayectoria Alternativa C}

Condición: El formato de los datos es incorrecto

\begin{tabular}{ccl}
 &  & \tabularnewline
1.  & \includegraphics{sistema}  & \multicolumn{1}{p{12cm}}{ Muestra el mensaje de error MSG7 'Formato Incorrecto de los datos'}\tabularnewline
2.  & \includegraphics{sistema}  & \multicolumn{1}{p{12cm}}{Regresa a la pantalla PI2}\tabularnewline
 &  & \multicolumn{1}{p{12cm}}{.... Fin del caso de uso}\tabularnewline
\end{tabular}
\newpage{}

%====================
%Nuevo caso de uso
%====================

\subsubsection{CUI4.2:Modificar Materia Prima}

\textbf{\large Resumen}{\large }\\
{\large{} Este Caso de uso tiene como objetivo permitir al usuario modificar registros de materias primas}\\


% = = = = = = = = = = = = = = = = = = = = = = = = =
%    INICIO DE TABLA
% = = = = = = = = = = = = = = = = = = = = = = = = = 
% Se utiliza para que no se salga la tabla de la hoja, porque sino se autodimensiona
%\multicolumn{1}{p{10cm}||}{Contenido}
%
\begin{table}[!ht]
\begin{centering}
\begin{tabular}{|c||c|l|}
\hline 
\multicolumn{2}{|c|}{Caso de Uso:} & CUI4.2: Modificar Materia Prima\tabularnewline
\hline 
\multicolumn{3}{|>{\columncolor[gray]{0.7}}c}{Resumen de Atributos}\tabularnewline
\hline 
\multicolumn{2}{|c|}{Autor} & \multicolumn{1}{p{10cm}||}{Mar Escoto Diana}\tabularnewline
\hline 
\multicolumn{2}{|c|}{Actor} & \multicolumn{1}{p{10cm}||}{Personal de Inventario}\tabularnewline
\hline 
\multicolumn{2}{|c|}{Propósito} & \multicolumn{1}{p{10cm}||}{Modificar información de una materia prima en el sistema}\tabularnewline
\hline 
\multicolumn{2}{|c|}{Entradas} & \multicolumn{1}{p{10cm}||}{}\tabularnewline
\hline 
\multicolumn{2}{|c|}{Salidas} & \multicolumn{1}{p{10cm}||}{Modificación de los registros de materia prima en la base de datos}\tabularnewline
\hline 
\multicolumn{2}{|c|}{Pre-condiciones} & \multicolumn{1}{p{10cm}||}{Que el usuario este dado de alta en el área de inventario,. Que la materia prima este registrada anteriormente}\tabularnewline
\hline 
\multicolumn{2}{|c|}{Pos-condiciones} & \multicolumn{1}{p{10cm}||}{La base de datos debe de estar disponible}\tabularnewline
\hline 
\multicolumn{2}{|c|}{Errores} & \multicolumn{1}{p{10cm}||}{MSG4 Error al conectar con la base de datos.MSG7 'Formato Incorrecto de los datos'.}\tabularnewline
\hline 
\multicolumn{2}{|c|}{Tipo} & \multicolumn{1}{p{10cm}||}{Secundario}\tabularnewline
\hline 
\multicolumn{2}{|c|}{Fuente} & \multicolumn{1}{p{10cm}||}{Basado en el funcionamiento etc.}\tabularnewline
\hline 
\end{tabular}
\par\end{centering}

\caption{Caso de Uso I4.2: Modificar Materia Prima}
\label{tab:CasosdeUso:nombredecasodeuso} 
\end{table}


% = = = = = = = = = = = = = = = = = = = = = = = = =
%	FIN DE TABLA
% = = = = = = = = = = = = = = = = = = = = = = = = =
% = = = = = = = = = = = = = = = = = = = = = = = = =
%	INICIO DE TRAYECTORIA
% = = = = = = = = = = = = = = = = = = = = = = = = = 


\textbf{\large Trayectorias del CU}{\large \par}

\textit{\large Trayectoria Principal}{\large {} }{\large \par}

%Solo hay que cambiar el nombre de la imagen dependiendo de si es actor o sistema

\begin{tabular}{ccl}
 &  & \tabularnewline
1.  & \includegraphics{actor}  & \multicolumn{1}{p{12cm}}{ Presiona el ícono {[}Editar Materia Prima{]} en la pantalla PI1}\tabularnewline
2.  & \includegraphics{sistema}  & \multicolumn{1}{p{12cm}}{Despliega la pantalla PI1 con los datos posibles a modificar}\tabularnewline
3.  & \includegraphics{actor}  & \multicolumn{1}{p{12cm}}{Modifica los registros de la base de datos}\tabularnewline
4.  & \includegraphics{actor}  & \multicolumn{1}{p{12cm}}{Da clic en el botón de aceptar{[}Trayectoria Alternativa A{]}}\tabularnewline
5.  & \includegraphics{sistema}  & \multicolumn{1}{p{12cm}}{Verifica verifica si los datos modificados se llenaron correctamente{[}Trayectoria Alternativa B{]}}\tabularnewline
6. &\includegraphics{sistema} & \multicolumn{1}{p{12cm}}{Envía mensaje de confirmación MSG1}\tabularnewline
 &  & \multicolumn{1}{p{12cm}}{.... Fin del caso de uso}\tabularnewline
\end{tabular}

\textit{Trayectoria Alternativa A}

Condición: No se pudieron cargar los datos ingresados a la base de datos

\begin{tabular}{ccl}
 &  & \tabularnewline
1.  & \includegraphics{sistema}  & \multicolumn{1}{p{12cm}}{ Muestra el mensaje de error MSG4 'No se pudo establecer conexión
con la base de datos'}\tabularnewline
2.  & \includegraphics{sistema}  & \multicolumn{1}{p{12cm}}{Regresa a la pantalla PI2}\tabularnewline
 &  & \multicolumn{1}{p{12cm}}{.... Fin del caso de uso}\tabularnewline
\end{tabular}
\\

\textit{Trayectoria Alternativa B}

Condición: El formato de los datos es incorrecto

\begin{tabular}{ccl}
 &  & \tabularnewline
1.  & \includegraphics{sistema}  & \multicolumn{1}{p{12cm}}{ Muestra el mensaje de error MSG7 'Formato Incorrecto de los datos'}\tabularnewline
2.  & \includegraphics{sistema}  & \multicolumn{1}{p{12cm}}{Regresa a la pantalla PI2}\tabularnewline
 &  & \multicolumn{1}{p{12cm}}{.... Fin del caso de uso}\tabularnewline
\end{tabular}
\newpage{}


%====================
%Nuevo caso de uso
%====================

\subsubsection{CUI4.3:Consultar Materia Prima}

\textbf{\large Resumen}{\large }\\
{\large{} Este Caso de uso tiene como objetivo permitir al usuario consultar registros de materias primas}\\


% = = = = = = = = = = = = = = = = = = = = = = = = =
%    INICIO DE TABLA
% = = = = = = = = = = = = = = = = = = = = = = = = = 
% Se utiliza para que no se salga la tabla de la hoja, porque sino se autodimensiona
%\multicolumn{1}{p{10cm}||}{Contenido}
%
\begin{table}[!ht]
\begin{centering}
\begin{tabular}{|c||c|l|}
\hline 
\multicolumn{2}{|c|}{Caso de Uso:} & CUI4.3: Buscar Materia Prima\tabularnewline
\hline 
\multicolumn{3}{|>{\columncolor[gray]{0.7}}c}{Resumen de Atributos}\tabularnewline
\hline 
\multicolumn{2}{|c|}{Autor} & \multicolumn{1}{p{10cm}||}{Mar Escoto Diana}\tabularnewline
\hline 
\multicolumn{2}{|c|}{Actor} & \multicolumn{1}{p{10cm}||}{Personal de Inventario}\tabularnewline
\hline 
\multicolumn{2}{|c|}{Propósito} & \multicolumn{1}{p{10cm}||}{Buscar información de una materia prima en el sistema}\tabularnewline
\hline 
\multicolumn{2}{|c|}{Entradas} & \multicolumn{1}{p{10cm}||}{}\tabularnewline
\hline 
\multicolumn{2}{|c|}{Salidas} & \multicolumn{1}{p{10cm}||}{Búsqueda de los registros de materia prima en la base de datos}\tabularnewline
\hline 
\multicolumn{2}{|c|}{Pre-condiciones} & \multicolumn{1}{p{10cm}||}{Que el usuario este dado de alta en el área de inventario,. Que la materia prima este registrada anteriormente}\tabularnewline
\hline 
\multicolumn{2}{|c|}{Pos-condiciones} & \multicolumn{1}{p{10cm}||}{La base de datos debe de estar disponible}\tabularnewline
\hline 
\multicolumn{2}{|c|}{Errores} & \multicolumn{1}{p{10cm}||}{MSG4 Error al conectar con la base de datos.MSG3 Datos no encontrados}\tabularnewline
\hline 
\multicolumn{2}{|c|}{Tipo} & \multicolumn{1}{p{10cm}||}{Secundario}\tabularnewline
\hline 
\multicolumn{2}{|c|}{Fuente} & \multicolumn{1}{p{10cm}||}{Basado en el funcionamiento etc.}\tabularnewline
\hline 
\end{tabular}
\par\end{centering}

\caption{Caso de Uso I4.3: Buscar Materia Prima}
\label{tab:CasosdeUso:nombredecasodeuso} 
\end{table}


% = = = = = = = = = = = = = = = = = = = = = = = = =
%    FIN DE TABLA
% = = = = = = = = = = = = = = = = = = = = = = = = =
% = = = = = = = = = = = = = = = = = = = = = = = = =
%	INICIO DE TRAYECTORIA
% = = = = = = = = = = = = = = = = = = = = = = = = = 


\textbf{\large Trayectorias del CU}{\large \par}

\textit{\large Trayectoria Principal}{\large {} }{\large \par}

%Solo hay que cambiar el nombre de la imagen dependiendo de si es actor o sistema

\begin{tabular}{ccl}
 &  & \tabularnewline
1.  & \includegraphics{actor}  & \multicolumn{1}{p{12cm}}{ Ingresa la palabra clave en la caja de texto{[}Buscar...{]} en la pantalla PI1}\tabularnewline
2.   & \includegraphics{actor}  & \multicolumn{1}{p{12cm}}{Ingresa la palabra clave de búsqueda y selecciona {[}Aceptar{]}}\tabularnewline
3.  & \includegraphics{sistema}  & \multicolumn{1}{p{12cm}}{Verifica la existencia de la palabra clave en las materias primas registradas en la base de datos{[}Trayectoria Alternativa B {]} {[}Trayectoria Alternativa A{]}}\tabularnewline
4.  & \includegraphics{sistema}  & \multicolumn{1}{p{12cm}}{Despliega la pantalla con la información de la materia prima{[}Trayectoria Alternativa B{]}}\tabularnewline
 &  & \multicolumn{1}{p{12cm}}{.... Fin del caso de uso}\tabularnewline
\end{tabular}


\textit{Trayectoria Alternativa A}

Condición: La materia prima no existe en la base de datos

\begin{tabular}{ccl}
 &  & \tabularnewline
1.  & \includegraphics{sistema}  & \multicolumn{1}{p{12cm}}{ Muestra el mensaje de error MSG3 'Datos no encontrados'}\tabularnewline
2.  & \includegraphics{sistema}  & \multicolumn{1}{p{12cm}}{Regresa a la pantalla PI2}\tabularnewline
 &  & \multicolumn{1}{p{12cm}}{.... Fin del caso de uso}\tabularnewline
\end{tabular}
\\

\textit{Trayectoria Alternativa B}

Condición: No se pudieron cargar los datos de la base de datos

\begin{tabular}{ccl}
 &  & \tabularnewline
1.  & \includegraphics{sistema}  & \multicolumn{1}{p{12cm}}{ Muestra el mensaje de error MSG4 'No se pudo establecer conexión
con la base de datos'}\tabularnewline
2.  & \includegraphics{sistema}  & \multicolumn{1}{p{12cm}}{Regresa a la pantalla PI2}\tabularnewline
 &  & \multicolumn{1}{p{12cm}}{.... Fin del caso de uso}\tabularnewline
\end{tabular}
\\

\newpage{}
\end{document}
