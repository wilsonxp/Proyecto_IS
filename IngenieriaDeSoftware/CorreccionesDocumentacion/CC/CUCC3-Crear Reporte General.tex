\documentclass[10pt,spanish]{article}
\usepackage[utf8x]{inputenc}
\usepackage[letterpaper]{geometry}
\geometry{verbose}
\usepackage{amsmath}
\usepackage{amssymb}
\usepackage{graphicx}

\makeatletter

\providecommand{\tabularnewline}{\\}

\usepackage{ucs}\usepackage[spanish]{babel}
\usepackage{amsfonts}\usepackage{colortbl}% = = = = = = = = = = = = = = = = = = = = = = = = =
% INICIO EL DOCUMENTO
% = = = = = = = = = = = = = = = = = = = = = = = = =


\makeatother

\usepackage{babel}
\addto\shorthandsspanish{\spanishdeactivate{~<>}}

\begin{document}


	\section{Documentación de Casos de Uso}

% = = = = = = = = = = = = = = = = = = = = = = = = = = = = = =
% Consultar pedidos en espera
% = = = = = = = = = = = = = = = = = = = = = = = = = = = = = =
		\subsection{CUCC3: Crear Reporte General}

		\textbf{\large Resumen}{\large }\\
		{Este caso de uso tiene como objetivo permitir al usuario crear reportes generales de los problemas que se han presentado en un determinado periodo de tiempo. Los reportes podrán contener
		datos acerca de los problemas registrados en áreas específicas o bien, en todas las áreas  
		y podrán ser ordenados en base a varios criterios con la finalidad de entregar a los
		usuarios reportes personalizados.}\\
		{\large \par}

		% = = = = = = = = = = = = = = = = = = = = = = = = =
		%	INICIO DE TABLA
		% = = = = = = = = = = = = = = = = = = = = = = = = = 
		\begin{table}[!ht]
		\begin{centering}
		\begin{tabular}{|c||c|l|}
		\hline 
		\multicolumn{2}{|c|}{Caso de Uso:} & CUCC3: Crear Reporte General   \tabularnewline
		\hline 
		\multicolumn{3}{|>{\columncolor[gray]{0.7}}c}{Resumen de Atributos}\tabularnewline
		\hline 
		\multicolumn{2}{|c|}{Actor} & Usuario de Control de Calidad\tabularnewline
		\hline 
		\multicolumn{2}{|c|}{Propósito} & \multicolumn{1}{p{12cm}|}{Permitir al usuario crear reportes de los problemas de calidad presentados en diversas áreas de la empresa durante un lapso de tiempo determinado.}\tabularnewline
		\hline 
		\multicolumn{2}{|c|}{Entradas} & \multicolumn{1}{p{12cm}|}{
		Para crear un reporte general de problemas, el usuario deberá:\newline		
		- Seleccionar con el mouse por medio de un Datepicker la fecha desde la cual empezará el 		 	reporte\newline
		- Seleccionar con el mouse por medio de un Datepicker la fecha hasta la cual abarcará el reporte\newline
		- Seleccionar por medio de los controles CheckBox las áreas que abarcará el reporte. \newline
		- Seleccionar por medio de un ComboBox la forma en que se ordenarán los datos resultantes.  
}\tabularnewline
		\hline 
		\multicolumn{2}{|c|}{Salidas} & \multicolumn{1}{p{12cm}|}{

    - Se mostrará en pantalla el mensaje MSG1: Mensaje de Confirmación.\newline
	- Además, se mostrará el reporte correspondiente en formato PDF según la estructura
	especificada en la regla de negocio BR5: Formato de Reportes.
}\tabularnewline
\hline 
		\multicolumn{2}{|c|}{Pre-condiciones} & \multicolumn{1}{p{12cm}|}{

    - El usuario debe haber iniciado sesión previamente como usuario de Control de Calidad. \newline
	- Deben existir registros de problemas en la base de datos.
}\tabularnewline				

\hline 
		\multicolumn{2}{|c|}{Post-condiciones} & \multicolumn{1}{p{12cm}|}{

    - El formulario es limpiado para permitir la generación de un nuevo reporte.    
}\tabularnewline				
		
\hline 
		\multicolumn{2}{|c|}{Errores} & \multicolumn{1}{p{12cm}|}{

    - El sistema muestra en pantalla el mensaje de error MSG3: Datos no encontrados, cuando a pesar de tener acceso a la base de datos, ésta no contenga registros sobre la información solicitada.\newline    
	- El sistema muestra en pantalla el mensaje de error MSG4: Error al conectar con la base de datos, cuando al intentar acceder a los datos necesarios para el reporte, no se logre hacer conexión con la base de datos.\newline
	- El sistema muestra en pantalla el mensaje de error MSG8: Campos obligatorios, cuando al dar clic en el botón [Crear reporte] no se haya llenado alguno de los campos.\newline
	- El sistema muestra en pantalla los mensajes de error MSG11 o MSG12 referentes a errores en las fechas cuando el usuario ingrese datos erróneos en los campos de fecha.
}\tabularnewline						
		
		\hline 		
		\multicolumn{2}{|c|}{Autor} & Durán Pineda Mario Ángel\tabularnewline
		
		\hline 				
		\end{tabular}
		\par\end{centering}
		
	
	\label{tab:CasosdeUso:nombredecasodeuso} 
	\end{table}


	% = = = = = = = = = = = = = = = = = = = = = = = = =
	%	FIN DE TABLA
	% = = = = = = = = = = = = = = = = = = = = = = = = =
	% = = = = = = = = = = = = = = = = = = = = = = = = =
	%	INICIO DE TRAYECTORIA
	% = = = = = = = = = = = = = = = = = = = = = = = = = 
	\newpage
	\textbf{\large Trayectorias del CU\\}{\large \par}
	\textit{\large Trayectoria Principal}{\large{} }{\large \par}
	\begin{tabular}{ccl}
	 &  & \tabularnewline
	1. & \includegraphics{actor} & El usuario solicita Crear un Reporte General de Problemas dando clic al botón   \tabularnewline
	& &  [Crear Reporte General] de la pantalla PCC6: Menú Principal de Control de Calidad.\tabularnewline

	2. & \includegraphics{sistema} &  El sistema despliega la pantalla PCC5: Creación de Reportes Generales.\tabularnewline
	
	3. & \includegraphics{actor} & El usuario selecciona la fecha a partir de la cual abarcará el reporte y la fecha de término \tabularnewline
	&  & del mismo.\tabularnewline
	
	4. & \includegraphics{actor} & El usuario selecciona las áreas que abarcará el reporte de problemas.\tabularnewline 

	5. & \includegraphics{actor} & El usuario selecciona la forma en que se ordenarán los datos de su reporte.  \tabularnewline
	
	6. & \includegraphics{actor} & El usuario da clic al botón [Crear reporte]. \tabularnewline	
	
	7. & \includegraphics{sistema} & El sistema valida que se hayan llenado todos los campos. [Trayectoria A] \tabularnewline
	
	8. & \includegraphics{sistema} & El sistema valida que se hayan ingresado correctamente las fechas. [Trayectoria B] \tabularnewline	

    9. & \includegraphics{sistema} & El sistema accede a la base de datos [Trayectoria C] y obtiene la información solicitada \tabularnewline
    & & por el usuario [Trayectoria D].   \tabularnewline	

    10. & \includegraphics{sistema} & El sistema despliega en pantalla el reporte de problemas personalizado en formato PDF.       \tabularnewline
	
	\tabularnewline
	 &  & - - - - Fin del caso de uso\tabularnewline \\
	\end{tabular}

\newpage
	\textit{\large \\Trayectoria Alternativa A: Campos obligatorios sin llenar.}
	
	\begin{tabular}{ccl}
	& & \tabularnewline
			
	A.1 & \includegraphics{sistema} & El sistema detecta que existen campos obligatorios sin llenar\tabularnewline
	
	A.2 & \includegraphics{sistema} & El sistema despliega el mensaje de error MSG8: Campos obligatorios sin llenar, e \tabularnewline
	& & indica al usuario los campos que faltan por llenar. \tabularnewline

    A.3 & \includegraphics{actor} & El usuario llena los campos que estaban vacíos.\tabularnewline
	\tabularnewline
	
	A.4 & &	Regresa a la trayectoria principal en el punto 6.\tabularnewline

	
	\tabularnewline\tabularnewline	
	\end{tabular}
	
	\textit{\large \\Trayectoria Alternativa B: Error de fechas.}
	
	\begin{tabular}{ccl}
	& & \tabularnewline
			
	B.1 & \includegraphics{sistema} & El sistema detecta inconsistencias en las fechas de inicio y/o fin del reporte.\tabularnewline
	
	B.2 & \includegraphics{sistema} & El sistema despliega el mensaje de error MSG11: Fecha de inicio posterior a Fecha fin,  \tabularnewline
	& & o MSG12: Fecha posterior al día actual, según el problema que se presente \tabularnewline

    B.3. & \includegraphics{actor} & El usuario selecciona nuevamente una fecha de inicio y una fecha de término \tabularnewline
	&  & para el reporte.\tabularnewline\tabularnewline
	
	B.4 & &	Regresa a la trayectoria principal en el punto 6.\tabularnewline

	
	\tabularnewline\tabularnewline	
	\end{tabular}

\textit{\large \\Trayectoria Alternativa C: Error de acceso a la BD.}

\begin{tabular}{ccl}
	& & \tabularnewline
	
	C.1 & \includegraphics{sistema} & El sistema detecta un error de conexión a la base de datos.\tabularnewline
		
	C.2 & \includegraphics{sistema} & El sistema despliega en pantalla el mensaje MSG4: Error al conectar con la Base de Datos. 	\tabularnewline	\tabularnewline	
		
	C.3 & & Regresar a la trayectoria principal en el punto 2.\tabularnewline

	
	\tabularnewline\tabularnewline	
	\end{tabular}

\newpage
\textit{\large \\Trayectoria Alternativa D: No se encontraron datos.}

\begin{tabular}{ccl}
	& & \tabularnewline
	
	D.1 & \includegraphics{sistema} & El sistema encuentra que la base de datos no contiene registros de la información buscada.\tabularnewline
		
	D.2 & \includegraphics{sistema} & El sistema despliega en pantalla el mensaje MSG3: Datos no encontrados. 	\tabularnewline	
	
	D.3 & \includegraphics{actor} & El usuario llena el formulario con nuevos datos. 	\tabularnewline	\tabularnewline	
		
	D.4 & & Regresar a la trayectoria principal en el punto 6.\tabularnewline

	
	\tabularnewline\tabularnewline	
	\end{tabular}


\end{document}
