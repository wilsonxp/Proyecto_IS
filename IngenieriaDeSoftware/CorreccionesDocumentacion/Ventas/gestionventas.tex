%% LyX 2.0.3 created this file.  For more info, see http://www.lyx.org/.
%% Do not edit unless you really know what you are doing.
\documentclass[10pt,spanish]{article}
\usepackage[utf8x]{inputenc}
\usepackage[letterpaper]{geometry}
\geometry{verbose}
\usepackage{amsmath}
\usepackage{amssymb}
\usepackage{graphicx}

\makeatletter

%%%%%%%%%%%%%%%%%%%%%%%%%%%%%% LyX specific LaTeX commands.
%% Because html converters don't know tabularnewline
\providecommand{\tabularnewline}{\\}

%%%%%%%%%%%%%%%%%%%%%%%%%%%%%% User specified LaTeX commands.

\usepackage{ucs}\usepackage[spanish]{babel}
\usepackage{amsfonts}\usepackage{colortbl}% = = = = = = = = = = = = = = = = = = = = = = = = =
% INICIO EL DOCUMENTO
% = = = = = = = = = = = = = = = = = = = = = = = = =



\usepackage{babel}
\addto\shorthandsspanish{\spanishdeactivate{~<>}}





\usepackage{babel}
\addto\shorthandsspanish{\spanishdeactivate{~<>}}



\usepackage{babel}
\addto\shorthandsspanish{\spanishdeactivate{~<>}}





\usepackage{babel}
\addto\shorthandsspanish{\spanishdeactivate{~<>}}





\usepackage{babel}
\addto\shorthandsspanish{\spanishdeactivate{~<>}}





\usepackage{babel}
\addto\shorthandsspanish{\spanishdeactivate{~<>}}





\usepackage{babel}
\addto\shorthandsspanish{\spanishdeactivate{~<>}}





\usepackage{babel}
\addto\shorthandsspanish{\spanishdeactivate{~<>}}





\usepackage{babel}
\addto\shorthandsspanish{\spanishdeactivate{~<>}}



\usepackage{babel}
\addto\shorthandsspanish{\spanishdeactivate{~<>}}





\usepackage{babel}
\addto\shorthandsspanish{\spanishdeactivate{~<>}}





\usepackage{babel}
\addto\shorthandsspanish{\spanishdeactivate{~<>}}





\usepackage{babel}
\addto\shorthandsspanish{\spanishdeactivate{~<>}}





\usepackage{babel}
\addto\shorthandsspanish{\spanishdeactivate{~<>}}





\usepackage{babel}
\addto\shorthandsspanish{\spanishdeactivate{~<>}}





\usepackage{babel}
\addto\shorthandsspanish{\spanishdeactivate{~<>}}



\usepackage{babel}
\addto\shorthandsspanish{\spanishdeactivate{~<>}}



\makeatother

\usepackage{babel}
\addto\shorthandsspanish{\spanishdeactivate{~<>}}

\begin{document}
\tableofcontents{}

\pagebreak{}


\section{Módulo Ventas.}


\subsection{CUV1: Gestionar Ventas}

\textbf{\large Resumen}\\
 {\large {} {} {} {} {} {} {} Este Caso de uso tiene como objetivo permitir al actor registrar, buscar, modificar y Cancelar Venta.}\\


% = = = = = = = = = = = = = = = = = = = = = = = = =
%	INICIO DE TABLA
% = = = = = = = = = = = = = = = = = = = = = = = = = 
% Se utiliza para que no se salga la tabla de la hoja, porque sino se autodimensiona
%\multicolumn{1}{p{10cm}||}{Contenido}
\begin{table}[!ht]
\begin{centering}
\begin{tabular}{|c||c|l|}
\hline 
\multicolumn{2}{|c|}{Caso de Uso:} & CUV1: Gestionar Ventas\tabularnewline
\hline 
\multicolumn{3}{|>{\columncolor[gray]{0.7}}c}{Resumen de Atributos}\tabularnewline
\hline 
\multicolumn{2}{|c|}{Autor} & \multicolumn{1}{p{10cm}||}{Cabrera Alvarez Estefany Viridiana}\tabularnewline
\hline 
\multicolumn{2}{|c|}{Actor} & \multicolumn{1}{p{10cm}||}{Usuario de Ventas}\tabularnewline
\hline 
\multicolumn{2}{|c|}{Propósito} & \multicolumn{1}{p{10cm}||}{Vizualisar ventas recientes y Acceder a opciones de Agregar, modificar, cancelar y buscar ventas. }\tabularnewline
\hline 
\multicolumn{2}{|c|}{Entradas} & \multicolumn{1}{p{10cm}||}{Ninguna}\tabularnewline
\hline 
\multicolumn{2}{|c|}{Salidas} & \multicolumn{1}{p{10cm}||}{Interfaz Correspondiente a gestionar Ventas}\tabularnewline
\hline 
\multicolumn{2}{|c|}{Pre-condiciones} & \multicolumn{1}{p{10cm}||}{El usuario de ventas debe estar autentificado}\tabularnewline
\hline 
\multicolumn{2}{|c|}{Pos-condiciones} & \multicolumn{1}{p{10cm}||}{La base de datos debe estar disponible}\tabularnewline
\hline 
\multicolumn{2}{|c|}{Errores} & \multicolumn{1}{p{10cm}||}{MSG4 Error al conectar con la base de datos.}\tabularnewline
\hline 
\multicolumn{2}{|c|}{Tipo} & \multicolumn{1}{p{10cm}||}{Primario}\tabularnewline
\hline 
\multicolumn{2}{|c|}{Fuente} & \multicolumn{1}{p{10cm}||}{Basado en el funcionamiento.}\tabularnewline
\hline 
\end{tabular}
\par\end{centering}

\caption{CUV 1: Gestionar Ventas}


\label{CUV1.0} 
\end{table}


% = = = = = = = = = = = = = = = = = = = = = = = = =
%	FIN DE TABLA
% = = = = = = = = = = = = = = = = = = = = = = = = =
% = = = = = = = = = = = = = = = = = = = = = = = = =
%	INICIO DE TRAYECTORIA
% = = = = = = = = = = = = = = = = = = = = = = = = = 


\textbf{\large Trayectorias del CU}{\large \par}

\textit{\large Trayectoria Principal}{\large {} }{\large \par}

%Solo hay que cambiar el nombre de la imagen dependiendo de si es actor o sistema


\begin{tabular}{ccl}
 &  & \tabularnewline
1.  & \includegraphics{actor}  & \multicolumn{1}{p{12cm}}{ Presiona el botón del menu {[}Gestionar Ventas{]}}\tabularnewline
2.  & \includegraphics{sistema}  & \multicolumn{1}{p{12cm}}{Carga los datos de las ventas registradas en la Base de   datos. {[}Trayectoria Alternativa A{]}}\tabularnewline
3.  & \includegraphics{sistema}  & \multicolumn{1}{p{12cm}}{Despliega la pantalla PV1 $"$ Gestionar Ventas$"$  }\tabularnewline
4.  & \includegraphics{actor}  & \multicolumn{1}{p{12cm}}{Da clic en el botón {[}Registrar ventas{]}{[}Trayectoria Alternativa B{]}{[}Trayectoria Alternativa C{]}{[}Trayectoria Alternativa D{]}}\tabularnewline
5.  & \includegraphics{sistema}  & \multicolumn{1}{p{12cm}}{Extiende a CUV1.1 $"$ Registrar Venta$"$ }\tabularnewline
 &  & \multicolumn{1}{p{12cm}}{.... Fin del caso de uso}\tabularnewline
\end{tabular}

\textit{Trayectoria Alternativa A}

Condición: No se pudo conectar con la base de datos.

\begin{tabular}{ccl}
 &  & \tabularnewline
1.  & \includegraphics{sistema}  & \multicolumn{1}{p{12cm}}{ Muestra el mensaje de error MSG4 'No se pudo establecer conexión con la base de datos'}\tabularnewline
2.  & \includegraphics{sistema}  & \multicolumn{1}{p{12cm}}{Regresa a la pantalla PV0 $"$ Inicio de Ventas$"$ }\tabularnewline
 &  & \multicolumn{1}{p{12cm}}{.... Fin del caso de uso}\tabularnewline
\end{tabular}

\textit{Trayectoria Alternativa B}

Condición: El actor escribe el idventa a buscar en el textbox y da clic en buscar.

\begin{tabular}{ccl}
 &  & \tabularnewline
1.  & \includegraphics{sistema} & \multicolumn{1}{p{12cm}}{ Extiende al CUV1.2 $"$  Consultar Ventas$"$ }\tabularnewline
 	
 &  & \multicolumn{1}{p{12cm}}{.... Fin del caso de uso}\tabularnewline
\end{tabular}

\textit{Trayectoria Alternativa C}

Condición: El actor da clic en el ícono de {[}cancelar{]} en una venta.

\begin{tabular}{ccl}
 &  & \tabularnewline
1. & \includegraphics{sistema} & \multicolumn{1}{p{12cm}}{ Extiende al CUV1.3 $"$ Cancelar Venta$"$ }\tabularnewline
 	
 &  & \multicolumn{1}{p{12cm}}{.... Fin del caso de uso}\tabularnewline
\end{tabular}

\textit{Trayectoria Alternativa D}

Condición: El actor da clic en el ícono de {[}modificar{]} en una venta.

\begin{tabular}{ccl}
 &  & \tabularnewline
1.  & \includegraphics{sistema} & \multicolumn{1}{p{12cm}}{ Extiende al CUV1.4 $"$ Modificar Venta$"$ }\tabularnewline
 	
 &  & \multicolumn{1}{p{12cm}}{.... Fin del caso de uso}\tabularnewline
\end{tabular}

\newpage{} 
\subsubsection{CUV1.1:Registrar Venta}
\textbf{\large Resumen}\\
 {\large {} {} {} {} {} {} {} Este Caso de uso tiene como objetivo permitir al usuario agregar una venta al sistema.}\\

% = = = = = = = = = = = = = = = = = = = = = = = = =
%	INICIO DE TABLA
% = = = = = = = = = = = = = = = = = = = = = = = = = 
% Se utiliza para que no se salga la tabla de la hoja, porque sino se autodimensiona
%\multicolumn{1}{p{10cm}||}{Contenido}
\begin{table}[!ht]
\begin{centering}
\begin{tabular}{|c||c|l|}
\hline 
\multicolumn{2}{|c|}{Caso de Uso:} & CUV1.1: Registrar Venta\tabularnewline
\hline 
\multicolumn{3}{|>{\columncolor[gray]{0.7}}c}{Resumen de Atributos}\tabularnewline
\hline 
\multicolumn{2}{|c|}{Autor} & \multicolumn{1}{p{10cm}||}{Cabrera Alvarez Estefany Viridiana}\tabularnewline
\hline 
\multicolumn{2}{|c|}{Actor} & \multicolumn{1}{p{10cm}||}{Usuario de Ventas}\tabularnewline
\hline 
\multicolumn{2}{|c|}{Propósito} & \multicolumn{1}{p{10cm}||}{Registrar una venta en el sistema.}\tabularnewline
\hline 
\multicolumn{2}{|c|}{Entradas} & \multicolumn{1}{p{10cm}||}{Datos del formulario de Agregar Ventas.}\tabularnewline
\hline 
\multicolumn{2}{|c|}{Salidas} & \multicolumn{1}{p{10cm}||}{MSG1 $"$Operación Exitosa"}\tabularnewline
\hline 
\multicolumn{2}{|c|}{Pre-condiciones} & \multicolumn{1}{p{10cm}||}{Cliente debe de estar registrado}\tabularnewline
\hline 
\multicolumn{2}{|c|}{Pos-condiciones} & \multicolumn{1}{p{10cm}||}{El sistema agrega la venta}\tabularnewline
\hline 
\multicolumn{2}{|c|}{Errores} & \multicolumn{1}{p{10cm}||}{MSG2 $"$ Mensaje de error$"$ , MSG4 $"$ Error al conectar con la base de datos$"$  ,MSG7 $"$ Formato incorrecto de los datos$"$ .}\tabularnewline
\hline 
\multicolumn{2}{|c|}{Tipo} & \multicolumn{1}{p{10cm}||}{Secundario}\tabularnewline
\hline 
\multicolumn{2}{|c|}{Fuente} & \multicolumn{1}{p{10cm}||}{Basado en el funcionamiento del CUV1.}\tabularnewline
\hline 
\end{tabular}
\par\end{centering}
\caption{Caso de Uso 1.1: Registrar Ventas}
\label{CUV1.1} 
\end{table}


% = = = = = = = = = = = = = = = = = = = = = = = = =
%	FIN DE TABLA
% = = = = = = = = = = = = = = = = = = = = = = = = =
% = = = = = = = = = = = = = = = = = = = = = = = = =
%	INICIO DE TRAYECTORIA
% = = = = = = = = = = = = = = = = = = = = = = = = = 
\textbf{\large Trayectorias del CU}{\large \par}

\textit{\large Trayectoria Principal}{\large {} }{\large \par}

%Solo hay que cambiar el nombre de la imagen dependiendo de si es actor o sistema
\begin{tabular}{ccl}

 &  & \tabularnewline
1.  & \includegraphics{sistema}  & \multicolumn{1}{p{12cm}}{Carga en el combobox de clientes elnombre de los clientes registrados en el sistema}\tabularnewline
2.  & \includegraphics{sistema}  & \multicolumn{1}{p{12cm}}{Carga en el combobox de productos el nombre de los prouctos registrados en el sistema}\tabularnewline
3.  & \includegraphics{sistema}  & \multicolumn{1}{p{12cm}}{Despliega la pantalla PV1.1 $"$  Registrar Venta$"$   con el formulario a llenar.}\tabularnewline
4.  & \includegraphics{actor}  & \multicolumn{1}{p{12cm}}{Selecciona el cliente del combobox.}\tabularnewline
5.  & \includegraphics{actor}  & \multicolumn{1}{p{12cm}}{Da clic en el botón de {[}Abrir Venta{]}}\tabularnewline
 6.  & \includegraphics{sistema}  & \multicolumn{1}{p{12cm}}{Verifica si la información es correcta. {[}Trayectoria Alternativa A{]}{[}Trayectoria Alternativa B{]}}\tabularnewline
7.  & \includegraphics{sistema}  & \multicolumn{1}{p{12cm}}{Habilita los controles del apartado de Articulo}\tabularnewline
8.  & \includegraphics{actor}  & \multicolumn{1}{p{12cm}}{Selecciona Producto e ingresa Cantidad}\tabularnewline
9.  & \includegraphics{sistema}  & \multicolumn{1}{p{12cm}}{Muestra la cantidad en almacen}\tabularnewline
10.  & \includegraphics{actor}  & \multicolumn{1}{p{12cm}}{Da clic en el botón de {[}Agregar Articulo{]}}\tabularnewline
11.   & \includegraphics{sistema}  & \multicolumn{1}{p{12cm}}{Verifica si los campos se llenaron correctamente. {[}Trayectoria Alternativa A{]}{[}Trayectoria Alternativa B{]}{[}Trayectoria Alternativa C{]}}\tabularnewline
12. & \includegraphics{sistema}  & \multicolumn{1}{p{12cm}}{Disminuye en el almacen y Guarda el articulo.}\tabularnewline
13. & \includegraphics{sistema}  & \multicolumn{1}{p{12cm}}{Desplega en la tabla los articulos agregados.} \tabularnewline
14. & \includegraphics{actor}  & \multicolumn{1}{p{12cm}}{Visualiza la tabla los articulos agregados. {[}Trayectoria Alternativa D{]}}\tabularnewline
15. & \includegraphics{actor} &\multicolumn{1}{p{12cm}}{Da clic en {[}Cerrar Venta{]} {[}Trayectoria Alternativa E{]}}\tabularnewline
16.  & \includegraphics{sistema}  & \multicolumn{1}{p{12cm}}{Muestra MSG1 Mensaje de confirmación $"$  La venta ha sido Agregada exitosamente.$"$  }\tabularnewline
\end{tabular}
\textit{Trayectoria Alternativa A}

Condición: Si la base de datos no esta disponible.

\begin{tabular}{ccl}
 &  & \tabularnewline
1.  & \includegraphics{sistema}  & \multicolumn{1}{p{12cm}}{Muestra MSG4 $"$ Error al conectar con la base de datos$"$  }\tabularnewline
2.  & \includegraphics{sistema}  & \multicolumn{1}{p{12cm}}{Regresa a la pantalla PV1.1 $"$  Registrar Venta$"$  }\tabularnewline
 &  & \multicolumn{1}{p{12cm}}{.... Fin del caso de uso}\tabularnewline
\end{tabular}
\textit{Trayectoria Alternativa B}

Condición: Si el formato de los datos es incorrecto.

\begin{tabular}{ccl}
 &  & \tabularnewline
1.  & \includegraphics{sistema}  & \multicolumn{1}{p{12cm}}{ Muestra el MSG7 $"$  Formato incorrecto de los datos.$"$  }\tabularnewline
2.  & \includegraphics{sistema}  & \multicolumn{1}{p{12cm}}{ Regresa a la trayectoria principal 8}\tabularnewline
 &  & \multicolumn{1}{p{12cm}}{.... Fin de trayectoria}\tabularnewline
\end{tabular}

\textit{Trayectoria Alternativa C}

Condición: No se llenaron todos los campos del formulario.

\begin{tabular}{ccl}
 &  & \tabularnewline
1.  & \includegraphics{sistema}  & \multicolumn{1}{p{12cm}}{ Muestra el mensaje de error MSG8 $"$  Campos Obligatorios$"$  .}\tabularnewline
2.  & \includegraphics{sistema}  & \multicolumn{1}{p{12cm}}{Continua con el paso 8.}\tabularnewline
 &  & \multicolumn{1}{p{12cm}}{.... Fin de Trayectoria}\tabularnewline
\end{tabular}

\textit{Trayectoria Alternativa D}

Condición: Si el actor da clic en el ícono eliminar \includegraphics{elim}  de un articulo

\begin{tabular}{ccl}
 &  & \tabularnewline
1.  & \includegraphics{sistema}  & \multicolumn{1}{p{12cm}}{ Muestra MSG9 "Confirmación de acción"}\tabularnewline
2.  & \includegraphics{actor}  & \multicolumn{1}{p{12cm}}{ Acepta la accion{[}Trayectoria Alternativa F{]}}\tabularnewline
2.  & \includegraphics{sistema}  & \multicolumn{1}{p{12cm}}{ Elimina de la lista el articulo eliminado}\tabularnewline
3.  & \includegraphics{sistema}  & \multicolumn{1}{p{12cm}}{ Muestra MSG1 Mensaje de confirmación $"$ El articulo ha sido eliminado exitosamente.$"$  }\tabularnewline
4.  & \includegraphics{sistema}  & \multicolumn{1}{p{12cm}}{ Regresa al paso 10 de la trayectoria principal$"$  }\tabularnewline
 &  & \multicolumn{1}{p{12cm}}{.... Fin de trayectoria}\tabularnewline
\end{tabular}

\textit{Trayectoria Alternativa E}

Condición: Si el usuario da clic en el botón de cancelar

\begin{tabular}{ccl}
 &  & \tabularnewline
1.  & \includegraphics{sistema}  & \multicolumn{1}{p{12cm}}{ Regresa a la pantalla PV1 $"$  Gestionar Ventas$"$  }\tabularnewline
 &  & \multicolumn{1}{p{12cm}}{.... Fin del caso de uso}\tabularnewline
\end{tabular}

\textit{Trayectoria Alternativa F}

Condición: Si el cancela la accion de eliminar articulo\tabularnewline
\begin{tabular}{ccl}
 &  & \tabularnewline
1.  & \includegraphics{sistema}  & \multicolumn{1}{p{12cm}}{ Permanese en la pantalla PV1.1 $"$  Registar Venta$"$  }\tabularnewline
 &  & \multicolumn{1}{p{12cm}}{.... Fin del caso de uso}\tabularnewline
\end{tabular}
\subsubsection{CUV1.2: Consultar Ventas}

\textbf{\large Resumen}\\
 {\large {} {} {} {} {} {} {} Este Caso de uso tiene como objetivo
permitir al usuario de ventas consultar las venta guardadas sistema.}\\


% = = = = = = = = = = = = = = = = = = = = = = = = =
%	INICIO DE TABLA
% = = = = = = = = = = = = = = = = = = = = = = = = = 
% Se utiliza para que no se salga la tabla de la hoja, porque sino se autodimensiona
%\multicolumn{1}{p{10cm}||}{Contenido}
\begin{table}[!ht]
\begin{centering}
\begin{tabular}{|c||c|l|}
\hline 
\multicolumn{2}{|c|}{Caso de Uso:} & CUV1.2: Consultar Venta\tabularnewline
\hline 
\multicolumn{3}{|>{\columncolor[gray]{0.7}}c}{Resumen de Atributos}\tabularnewline
\hline 
\multicolumn{2}{|c|}{Autor} & \multicolumn{1}{p{10cm}||}{Cabrera Alvarez Estefany Viridiana}\tabularnewline
\hline 
\multicolumn{2}{|c|}{Actor} & \multicolumn{1}{p{10cm}||}{Usuario de Ventas}\tabularnewline
\hline 
\multicolumn{2}{|c|}{Propósito} & \multicolumn{1}{p{10cm}||}{Consultar una venta en el sistema}\tabularnewline
\hline 
\multicolumn{2}{|c|}{Entradas} & \multicolumn{1}{p{10cm}||}{idVenta o RFC del cliente}\tabularnewline
\hline 
\multicolumn{2}{|c|}{Salidas} & \multicolumn{1}{p{10cm}||}{Datos de la venta filtrados}\tabularnewline
\hline 
\multicolumn{2}{|c|}{Pre-condiciones} & \multicolumn{1}{p{10cm}||}{La venta debe existir en el sistema.}\tabularnewline
\hline 
\multicolumn{2}{|c|}{Pos-condiciones} & \multicolumn{1}{p{10cm}||}{Ninguna}\tabularnewline
\hline 
\multicolumn{2}{|c|}{Errores} & \multicolumn{1}{p{10cm}||}{MSG3 Datos no encontrados, MSG4 Error al conectar con la base de datos.}\tabularnewline
\hline 
\multicolumn{2}{|c|}{Tipo} & \multicolumn{1}{p{10cm}||}{Secundario}\tabularnewline
\hline 
\multicolumn{2}{|c|}{Fuente} & \multicolumn{1}{p{10cm}||}{Basado en el funcionamiento}\tabularnewline
\hline 
\end{tabular}
\par\end{centering}

\caption{Caso de Uso 1.2: Consultar ventas}


\label{CUV1.2} 
\end{table}


% = = = = = = = = = = = = = = = = = = = = = = = = =
%	FIN DE TABLA
% = = = = = = = = = = = = = = = = = = = = = = = = =
% = = = = = = = = = = = = = = = = = = = = = = = = =
%	INICIO DE TRAYECTORIA
% = = = = = = = = = = = = = = = = = = = = = = = = = 


\textbf{\large Trayectorias del CU}{\large \par}

\textit{\large Trayectoria Principal}{\large {} }{\large \par}

%Solo hay que cambiar el nombre de la imagen dependiendo de si es actor o sistema


\begin{tabular}{ccl}
 &  & \tabularnewline
1.  & \includegraphics{sistema}  & \multicolumn{1}{p{12cm}}{Verifica si lleno correctamente el textbox. {[}Trayectoria Alternativa A{]}}\tabularnewline
2.  & \includegraphics{sistema}  & \multicolumn{1}{p{12cm}}{Busca los datos correspondientes con la busqueda.{[}Trayectoria Alternativa B{]}{[}Trayectoria Alternativa C{]}}\tabularnewline
3.  & \includegraphics{sistema}  & \multicolumn{1}{p{12cm}}{Muestra la pantalla PV1 $"$  Gestionar Ventas$"$   con los datos filtrados.}\tabularnewline
 &  & \multicolumn{1}{p{12cm}}{.... Fin del caso de uso}\tabularnewline
\end{tabular}

\textit{Trayectoria Alternativa A}

Condición: Si el formato de los datos ingresados en el Textbox es incorrecto.

\begin{tabular}{ccl}
 &  & \tabularnewline
1.  & \includegraphics{sistema}  & \multicolumn{1}{p{12cm}}{ Muestra el MSG7 'Formato incorrecto de los datos.'}\tabularnewline
2.  & \includegraphics{sistema}  & \multicolumn{1}{p{12cm}}{ Regresa al paso PV1 $"$  Gestionar Ventas$"$  }\tabularnewline
 &  & \multicolumn{1}{p{12cm}}{.... Fin del caso de uso}\tabularnewline
\end{tabular}

\textit{Trayectoria Alternativa B}

Condición: Si los datos no fueron encontrados

\begin{tabular}{ccl}
 &  & \tabularnewline
1.  & \includegraphics{sistema}  & \multicolumn{1}{p{12cm}}{Muestra MSG3 $"$  Datos no encontrados.$"$  }\tabularnewline
 &  & \multicolumn{1}{p{12cm}}{.... Fin del caso de uso}\tabularnewline
\end{tabular}

\textit{Trayectoria Alternativa C}

Condición: Si la base de datos no esta disponible.

\begin{tabular}{ccl}
 &  & \tabularnewline
1.  & \includegraphics{sistema}  & \multicolumn{1}{p{12cm}}{Muestra MSG4 $"$ Error al conectar con la base de datos$"$  }\tabularnewline
 &  & \multicolumn{1}{p{12cm}}{.... Fin del caso de uso}\tabularnewline
\end{tabular}

%============================================Cancelar




\subsubsection{CUV1.3: Cancelar Venta}

\textbf{\large Resumen}\\
 {\large {} {} {} {} {} {} {} Este Caso de uso tiene como objetivo
permitir al actor marcar como canceladas alguna venta guardada en  sistema.}\\


% = = = = = = = = = = = = = = = = = = = = = = = = =
%	INICIO DE TABLA
% = = = = = = = = = = = = = = = = = = = = = = = = = 
% Se utiliza para que no se salga la tabla de la hoja, porque sino se autodimensiona
%\multicolumn{1}{p{10cm}||}{Contenido}
\begin{table}[!ht]
\begin{centering}
\begin{tabular}{|c||c|l|}
\hline 
\multicolumn{2}{|c|}{Caso de Uso:} & CUG1.3: Cancelar Venta\tabularnewline
\hline 
\multicolumn{3}{|>{\columncolor[gray]{0.7}}c}{Resumen de Atributos}\tabularnewline
\hline 
\multicolumn{2}{|c|}{Autor} & \multicolumn{1}{p{10cm}||}{Cabrera Alvarez Estefany Viridiana}\tabularnewline
\hline 
\multicolumn{2}{|c|}{Actor} & \multicolumn{1}{p{10cm}||}{Usuario de Ventas}\tabularnewline
\hline 
\multicolumn{2}{|c|}{Propósito} & \multicolumn{1}{p{10cm}||}{Cancelar una venta en el sistema}\tabularnewline
\hline 
\multicolumn{2}{|c|}{Entradas} & \multicolumn{1}{p{10cm}||}{Ninguna}\tabularnewline
\hline 
\multicolumn{2}{|c|}{Salidas} & \multicolumn{1}{p{10cm}||}{Pantalla PV1 $"$ Gestión de Ventas"}\tabularnewline
\hline 
\multicolumn{2}{|c|}{Pre-condiciones} & \multicolumn{1}{p{10cm}||}{El usuario de ventas debe estar autentificado, la venta debe existir en el sistema.}\tabularnewline
\hline 
\multicolumn{2}{|c|}{Pos-condiciones} & \multicolumn{1}{p{10cm}||}{Ninguna}\tabularnewline
\hline 
\multicolumn{2}{|c|}{Errores} & \multicolumn{1}{p{10cm}||}{ MSG4 Error al conectar con la base de datos.}\tabularnewline
\hline 
\multicolumn{2}{|c|}{Tipo} & \multicolumn{1}{p{10cm}||}{Secundario}\tabularnewline
\hline 
\multicolumn{2}{|c|}{Fuente} & \multicolumn{1}{p{10cm}||}{Basado en el funcionamiento del CUV1}\tabularnewline
\hline 
\end{tabular}
\par\end{centering}

\caption{Caso de Uso 1.3: Cancelar Venta}


\label{CUV1.3} 
\end{table}


% = = = = = = = = = = = = = = = = = = = = = = = = =
%	FIN DE TABLA
% = = = = = = = = = = = = = = = = = = = = = = = = =
% = = = = = = = = = = = = = = = = = = = = = = = = =
%	INICIO DE TRAYECTORIA
% = = = = = = = = = = = = = = = = = = = = = = = = = 


\textbf{\large Trayectorias del CU}{\large \par}

\textit{\large Trayectoria Principal}{\large {} }{\large \par}

%Solo hay que cambiar el nombre de la imagen dependiendo de si es actor o sistema


\begin{tabular}{ccl}
 &  & \tabularnewline
1.  & \includegraphics{sistema}  & \multicolumn{1}{p{12cm}}{Muestra MSG9 $"$Confirmar acción"{[}Trayectoria Alternativa A{]}}\tabularnewline
2.  & \includegraphics{sistema}  & \multicolumn{1}{p{12cm}}{Actualiza el estado de la venta a $"$  cancelada$"$   en la base de datos.{[}Trayectoria Alternativa B{]}}\tabularnewline
3.  & \includegraphics{sistema}  & \multicolumn{1}{p{12cm}}{Muestra MSG1 Mensaje de confirmación. $"$  la venta ha sido cancelada$"$  }\tabularnewline
4.  & \includegraphics{sistema}  & \multicolumn{1}{p{12cm}}{Muestra la pantalla PV1 $"$  Gestionar Ventas$"$  }\tabularnewline
 &  & \multicolumn{1}{p{12cm}}{.... Fin del caso de uso}\tabularnewline
\end{tabular}

\textit{Trayectoria Alternativa B}

Condición: Si responde que no al MS9 "Confirmar acción".

\begin{tabular}{ccl}
 &  & \tabularnewline
1.  & \includegraphics{sistema}  & \multicolumn{1}{p{12cm}}{Muestra la pantalla PV1 $"$  Gestionar Ventas$"$  }\tabularnewline
 &  & \multicolumn{1}{p{12cm}}{.... Fin del caso de uso}\tabularnewline
\end{tabular}


\textit{Trayectoria Alternativa B}

Condición: Si la base de datos no esta disponible.

\begin{tabular}{ccl}
 &  & \tabularnewline
1.  & \includegraphics{sistema}  & \multicolumn{1}{p{12cm}}{Muestra MSG4 $"$ Error al conectar con la base de datos$"$  }\tabularnewline
2.  & \includegraphics{sistema}  & \multicolumn{1}{p{12cm}}{Muestra MSG2 Mensaje de error. $"$ Error al intentar cancelar venta 'idVenta'$"$  }\tabularnewline
3.  & \includegraphics{sistema}  & \multicolumn{1}{p{12cm}}{Muestra la pantalla PV1 $"$  Gestionar Ventas$"$  }\tabularnewline
 &  & \multicolumn{1}{p{12cm}}{.... Fin del caso de uso}\tabularnewline
\end{tabular}



%%%%%%%%%%%%%%%%%%%%%%%%%%%%%%%%%%%%%


\subsubsection{CUV1.4: Modificar Venta}

\textbf{\large Resumen}\\
 {\large {} {} {} {} {} {} {} Este Caso de uso tiene como objetivo
permitir a los actores Modificar Venta guardadas en el sistema.}\\


% = = = = = = = = = = = = = = = = = = = = = = = = =
%	INICIO DE TABLA
% = = = = = = = = = = = = = = = = = = = = = = = = = 
% Se utiliza para que no se salga la tabla de la hoja, porque sino se autodimensiona
%\multicolumn{1}{p{10cm}||}{Contenido}
\begin{table}[!ht]
\begin{centering}
\begin{tabular}{|c||c|l|}
\hline 
\multicolumn{2}{|c|}{Caso de Uso:} & CUV1.4: Modificar Venta\tabularnewline
\hline 
\multicolumn{3}{|>{\columncolor[gray]{0.7}}c}{Resumen de Atributos}\tabularnewline
\hline 
\multicolumn{2}{|c|}{Autor} & \multicolumn{1}{p{10cm}||}{Cabrera Alvarez Estefany Viridiana}\tabularnewline
\hline 
\multicolumn{2}{|c|}{Actor} & \multicolumn{1}{p{10cm}||}{Usuario de Ventas}\tabularnewline
\hline 
\multicolumn{2}{|c|}{Propósito} & \multicolumn{1}{p{10cm}||}{Modificar una venta en el sistema}\tabularnewline
\hline 
\multicolumn{2}{|c|}{Entradas} & \multicolumn{1}{p{10cm}||}{Cambios en el formulario}\tabularnewline
\hline 
\multicolumn{2}{|c|}{Salidas} & \multicolumn{1}{p{10cm}||}{Ninguna}\tabularnewline
\hline 
\multicolumn{2}{|c|}{Pre-condiciones} & \multicolumn{1}{p{10cm}||}{ La venta debe existir en el  sistema.}\tabularnewline
\hline 
\multicolumn{2}{|c|}{Pos-condiciones} & \multicolumn{1}{p{10cm}||}{Ninguna}\tabularnewline
\hline 
\multicolumn{2}{|c|}{Errores} & \multicolumn{1}{p{10cm}||}{ MSG4 $"$ Error al conectar con la base de datos.$"$   MSG2 Mensaje de error.  $"$ Error al intentar modificar cliente 'idcliente'$"$  }\tabularnewline
\hline 
\multicolumn{2}{|c|}{Tipo} & \multicolumn{1}{p{10cm}||}{Secundario}\tabularnewline
\hline 
\multicolumn{2}{|c|}{Fuente} & \multicolumn{1}{p{10cm}||}{Basado en el funcionamiento CUV1}\tabularnewline
\hline 
\end{tabular}
\par\end{centering}

\caption{Caso de Uso 1.4: Modificar Venta}


\label{tab:CasosdeUso:nombredecasodeuso} 
\end{table}


% = = = = = = = = = = = = = = = = = = = = = = = = =
%	FIN DE TABLA
% = = = = = = = = = = = = = = = = = = = = = = = = =
% = = = = = = = = = = = = = = = = = = = = = = = = =
%	INICIO DE TRAYECTORIA
% = = = = = = = = = = = = = = = = = = = = = = = = = 


\textbf{\large Trayectorias del CU}{\large \par}

\textit{\large Trayectoria Principal}{\large {} }{\large \par}

%Solo hay que cambiar el nombre de la imagen dependiendo de si es actor o sistema


\begin{tabular}{ccl}
 &  & \tabularnewline
1.  & \includegraphics{sistema}  & \multicolumn{1}{p{12cm}}{Extrae los datos de la venta a modificar.{[}Trayectoria alternativa A{]}}\tabularnewline
2.  & \includegraphics{sistema}  & \multicolumn{1}{p{12cm}}{Muestra la pantalla  PV1.2 $"$  Modificar Venta$"$   con el formulario de modificación con los datos que ya se encuentran registrados}\tabularnewline
3.  & \includegraphics{actor}  & \multicolumn{1}{p{12cm}}{Da clic en el ícono de {[}Modificar{]} del articulo a modificar{[}Trayectoria alternativa B{]}}\tabularnewline
4.  & \includegraphics{sistema}  & \multicolumn{1}{p{12cm}}{Extrae los datos del articulo a modificar.{[}Trayectoria alternativa A{]}}\tabularnewline
5.  & \includegraphics{sistema}  & \multicolumn{1}{p{12cm}}{Coloca los datos en el formulario.}\tabularnewline
6.  & \includegraphics{actor}  & \multicolumn{1}{p{12cm}}{ modifica los datos que desee del formulario}\tabularnewline
7.  & \includegraphics{actor}  & \multicolumn{1}{p{12cm}}{ Da clic en el botón de {[}modificar{]}}\tabularnewline
8.  & \includegraphics{sistema}  & \multicolumn{1}{p{12cm}}{Procesa los datos que se modificaron{[}Trayectoria alternativa A{]}{[}Trayectoria alternativa D{]}}\tabularnewline
9. & \includegraphics{sistema}  & \multicolumn{1}{p{12cm}}{Limpia el formulario}\tabularnewline
10.  & \includegraphics{sistema}  & \multicolumn{1}{p{12cm}}{Muestra los datos actualizados en la tabla}\tabularnewline
11.  & \includegraphics{actor}  & \multicolumn{1}{p{12cm}}{Presiona botón de aceptar{[}Trayectoria alternativa C {]}{[}Trayectoria alternativa E {]}{[}Trayectoria alternativa F {]}}\tabularnewline
12.  & \includegraphics{sistema}  & \multicolumn{1}{p{12cm}}{Muestra MSG1 Mensaje de confirmación $"$   La venta <<idVenta>> ha sido modificado exitosamente.$"$  }\tabularnewline
13.  & \includegraphics{actor}  & \multicolumn{1}{p{12cm}}{Vuelve a la pantalla PV1 $"$  Gestionar Ventas$"$  }\tabularnewline
 &  & \multicolumn{1}{p{12cm}}{.... Fin del caso de uso}\tabularnewline
\end{tabular}
\textit{Trayectoria Alternativa A}

Condición: Si la base de datos no esta disponible.

\begin{tabular}{ccl}
 &  & \tabularnewline
1.  & \includegraphics{sistema}  & \multicolumn{1}{p{12cm}}{Muestra MSG4 $"$ Error al conectar con la base de datos$"$  }\tabularnewline
2.  & \includegraphics{sistema}  & \multicolumn{1}{p{12cm}}{Muestra MSG2 Mensaje de error. $"$ Error al intentar modificar venta 'idVenta'$"$  }\tabularnewline
3.  & \includegraphics{sistema}  & \multicolumn{1}{p{12cm}}{Regresa a la pantalla PV1.2}\tabularnewline
 &  & \multicolumn{1}{p{12cm}}{.... Fin de trayectoria}\tabularnewline
\end{tabular}

\textit{Trayectoria Alternativa B}

Condición: Si el actor preciona el ícono de cancelar

\begin{tabular}{ccl}
 &  & \tabularnewline
1.  & \includegraphics{sistema}  & \multicolumn{1}{p{12cm}}{Marca el articulo como cancelado}\tabularnewline
2.  & \includegraphics{sistema}  & \multicolumn{1}{p{12cm}}{Regresa al paso 10 de la trayectoría principal}\tabularnewline
 &  & \multicolumn{1}{p{12cm}}{.... Fin de trayectoria}\tabularnewline
\end{tabular}
\textit{Trayectoria Alternativa C}

Condición: Si el usuario da clic en el botón de cancelar

\begin{tabular}{ccl}
 &  & \tabularnewline
1.  & \includegraphics{sistema}  & \multicolumn{1}{p{12cm}}{ Muestra PV1 $"$  Gestionar Ventas$"$  }\tabularnewline
 &  & \multicolumn{1}{p{12cm}}{.... Fin de trayectoria}\tabularnewline
\end{tabular}

\textit{Trayectoria Alternativa D}

Condición: Si el formato de los datos ingresados son incorrectos.

\begin{tabular}{ccl}
 &  & \tabularnewline
1.  & \includegraphics{sistema}  & \multicolumn{1}{p{12cm}}{ Muestra el MSG7 $"$  Formato incorrecto de los datos.$"$  }\tabularnewline
2.  & \includegraphics{sistema}  & \multicolumn{1}{p{12cm}}{ Muestra el MSG2 $"$ Error al intentar modificar venta <<numcliente>>$"$  }\tabularnewline
2.  & \includegraphics{sistema}  & \multicolumn{1}{p{12cm}}{ Regresa al paso 2 de la trayectoria principal}\tabularnewline
 &  & \multicolumn{1}{p{12cm}}{.... Fin de trayectoria}\tabularnewline
\end{tabular}

\textit{Trayectoria Alternativa E}

Condición: Si el usuario desea modificar o eliminar otro articulo de la venta

\begin{tabular}{ccl}
 &  & \tabularnewline
1.  & \includegraphics{sistema}  & \multicolumn{1}{p{12cm}}{ Regresa al paso 3 de la trayectoria principal}\tabularnewline
 &  & \multicolumn{1}{p{12cm}}{.... Fin de trayectoria}\tabularnewline
\end{tabular}

\textit{Trayectoria Alternativa F}

Condición: Si no se cumple la BRV1 "Modificacion de Venta"\tabularnewline

\begin{tabular}{ccl}
 &  & \tabularnewline
1.  & \includegraphics{sistema}  & \multicolumn{1}{p{12cm}}{ Muestra el MSG2 $"$ Error al intentar modificar venta <<numcliente>>$"$  }\tabularnewline
2.  & \includegraphics{sistema}  & \multicolumn{1}{p{12cm}}{ Regresa al paso 1 de la trayectoria principal}\tabularnewline
 &  & \multicolumn{1}{p{12cm}}{.... Fin de trayectoria}\tabularnewline
\end{tabular}

%TrayClientes
%=================================GESTION CLIENTES
\subsection{CUV2: Gestionar Clientes}

\textbf{\large Resumen}\\
 {\large {} {} {} {} {} {} {} Este Caso de uso tiene como objetivo
permitir al actor registrar, modificar y buscar clientes.}\\


% = = = = = = = = = = = = = = = = = = = = = = = = =
%	INICIO DE TABLA
% = = = = = = = = = = = = = = = = = = = = = = = = = 
% Se utiliza para que no se salga la tabla de la hoja, porque sino se autodimensiona
%\multicolumn{1}{p{10cm}||}{Contenido}

\begin{table}[!ht]
\begin{centering}
\begin{tabular}{|c||c|l|}
\hline 
\multicolumn{2}{|c|}{Caso de Uso:} & CUV2.0: Gestionar Clientes\tabularnewline
\hline 
\multicolumn{3}{|>{\columncolor[gray]{0.7}}c}{Resumen de Atributos}\tabularnewline
\hline 
\multicolumn{2}{|c|}{Autor} & \multicolumn{1}{p{10cm}||}{Cabrera Alvarez Estefany Viridiana}\tabularnewline
\hline 
\multicolumn{2}{|c|}{Actor} & \multicolumn{1}{p{10cm}||}{Usuario de Ventas, Administrador}\tabularnewline
\hline 
\multicolumn{2}{|c|}{Propósito} & \multicolumn{1}{p{10cm}||}{Acceder a opciones de crear, modificar,consultar o buscar clientes. }\tabularnewline
\hline 
\multicolumn{2}{|c|}{Entradas} & \multicolumn{1}{p{10cm}||}{Ninguna}\tabularnewline
\hline 
\multicolumn{2}{|c|}{Salidas} & \multicolumn{1}{p{10cm}||}{Interfaz Correspondiente PV6.0}\tabularnewline
\hline 
\multicolumn{2}{|c|}{Pre-condiciones} & \multicolumn{1}{p{10cm}||}{El usuario de ventas debe estar autentificado}\tabularnewline
\hline 
\multicolumn{2}{|c|}{Pos-condiciones} & \multicolumn{1}{p{10cm}||}{Ninguna}\tabularnewline
\hline 
\multicolumn{2}{|c|}{Errores} & \multicolumn{1}{p{10cm}||}{MSG4 $"$ Error al conectar con la base de datos$"$  .}\tabularnewline
\hline 
\multicolumn{2}{|c|}{Tipo} & \multicolumn{1}{p{10cm}||}{Primario}\tabularnewline
\hline 
\multicolumn{2}{|c|}{Fuente} & \multicolumn{1}{p{10cm}||}{En base al funcionamiento}\tabularnewline
\hline 
\end{tabular}
\par\end{centering}

\caption{Caso de Uso 2.0: Gestionar Clientes}


\label{CUV2.0} 
\end{table}


% = = = = = = = = = = = = = = = = = = = = = = = = =
%	FIN DE TABLA
% = = = = = = = = = = = = = = = = = = = = = = = = =
% = = = = = = = = = = = = = = = = = = = = = = = = =
%	INICIO DE TRAYECTORIA
% = = = = = = = = = = = = = = = = = = = = = = = = = 


\textbf{\large Trayectorias del CU}{\large \par}

\textit{\large Trayectoria Principal}{\large {} }{\large \par}

%Solo hay que cambiar el nombre de la imagen dependiendo de si es actor o sistema


\begin{tabular}{ccl}
 &  & \tabularnewline
1.  & \includegraphics{actor}  & \multicolumn{1}{p{12cm}}{ Presiona el botón {[}Clientes{]} en la pantalla PV0}\tabularnewline
2.  & \includegraphics{sistema}  & \multicolumn{1}{p{12cm}}{Carga los datos de los clientes registrados en la Base de datos. {[}Trayectoria
Alternativa A{]}}\tabularnewline
3.  & \includegraphics{sistema}  & \multicolumn{1}{p{12cm}}{Despliega la pantalla PV2 "Gestionar Clientes"}\tabularnewline
4.  & \includegraphics{actor}  & \multicolumn{1}{p{12cm}}{Da clic en el botón {[}Registrar cliente{]}{[}Trayectoria Alternativa B{]}{[}Trayectoria Alternativa C{]}{[}Trayectoria Alternativa D{]}}\tabularnewline
5.  & \includegraphics{sistema}  & \multicolumn{1}{p{12cm}}{Extiende a CUV2.1 $"$  Registrar Cliente$"$  }\tabularnewline
 &  & \multicolumn{1}{p{12cm}}{.... Fin del caso de uso}\tabularnewline
\end{tabular}

\textit{Trayectoria Alternativa A}

Condición: No se pudieron cargar los datos de los clientes registrados

\begin{tabular}{ccl}
 &  & \tabularnewline
1.  & \includegraphics{sistema}  & \multicolumn{1}{p{12cm}}{ Muestra el mensaje de error MSG4 $"$  No se pudo establecer conexión con la base de datos$"$  }\tabularnewline
2.  & \includegraphics{sistema}  &\multicolumn{1}{p{12cm}}{Regresa a la pantalla PV0$"$  Inicio de Ventas$"$  }\tabularnewline
 &  & \multicolumn{1}{p{12cm}}{.... Fin del caso de uso}\tabularnewline
\end{tabular}\newpage{}

\textit{Trayectoria Alternativa B}

Condición: El actor escribe el dato a buscar en el textbox y da clic en buscar.

\begin{tabular}{ccl}
 &  & \tabularnewline
1.  & \includegraphics{sistema} & \multicolumn{1}{p{12cm}}{ Extiende al CUV2.2 $"$  Consultar Clientes$"$  }\tabularnewline
 	
 &  & \multicolumn{1}{p{12cm}}{.... Fin del caso de uso}\tabularnewline
\end{tabular}

\textit{Trayectoria Alternativa C}

Condición: El actor da clic en el ícono de {[}cancelar{]} en una cliente.

\begin{tabular}{ccl}
 &  & \tabularnewline
1. & \includegraphics{sistema} & \multicolumn{1}{p{12cm}}{ Extiende al CUV2.3 $"$  Cancelar Cliente$"$  }\tabularnewline
 	
 &  & \multicolumn{1}{p{12cm}}{.... Fin del caso de uso}\tabularnewline
\end{tabular}

\textit{Trayectoria Alternativa D}

Condición: El actor da clic en el ícono de {[}modificar{]} en un cliente.

\begin{tabular}{ccl}
 &  & \tabularnewline
1.  & \includegraphics{sistema} & \multicolumn{1}{p{12cm}}{ Extiende al CUV2.4 $"$  Modificar Venta$"$  }\tabularnewline
 	
 &  & \multicolumn{1}{p{12cm}}{.... Fin del caso de uso}\tabularnewline
\end{tabular}


\subsubsection{CUV2.1:Registrar Cliente}

\textbf{\large Resumen}\\
 {\large {} {} {} {} {} {} {} Este Caso de uso tiene como objetivo
permitir a los actores agregar un cliente al sistema.}\\


% = = = = = = = = = = = = = = = = = = = = = = = = =
%	INICIO DE TABLA
% = = = = = = = = = = = = = = = = = = = = = = = = = 
% Se utiliza para que no se salga la tabla de la hoja, porque sino se autodimensiona
%\multicolumn{1}{p{10cm}||}{Contenido}
\begin{table}[!ht]
\begin{centering}
\begin{tabular}{|c||c|l|}
\hline 
\multicolumn{2}{|c|}{Caso de Uso:} & CUV2.1: Registrar Cliente\tabularnewline
\hline 
\multicolumn{3}{|>{\columncolor[gray]{0.7}}c}{Resumen de Atributos}\tabularnewline
\hline 
\multicolumn{2}{|c|}{Autor} & \multicolumn{1}{p{10cm}||}{Cabrera Alvarez Estefany Viridiana}\tabularnewline
\hline 
\multicolumn{2}{|c|}{Actor} & \multicolumn{1}{p{10cm}||}{Usuario de Ventas}\tabularnewline
\hline 
\multicolumn{2}{|c|}{Propósito} & \multicolumn{1}{p{10cm}||}{Registrar un cliente en el sistema}\tabularnewline
\hline 
\multicolumn{2}{|c|}{Entradas} & \multicolumn{1}{p{10cm}||}{Datos de formulaario de Registro}\tabularnewline
\hline 
\multicolumn{2}{|c|}{Salidas} & \multicolumn{1}{p{10cm}||}{Ninguna}\tabularnewline
\hline 
\multicolumn{2}{|c|}{Pre-condiciones} & \multicolumn{1}{p{10cm}||}{El cliente no debe existir}\tabularnewline
\hline 
\multicolumn{2}{|c|}{Pos-condiciones} & \multicolumn{1}{p{10cm}||}{Ninguna}\tabularnewline
\hline 
\multicolumn{2}{|c|}{Errores} & \multicolumn{1}{p{10cm}||}{MSG2 Mensaje de error, MSG4 Error al conectar con la base de datos, MSG7 'Formato incorrecto de los datos.'.}\tabularnewline
\hline 
\multicolumn{2}{|c|}{Tipo} & \multicolumn{1}{p{10cm}||}{Secundario}\tabularnewline
\hline 
\multicolumn{2}{|c|}{Fuente} & \multicolumn{1}{p{10cm}||}{Basado en el funcionamiento del CU 2.0}\tabularnewline
\hline 
\end{tabular}
\par\end{centering}

\caption{Caso de Uso 2.1: Registrar Cliente}


\label{CUV2.1} 
\end{table}


% = = = = = = = = = = = = = = = = = = = = = = = = =
%	FIN DE TABLA
% = = = = = = = = = = = = = = = = = = = = = = = = =
% = = = = = = = = = = = = = = = = = = = = = = = = =
%	INICIO DE TRAYECTORIA
% = = = = = = = = = = = = = = = = = = = = = = = = = 


\textbf{\large Trayectorias del CU}{\large \par}

\textit{\large Trayectoria Principal}{\large {} }{\large \par}

%Solo hay que cambiar el nombre de la imagen dependiendo de si es actor o sistema


\begin{tabular}{ccl}
 &  & \tabularnewline
1.  & \includegraphics{sistema}  & \multicolumn{1}{p{12cm}}{Despliega la pantalla PV2.1$"$  Registrar Cliente$"$  con el formulario a llenar.}\tabularnewline
2.  & \includegraphics{actor}  & \multicolumn{1}{p{12cm}}{Llena los datos del formulario.}\tabularnewline
3.  & \includegraphics{actor}  & \multicolumn{1}{p{12cm}}{Da clic en el botón de aceptar. {[}Trayectoria Alternativa A{]}}\tabularnewline
4.  & \includegraphics{sistema}  & \multicolumn{1}{p{12cm}}{Verifica si los campos se llenaron correctamente. {[}Trayectoria Alternativa B{]}{[}Trayectoria Alternativa C{]} {[}Trayectoria Alternativa D{]}}\tabularnewline
5.  & \includegraphics{sistema}  & \multicolumn{1}{p{12cm}}{Muestra MSG1 Mensaje de confirmación 'El cliente ha sido Agregado exitosamente.}\tabularnewline
 &  & \multicolumn{1}{p{12cm}}{.... Fin del caso de uso}\tabularnewline
\end{tabular}\newpage{}

\textit{Trayectoria Alternativa A}

Condición: Si el usuario da clic en el botón de cancelar

\begin{tabular}{ccl}
 &  & \tabularnewline
1.  & \includegraphics{sistema}  & \multicolumn{1}{p{12cm}}{ Regresa a la pantalla PV5.0}\tabularnewline
 &  & \multicolumn{1}{p{12cm}}{.... Fin del caso de uso}\tabularnewline
\end{tabular}

\textit{Trayectoria Alternativa B}

Condición: Si la base de datos no esta disponible.

\begin{tabular}{ccl}
 &  & \tabularnewline
1.  & \includegraphics{sistema}  & \multicolumn{1}{p{12cm}}{Muestra MSG4 Error al conectar con la base de datos}\tabularnewline
1.  & \includegraphics{sistema}  & \multicolumn{1}{p{12cm}}{Regresa a la pantalla PV45.1}\tabularnewline
 &  & \multicolumn{1}{p{12cm}}{.... Fin de trayectoria}\tabularnewline
\end{tabular}

\textit{Trayectoria Alternativa C}

Condición: Si el formato de los datos es incorrecto.

\begin{tabular}{ccl}
 &  & \tabularnewline
1.  & \includegraphics{sistema}  & \multicolumn{1}{p{12cm}}{ Muestra el MSG7 'Formato incorrecto de los datos.'}\tabularnewline
2.  & \includegraphics{sistema}  & \multicolumn{1}{p{12cm}}{ Continua en el paso 3 de la trayectoria principal}\tabularnewline
 &  & \multicolumn{1}{p{12cm}}{.... Fin de trayectoria}\tabularnewline
\end{tabular}

\textit{Trayectoria Alternativa D}

Condición: No se llenaron todos los campos del formulario.

\begin{tabular}{ccl}
 &  & \tabularnewline
1.  & \includegraphics{sistema}  & \multicolumn{1}{p{12cm}}{ Muestra el mensaje de error MSG8.}\tabularnewline
2.  & \includegraphics{sistema}  & \multicolumn{1}{p{12cm}}{Continua con el paso 3 de la trayectoria principal.}\tabularnewline
 &  & \multicolumn{1}{p{12cm}}{.... Fin de Trayectoria}\tabularnewline
\end{tabular}

\textit{Trayectoria Alternativa D}

Condición: Si el RFC no cumple con el formato mencionado en BRV2.

\begin{tabular}{ccl}
 &  & \tabularnewline
1.  & \includegraphics{sistema}  & \multicolumn{1}{p{12cm}}{ Muestra el mensaje de error MSG7 'Formato incorrecto de los datos.'}\tabularnewline
2.  & \includegraphics{sistema}  & \multicolumn{1}{p{12cm}}{Continua con el paso 3 de la trayectoria principal.}\tabularnewline
 &  & \multicolumn{1}{p{12cm}}{.... Fin de Trayectoria}\tabularnewline
\end{tabular}
%===================================5.2



\subsubsection{CUV2.2: Buscar Clientes}

\textbf{\large Resumen}\\
 {\large {} {} {} {} {} {} {} Este Caso de uso tiene como objetivo
permitir a los actores consultar los clientes guardadas sistema.}\\


% = = = = = = = = = = = = = = = = = = = = = = = = =
%	INICIO DE TABLA
% = = = = = = = = = = = = = = = = = = = = = = = = = 
% Se utiliza para que no se salga la tabla de la hoja, porque sino se autodimensiona
%\multicolumn{1}{p{10cm}||}{Contenido}
\begin{table}[!ht]
\begin{centering}
\begin{tabular}{|c||c|l|}
\hline 
\multicolumn{2}{|c|}{Caso de Uso:} & CUV2.2 Buscar Clientes\tabularnewline
\hline 
\multicolumn{3}{|>{\columncolor[gray]{0.7}}c}{Resumen de Atributos}\tabularnewline
\hline 
\multicolumn{2}{|c|}{Autor} & \multicolumn{1}{p{10cm}||}{Cabrera Alvarez Estefany Viridiana}\tabularnewline
\hline 
\multicolumn{2}{|c|}{Actor} & \multicolumn{1}{p{10cm}||}{Usuario de Ventas}\tabularnewline
\hline 
\multicolumn{2}{|c|}{Propósito} & \multicolumn{1}{p{10cm}||}{Buscar cliente en el sistema}\tabularnewline
\hline 
\multicolumn{2}{|c|}{Entradas} & \multicolumn{1}{p{10cm}||}{idcliente}\tabularnewline
\hline 
\multicolumn{2}{|c|}{Salidas} & \multicolumn{1}{p{10cm}||}{Datos del cliente encontrado}\tabularnewline
\hline 
\multicolumn{2}{|c|}{Pre-condiciones} & \multicolumn{1}{p{10cm}||}{Ninguna}\tabularnewline
\hline 
\multicolumn{2}{|c|}{Pos-condiciones} & \multicolumn{1}{p{10cm}||}{La base de datos debe de estar disponible}\tabularnewline
\hline 
\multicolumn{2}{|c|}{Errores} & \multicolumn{1}{p{10cm}||}{MSG3 $"$  Datos no encontrados$"$  , MSG4 $"$  Error al conectar con la base de datos.$"$  }\tabularnewline
\hline 
\multicolumn{2}{|c|}{Tipo} & \multicolumn{1}{p{10cm}||}{Secundario}\tabularnewline
\hline 
\multicolumn{2}{|c|}{Fuente} & \multicolumn{1}{p{10cm}||}{Basado en el funcionamiento del CU2.0}\tabularnewline
\hline 
\end{tabular}
\par\end{centering}

\caption{Caso de Uso 2.2: Buscar clientes}


\label{CUV2.2} 
\end{table}


% = = = = = = = = = = = = = = = = = = = = = = = = =
%	FIN DE TABLA
% = = = = = = = = = = = = = = = = = = = = = = = = =
% = = = = = = = = = = = = = = = = = = = = = = = = =
%	INICIO DE TRAYECTORIA
% = = = = = = = = = = = = = = = = = = = = = = = = = 


\textbf{\large Trayectorias del CU}{\large \par}

\textit{\large Trayectoria Principal}{\large {} }{\large \par}

%Solo hay que cambiar el nombre de la imagen dependiendo de si es actor o sistema


\begin{tabular}{ccl}
 &  & \tabularnewline
1.  & \includegraphics{sistema}  & \multicolumn{1}{p{12cm}}{Verifica si se llenaron correctamente el textbox. {[}Trayectoria Alternativa A{]}}\tabularnewline
2.  & \includegraphics{sistema}  & \multicolumn{1}{p{12cm}}{Busca los datos correspondientes con la busqueda.{[}Trayectoria Alternativa B{]}{[}Trayectoria Alternativa C{]}}\tabularnewline
3.  & \includegraphics{sistema}  & \multicolumn{1}{p{12cm}}{Muestra la pantalla PV2 con los datos filtrados.}\tabularnewline
 &  & \multicolumn{1}{p{12cm}}{.... Fin del caso de uso}\tabularnewline
\end{tabular}

\textit{Trayectoria Alternativa A}

Condición: Si el formato de los datos ingresados en el Textbox es
incorrecto.

\begin{tabular}{ccl}
 &  & \tabularnewline
1.  & \includegraphics{sistema}  & \multicolumn{1}{p{12cm}}{ Muestra el MSG7 $"$  Formato incorrecto de los datos.$"$  }\tabularnewline
2.  & \includegraphics{sistema}  & \multicolumn{1}{p{12cm}}{Permanese en la pantalla PV2 $"$  Gestionar clientes$"$   con los datos anteriores.}\tabularnewline
 &  & \multicolumn{1}{p{12cm}}{.... Fin de trayectoria}\tabularnewline
\end{tabular}

\textit{Trayectoria Alternativa B}

Condición: Si los datos no fueron encontrados

\begin{tabular}{ccl}
 &  & \tabularnewline
1.  & \includegraphics{sistema}  & \multicolumn{1}{p{12cm}}{Muestra MSG3 $"$  Datos no encontrados$"$  .}\tabularnewline
2.  & \includegraphics{sistema}  & \multicolumn{1}{p{12cm}}{Permanese en la pantalla PV2 $"$  Gestionar clientes$"$   con los datos anteriores.}\tabularnewline
 &  & \multicolumn{1}{p{12cm}}{.... Fin del caso de uso}\tabularnewline
\end{tabular}

\textit{Trayectoria Alternativa C}

Condición: Si la base de datos no esta disponible.

\begin{tabular}{ccl}
 &  & \tabularnewline
1.  & \includegraphics{sistema}  & \multicolumn{1}{p{12cm}}{Muestra MSG4 $"$ Error al conectar con la base de datos$"$  }\tabularnewline
2.  & \includegraphics{sistema}  & \multicolumn{1}{p{12cm}}{Permanece a la pantalla PV2 $"$  Gestionar clientes$"$   con los datos anteriores}\tabularnewline
 &  & \multicolumn{1}{p{12cm}}{.... Fin de trayectoria}\tabularnewline
\end{tabular}

%====================================5.3


%============================================ELIMINAR



\subsubsection{CUV2.3: Cancelar Cliente}

\textbf{\large Resumen}\\
 {\large {} {} {} {} {} {} {} Este Caso de uso tiene como objetivo
permitir a los actores marcar como cancelado clientes
guardados en el sistema.}\\


% = = = = = = = = = = = = = = = = = = = = = = = = =
%	INICIO DE TABLA
% = = = = = = = = = = = = = = = = = = = = = = = = = 
% Se utiliza para que no se salga la tabla de la hoja, porque sino se autodimensiona
%\multicolumn{1}{p{10cm}||}{Contenido}
\begin{table}[!ht]
\begin{centering}
\begin{tabular}{|c||c|l|}
\hline 
\multicolumn{2}{|c|}{Caso de Uso:} & CUV2.3: Cancelar Cliente\tabularnewline
\hline 
\multicolumn{3}{|>{\columncolor[gray]{0.7}}c}{Resumen de Atributos}\tabularnewline
\hline 
\multicolumn{2}{|c|}{Autor} & \multicolumn{1}{p{10cm}||}{Cabrera Alvarez Estefany Viridiana}\tabularnewline
\hline 
\multicolumn{2}{|c|}{Actor} & \multicolumn{1}{p{10cm}||}{Usuario de Ventas}\tabularnewline
\hline 
\multicolumn{2}{|c|}{Propósito} & \multicolumn{1}{p{10cm}||}{Marcar como cancelado un cliente en el sistema}\tabularnewline
\hline 
\multicolumn{2}{|c|}{Entradas} & \multicolumn{1}{p{10cm}||}{Ninguna}\tabularnewline
\hline 
\multicolumn{2}{|c|}{Salidas} & \multicolumn{1}{p{10cm}||}{Pantalla PV2 $"$Gestionar Clientes" con los datos filtrados}\tabularnewline
\hline 
\multicolumn{2}{|c|}{Pre-condiciones} & \multicolumn{1}{p{10cm}||}{El usuario de ventas debe estar autentificado, El cliente debe existir en el sistema.}\tabularnewline
\hline 
\multicolumn{2}{|c|}{Pos-condiciones} & \multicolumn{1}{p{10cm}||}{Ninguna}\tabularnewline
\hline 
\multicolumn{2}{|c|}{Errores} & \multicolumn{1}{p{10cm}||}{MSG4 $"$  Error al conectar con la base de datos$"$  , MSG2 Mensaje de error 'Error al intentar eliminar cliente'}\tabularnewline
\hline 
\multicolumn{2}{|c|}{Tipo} & \multicolumn{1}{p{10cm}||}{Secundario}\tabularnewline
\hline 
\multicolumn{2}{|c|}{Fuente} & \multicolumn{1}{p{10cm}||}{Basado en el funcionamiento del CUV 2.1}\tabularnewline
\hline 
\end{tabular}
\par \end{centering}

\caption{Caso de Uso 2.3: Cancelar Cliente}


\label{CUV2.3} 
\end{table}


% = = = = = = = = = = = = = = = = = = = = = = = = =
%	FIN DE TABLA
% = = = = = = = = = = = = = = = = = = = = = = = = =
% = = = = = = = = = = = = = = = = = = = = = = = = =
%	INICIO DE TRAYECTORIA
% = = = = = = = = = = = = = = = = = = = = = = = = = 


\textbf{\large Trayectorias del CU}{\large \par}

\textit{\large Trayectoria Principal}{\large {} }{\large \par}

%Solo hay que cambiar el nombre de la imagen dependiendo de si es actor o sistema


\begin{tabular}{ccl}
 &  & \tabularnewline
1.  & \includegraphics{actor}  & \multicolumn{1}{p{12cm}}{ Selecciona el cliente eliminar.}\tabularnewline
2.  & \includegraphics{actor}  & \multicolumn{1}{p{12cm}}{Presionar ícono de {[}Eliminar{]}}\tabularnewline
3.  & \includegraphics{sistema}  & \multicolumn{1}{p{12cm}}{Marca como eliminado al cliente en la base de datos {[}Trayectoria Alternativa A{]}}\tabularnewline
4.  & \includegraphics{sistema}  & \multicolumn{1}{p{12cm}}{Muestra MSG1 Mensaje de confirmación. El cliente ha sido eliminado}\tabularnewline
5.  & \includegraphics{sistema}  & \multicolumn{1}{p{12cm}}{Muestra pantalla PV2 $"$  Gestionar clientes$"$  }\tabularnewline
 &  & \multicolumn{1}{p{12cm}}{.... Fin del caso de uso}\tabularnewline
\end{tabular}

\textit{Trayectoria Alternativa A}

Condición: Si la base de datos no esta disponible.

\begin{tabular}{ccl}
 &  & \tabularnewline
1.  & \includegraphics{sistema}  & \multicolumn{1}{p{12cm}}{Muestra MSG4 $"$ Error al conectar con la base de datos$"$  }\tabularnewline
2.  & \includegraphics{sistema}  & \multicolumn{1}{p{12cm}}{Muestra MSG2 $"$  Mensaje de error. 'Error al intentar modificar cliente 'rfccliente'$"$  }\tabularnewline
3.  & \includegraphics{sistema}  & \multicolumn{1}{p{12cm}}{Regresa a la pantalla PV2 $"$  Gestionar clientes$"$  }\tabularnewline
 &  & \multicolumn{1}{p{12cm}}{.... Fin de trayectoria}\tabularnewline
\end{tabular}

%====================================5.4


%============================================ELIMINAR



\subsubsection{CUV2.4: Modificar Clientes}

\textbf{\large Resumen}\\
 {\large {} {} {} {} {} {} {} Este Caso de uso tiene como objetivo
permitir a los actores modificar los datos de clientes guardados en el sistema.}\\


% = = = = = = = = = = = = = = = = = = = = = = = = =
%	INICIO DE TABLA
% = = = = = = = = = = = = = = = = = = = = = = = = = 
% Se utiliza para que no se salga la tabla de la hoja, porque sino se autodimensiona
%\multicolumn{1}{p{10cm}||}{Contenido}
\begin{table}[!ht]
\begin{centering}
\begin{tabular}{|c||c|l|}
\hline 
\multicolumn{2}{|c|}{Caso de Uso:} & CUV2.4: ModificarCliente\tabularnewline
\hline 
\multicolumn{3}{|>{\columncolor[gray]{0.7}}c}{Resumen de Atributos}\tabularnewline
\hline 
\multicolumn{2}{|c|}{Autor} & \multicolumn{1}{p{10cm}||}{Cabrera Alvarez Estefany Viridiana}\tabularnewline
\hline 
\multicolumn{2}{|c|}{Actor} & \multicolumn{1}{p{10cm}||}{Usuario de Ventas}\tabularnewline
\hline 
\multicolumn{2}{|c|}{Propósito} & \multicolumn{1}{p{10cm}||}{Modificar un cliente en el sistema}\tabularnewline
\hline 
\multicolumn{2}{|c|}{Entradas} & \multicolumn{1}{p{10cm}||}{Cambios en el formulario}\tabularnewline
\hline 
\multicolumn{2}{|c|}{Salidas} & \multicolumn{1}{p{10cm}||}{Ninguna}\tabularnewline
\hline 
\multicolumn{2}{|c|}{Pre-condiciones} & \multicolumn{1}{p{10cm}||}{El cliente debe existir
en el sistema.}\tabularnewline
\hline 
\multicolumn{2}{|c|}{Pos-condiciones} & \multicolumn{1}{p{10cm}||}{Ninguna}\tabularnewline
\hline 
\multicolumn{2}{|c|}{Errores} & \multicolumn{1}{p{10cm}||}{ MSG4 $"$ Error al conectar con la base de datos.$"$   MSG2 Mensaje de error $"$ Error al intentar modificar cliente 'idcliente'$"$  }\tabularnewline
\hline 
\multicolumn{2}{|c|}{Tipo} & \multicolumn{1}{p{10cm}||}{Secundario}\tabularnewline
\hline 
\multicolumn{2}{|c|}{Fuente} & \multicolumn{1}{p{10cm}||}{Basado en el funcionamiento del CU2}\tabularnewline
\hline 
\end{tabular}
\par\end{centering}

\caption{Caso de Uso 2.4: Modificar Cliente}


\label{CUV2.4} 
\end{table}


% = = = = = = = = = = = = = = = = = = = = = = = = =
%	FIN DE TABLA
% = = = = = = = = = = = = = = = = = = = = = = = = =
% = = = = = = = = = = = = = = = = = = = = = = = = =
%	INICIO DE TRAYECTORIA
% = = = = = = = = = = = = = = = = = = = = = = = = = 


\textbf{\large Trayectorias del CU}{\large \par}

\textit{\large Trayectoria Principal}{\large {} }{\large \par}

%Solo hay que cambiar el nombre de la imagen dependiendo de si es actor o sistema


\begin{tabular}{ccl}
 &  & \tabularnewline
1.  & \includegraphics{sistema}  & \multicolumn{1}{p{12cm}}{Extrae los datos del cliente a modifiar.{[}Trayectoria alternativa A{]}}\tabularnewline
2.  & \includegraphics{sistema}  & \multicolumn{1}{p{12cm}}{Muestra formulario de modificación con los datos que ya se encuentran registrados PV2.2 $"$  Modificar Cliente$"$  }\tabularnewline
3.  & \includegraphics{actor}  & \multicolumn{1}{p{12cm}}{Actor modifica los datos correspondientes}\tabularnewline
4.  & \includegraphics{actor}  & \multicolumn{1}{p{12cm}}{Preciona boton de Modificar{[}Trayectoria alternativa A{]}{[}Trayectoria alternativa B{]}}\tabularnewline
5.  & \includegraphics{actor}  & \multicolumn{1}{p{12cm}}{Procesa los datos que se modificaronr{[}Trayectoria
alternativa C{]}}\tabularnewline
6.  & \includegraphics{sistema}  & \multicolumn{1}{p{12cm}}{Muestra MSG1 Mensaje de confirmación $"$   El cliente con 'RFC' ha sido modificado exitosamente.$"$  }\tabularnewline
7.  & \includegraphics{actor}  & \multicolumn{1}{p{12cm}}{Vuelve a la pantalla PV2 $"$  Gestionar Clientes$"$  }\tabularnewline
 &  & \multicolumn{1}{p{12cm}}{.... Fin del caso de uso}\tabularnewline
\end{tabular}

\textit{Trayectoria Alternativa A}

Condición: Si la base de datos no esta disponible.

\begin{tabular}{ccl}
 &  & \tabularnewline
1.  & \includegraphics{sistema}  & \multicolumn{1}{p{12cm}}{Muestra MSG4 $"$  Error al conectar con la base de datos$"$  }\tabularnewline
2.  & \includegraphics{sistema}  & \multicolumn{1}{p{12cm}}{Muestra MSG2 Mensaje de error. $"$  Error al intentar modificar cliente 'RFC'$"$  }\tabularnewline
3.  & \includegraphics{sistema}  & \multicolumn{1}{p{12cm}}{Regresa a la pantalla PV2 $"$  Gestionar Clientes$"$  }\tabularnewline
 &  & \multicolumn{1}{p{12cm}}{.... Fin del caso de uso}\tabularnewline
\end{tabular}

\textit{Trayectoria Alternativa B}

Condición: Si el usuario da clic en el botón de cancelar

\begin{tabular}{ccl}
 &  & \tabularnewline
1.  & \includegraphics{sistema}  & \multicolumn{1}{p{12cm}}{ Regresa a la pantalla PV2 $"$  Gestionar Clientes$"$  }\tabularnewline
 &  & \multicolumn{1}{p{12cm}}{.... Fin del caso de uso}\tabularnewline
\end{tabular}

\textit{Trayectoria Alternativa C}

Condición: Si el formato de los datos ingresados es incorrecto.

\begin{tabular}{ccl}
 &  & \tabularnewline
1.  & \includegraphics{sistema}  & \multicolumn{1}{p{12cm}}{ Muestra el MSG7 $"$  Formato incorrecto de los datos.$"$  }\tabularnewline
2.  & \includegraphics{sistema}  & \multicolumn{1}{p{12cm}}{ Muestra el MSG2 $"$  Error al intentar modificar cliente 'RFC'$"$  }\tabularnewline
3.  & \includegraphics{sistema}  & \multicolumn{1}{p{12cm}}{ Regresa al paso 1 de la trayectoria principal}\tabularnewline
 &  & \multicolumn{1}{p{12cm}}{.... Fin de trayectoria}\tabularnewline
\end{tabular}


%===============ALAN
\subsection{CUV3: Reportes }

\textbf{\large Resumen}{\large }\\
{\large{} Este Caso de uso tiene como objetivo permitir al usuario consultar los reportes.}\\


% = = = = = = = = = = = = = = = = = = = = = = = = =
%	INICIO DE TABLA
% = = = = = = = = = = = = = = = = = = = = = = = = = 
% Se utiliza para que no se salga la tabla de la hoja, porque sino se autodimensiona
%\multicolumn{1}{p{10cm}||}{Contenido}
%
\begin{table}[!ht]
\begin{centering}
\begin{tabular}{|c||c|l|}
\hline 
\multicolumn{2}{|c|}{Caso de Uso:} & CUV3: Reportes\tabularnewline
\hline 
\multicolumn{3}{|>{\columncolor[gray]{0.7}}c}{Resumen de Atributos}\tabularnewline
\hline 
\multicolumn{2}{|c|}{Autor} & \multicolumn{1}{p{10cm}||}{Diaz Rodriguez Kevin Alan}\tabularnewline
\hline 
\multicolumn{2}{|c|}{Actor} & \multicolumn{1}{p{10cm}||}{Usuario de Ventas, Administrador}\tabularnewline
\hline 
\multicolumn{2}{|c|}{Propósito} & \multicolumn{1}{p{10cm}||}{Acceder a opciones de consultar reportes.
}\tabularnewline
\hline 
\multicolumn{2}{|c|}{Entradas} & \multicolumn{1}{p{10cm}||}{Llenar Formulario}\tabularnewline
\hline 
\multicolumn{2}{|c|}{Salidas} & \multicolumn{1}{p{10cm}||}{Interfaz Correspondiente a Consultar reportes}\tabularnewline
\hline 
\multicolumn{2}{|c|}{Pre-condiciones} & \multicolumn{1}{p{10cm}||}{Sesion Iniciada}\tabularnewline
\hline 
\multicolumn{2}{|c|}{Pos-condiciones} & \multicolumn{1}{p{10cm}||}{La base de datos debe estar disponible}\tabularnewline
\hline 
\multicolumn{2}{|c|}{Errores} & \multicolumn{1}{p{10cm}||}{MSG3 Datos no encontrados, MSG4 Error al conectar con la base de datos, MSG8 Campos del formulario vacios. MSG13 La fecha indicada es posterior a la fecha del dia de hoy}\tabularnewline
\hline 
\multicolumn{2}{|c|}{Tipo} & \multicolumn{1}{p{10cm}||}{Primario}\tabularnewline
\hline 
\multicolumn{2}{|c|}{Fuente} & \multicolumn{1}{p{10cm}||}{Basado en el funcionamiento.}\tabularnewline
\hline 
\end{tabular}
\par\end{centering}

\caption{Caso de Uso 3: Reportes}
\label{tab:CasosdeUso:nombredecasodeuso} 
\end{table}


% = = = = = = = = = = = = = = = = = = = = = = = = =
%	FIN DE TABLA
% = = = = = = = = = = = = = = = = = = = = = = = = =
% = = = = = = = = = = = = = = = = = = = = = = = = =
%	INICIO DE TRAYECTORIA
% = = = = = = = = = = = = = = = = = = = = = = = = = 


\textbf{\large Trayectorias del CU}{\large \par}

\textit{\large Trayectoria Principal}{\large {} }{\large \par}

%Solo hay que cambiar el nombre de la imagen dependiendo de si es actor o sistema

\begin{tabular}{ccl}
 &  & \tabularnewline
1.  & \includegraphics{actor}  & \multicolumn{1}{p{12cm}}{ Presiona el botón {[}Reportes{]} en la pantalla PV0.}\tabularnewline
2.  & \includegraphics{sistema}  & \multicolumn{1}{p{12cm}}{Muestra la pantalla PV3 con el formulario.}\tabularnewline
3.  & \includegraphics{actor}  & \multicolumn{1}{p{12cm}}{Llena el formulario(selecciona la opción PEDIDOS). {[}Trayectoria Alternativa A{] { [}Trayectoria Alternativa B{]}}}\tabularnewline
4.  & \includegraphics{sistema}  & \multicolumn{1}{p{12cm}}{Muestra los estados del pedido.}\tabularnewline
5.  & \includegraphics{actor}  & \multicolumn{1}{p{12cm}}{Selecciona un estado.}\tabularnewline
6.  & \includegraphics{actor}  & \multicolumn{1}{p{12cm}}{Presiona el botón {[}Aceptar{]}.{[}Trayectoria Alternativa C{]}}\tabularnewline
7.  & \includegraphics{sistema}  & \multicolumn{1}{p{12cm}}{Verifica si los campos se llenaron correctamente.{[}Trayectoria Alternativa D{]}}\tabularnewline
8.  & \includegraphics{sistema}  & \multicolumn{1}{p{12cm}}{Manda solicitud a la base de datos.{[}Trayectoria Alternativa E{]}}\tabularnewline
9.  & \includegraphics{sistema}  & \multicolumn{1}{p{12cm}}{Filtra peticiones en la Base de Datos.{[}Trayectoria Alternativa F{]}}\tabularnewline
10.  & \includegraphics{sistema}  & \multicolumn{1}{p{12cm}}{Muestra los datos correspondientes a la solicitud realizada.(Cumpliendo con la regla de negocio BR5 Formato de los reportes)}\tabularnewline
 &  & \multicolumn{1}{p{12cm}}{.... Fin del caso de uso}\tabularnewline
\end{tabular}

\textit{Trayectoria Alternativa A}

Condición: Selecciona la opción VENTAS.

\begin{tabular}{ccl}
 &  & \tabularnewline
1.   & \includegraphics{sistema}  & \multicolumn{1}{p{12cm}}{Muestra los estados de ventas.}\tabularnewline
2.  & \includegraphics{sistema}  & \multicolumn{1}{p{12cm}}{Continua con el paso 5.}\tabularnewline
 &  & \multicolumn{1}{p{12cm}}{.... Fin de Trayectoria}\tabularnewline
\end{tabular}

\textit{Trayectoria Alternativa B}

Condición: Intervalo de Fecha incorrecto.

\begin{tabular}{ccl}
 &  & \tabularnewline
1.   & \includegraphics{sistema}  & \multicolumn{1}{p{12cm}}{Muestra el mensaje de error MSG12.}\tabularnewline
2.  & \includegraphics{sistema}  & \multicolumn{1}{p{12cm}}{Continua con el paso 2.}\tabularnewline
&  & \multicolumn{1}{p{12cm}}{.... Fin de Trayectoria }\tabularnewline
\end{tabular}

\textit{Trayectoria Alternativa C}

Condición: Presiona el botón {[}Cancelar{]}.

\begin{tabular}{ccl}
 &  & \tabularnewline
1.   & \includegraphics{sistema}  & \multicolumn{1}{p{12cm}}{Regresa a la pantalla PV0.}\tabularnewline
&  & \multicolumn{1}{p{12cm}}{.... Fin del caso de uso }\tabularnewline
\end{tabular}

\textit{Trayectoria Alternativa D}

Condición: No se llenaron todos los campos del formulario.

\begin{tabular}{ccl}
 &  & \tabularnewline
1.  & \includegraphics{sistema}  & \multicolumn{1}{p{12cm}}{ Muestra el mensaje de error MSG8.}\tabularnewline
2.  & \includegraphics{sistema}  & \multicolumn{1}{p{12cm}}{Continua con el paso 3.}\tabularnewline
 &  & \multicolumn{1}{p{12cm}}{.... Fin de Trayectoria}\tabularnewline
\end{tabular}

\textit{Trayectoria Alternativa E}

Condición: No se pudo establecer conexion con la base de datos.

\begin{tabular}{ccl}
 &  & \tabularnewline
1.  & \includegraphics{sistema}  & \multicolumn{1}{p{12cm}}{ Muestra el mensaje de error MSG4.}\tabularnewline
 &  & \multicolumn{1}{p{12cm}}{.... Fin del caso de uso}\tabularnewline
\end{tabular}

\textit{Trayectoria Alternativa F}

Condición: Datos solicitados no arrojaron resultados.

\begin{tabular}{ccl}
 &  & \tabularnewline
1.  & \includegraphics{sistema}  & \multicolumn{1}{p{12cm}}{ Muestra el mensaje de error MSG3.}\tabularnewline
2.  & \includegraphics{sistema}  & \multicolumn{1}{p{12cm}}{Continua con el paso 2.}\tabularnewline
 &  & \multicolumn{1}{p{12cm}}{.... Fin de Trayectoria}\tabularnewline
\end{tabular}
\end{document}
