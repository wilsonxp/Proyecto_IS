\documentclass[10pt,spanish]{article}
\usepackage[utf8x]{inputenc}
\usepackage[letterpaper]{geometry}
\geometry{verbose}
\usepackage{amsmath}
\usepackage{amssymb}
\usepackage{graphicx}

\makeatletter
\providecommand{\tabularnewline}{\\}

\usepackage{ucs}\usepackage[spanish]{babel}
\usepackage{amsfonts}\usepackage{colortbl}
% = = = = = = = = = = = = = = = = = = = = = = = = =
% INICIO EL DOCUMENTO
% = = = = = = = = = = = = = = = = = = = = = = = = =

\usepackage{babel}
\addto\shorthandsspanish{\spanishdeactivate{~<>}}

\usepackage{babel}
\addto\shorthandsspanish{\spanishdeactivate{~<>}}

\makeatother

\usepackage{babel}
\addto\shorthandsspanish{\spanishdeactivate{~<>}}

\begin{document}

\section{Documentación de Casos de Uso}

\subsection{CUG1: Login}

\textbf{\large Resumen}{\large }\\
{\large{} Este Caso de uso tiene como objetivo permitir al usuario
ingresar al sistema.}\\


% = = = = = = = = = = = = = = = = = = = = = = = = =
%	INICIO DE TABLA
% = = = = = = = = = = = = = = = = = = = = = = = = = 

%\multicolumn{1}{p{10cm}||}{Contenido}
%
\begin{table}[!ht]
\begin{centering}
\begin{tabular}{|c||c|l|}
\hline 
\multicolumn{2}{|c|}{Caso de Uso:} & CUG1: Login\tabularnewline
\hline 
\multicolumn{3}{|>{\columncolor[gray]{0.7}}c}{Resumen de Atributos}\tabularnewline
\hline 
\multicolumn{2}{|c|}{Autor} & \multicolumn{1}{p{10cm}||}{Diaz Rodriguez Kevin Alan}\tabularnewline
\hline 
\multicolumn{2}{|c|}{Actor} & \multicolumn{1}{p{10cm}||}{Todos los usuarios}\tabularnewline
\hline 
\multicolumn{2}{|c|}{Propósito} & \multicolumn{1}{p{10cm}||}{Permitir el acceso al sistema.
}\tabularnewline
\hline 
\multicolumn{2}{|c|}{Entradas} & \multicolumn{1}{p{10cm}||}{Usuario y contraseña}\tabularnewline
\hline 
\multicolumn{2}{|c|}{Salidas} & \multicolumn{1}{p{10cm}||}{Interfaz correspondiente al usuario y contraseña ingresados}\tabularnewline
\hline 
\multicolumn{2}{|c|}{Pre-condiciones} & \multicolumn{1}{p{10cm}||}{El usuario debe estar registrado.}\tabularnewline
\hline 
\multicolumn{2}{|c|}{Pos-condiciones} & \multicolumn{1}{p{10cm}||}{Debe existir conexion con la base de datos}\tabularnewline
\hline 
\multicolumn{2}{|c|}{Errores} & \multicolumn{1}{p{10cm}||}{MSG5 Error al iniciar sesion.}\tabularnewline
\hline 
\multicolumn{2}{|c|}{Tipo} & \multicolumn{1}{p{10cm}||}{Primario}\tabularnewline
\hline 
\multicolumn{2}{|c|}{Fuente} & \multicolumn{1}{p{10cm}||}{Basado en el funcionamiento}\tabularnewline
\hline 
\end{tabular}
\par\end{centering}

\caption{Caso de Uso 1: Login}
\label{tab:CasosdeUso:nombredecasodeuso} 
\end{table}


% = = = = = = = = = = = = = = = = = = = = = = = = =
%	FIN DE TABLA
% = = = = = = = = = = = = = = = = = = = = = = = = =
% = = = = = = = = = = = = = = = = = = = = = = = = =
%	INICIO DE TRAYECTORIA
% = = = = = = = = = = = = = = = = = = = = = = = = = 


\textbf{\large Trayectorias del CU}{\large \par}

\textit{\large Trayectoria Principal}{\large {} }{\large \par}


\begin{tabular}{ccl}
 &  & \tabularnewline
1.  & \includegraphics{sistema}  & \multicolumn{1}{p{12cm}}{Muestra pantalla de inicio de sesion PG1}\tabularnewline
2.  & \includegraphics{actor}  & \multicolumn{1}{p{12cm}}{Ingresa Usuario y contraseña.}\tabularnewline
3.  & \includegraphics{sistema}  & \multicolumn{1}{p{12cm}}{Verifica que el usuario y contraseña sean correctos.{[}Trayectoria Alternativa A{]}}\tabularnewline
4.  & \includegraphics{sistema}  & \multicolumn{1}{p{12cm}}{Despliega interfaz correspondiente al usuario y contraseña.}\tabularnewline
 &  & \multicolumn{1}{p{12cm}}{.... Fin del caso de uso}\tabularnewline
\end{tabular}

\textit{Trayectoria Alternativa A}

Condición: Usuario o Contraseña incorrecta

\begin{tabular}{ccl}
 &  & \tabularnewline
1.  & \includegraphics{sistema}  & \multicolumn{1}{p{12cm}}{ Muestra el mensaje de error MSG5}\tabularnewline
2.  & \includegraphics{sistema}  & \multicolumn{1}{p{12cm}}{Continua con el paso 2}\tabularnewline
 &  & \multicolumn{1}{p{12cm}}{.... Fin de la Trayectoria}\tabularnewline
\end{tabular}

% = = = = = = = = = = = = = = = = = = = = = = = = =
%	FIN DEL DOCUMENTO
% = = = = = = = = = = = = = = = = = = = = = = = = =

\end{document}