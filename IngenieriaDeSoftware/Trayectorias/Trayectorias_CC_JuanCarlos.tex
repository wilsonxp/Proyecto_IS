%% LyX 2.0.3 created this file.  For more info, see http://www.lyx.org/.
%% Do not edit unless you really know what you are doing.
\documentclass[10pt,spanish]{article}
\usepackage[utf8x]{inputenc}
\usepackage[letterpaper]{geometry}
\geometry{verbose}
\usepackage{amsmath}
\usepackage{amssymb}
\usepackage{graphicx}

\makeatletter

%%%%%%%%%%%%%%%%%%%%%%%%%%%%%% LyX specific LaTeX commands.
%% Because html converters don't know tabularnewline
\providecommand{\tabularnewline}{\\}

%%%%%%%%%%%%%%%%%%%%%%%%%%%%%% User specified LaTeX commands.


%%%%%%%%%%%%%%%%%%%%%%%%%%%%%% LyX specific LaTeX commands.
%% Because html converters don't know tabularnewline
\providecommand{\tabularnewline}{\\}

%%%%%%%%%%%%%%%%%%%%%%%%%%%%%% User specified LaTeX commands.

\usepackage{ucs}\usepackage[spanish]{babel}
\usepackage{amsfonts}\usepackage{colortbl}% = = = = = = = = = = = = = = = = = = = = = = = = =
% INICIO EL DOCUMENTO
% = = = = = = = = = = = = = = = = = = = = = = = = =

\usepackage{babel}
\addto\shorthandsspanish{\spanishdeactivate{~<>}}

\usepackage{babel}
\addto\shorthandsspanish{\spanishdeactivate{~<>}}



\usepackage{babel}
\addto\shorthandsspanish{\spanishdeactivate{~<>}}

\makeatother

\usepackage{babel}
\addto\shorthandsspanish{\spanishdeactivate{~<>}}

\begin{document}

\section{Módulo Control de Calidad}

\label{sec:MCC}


\subsection{CUPCC1.0 : Visualizar problemas }

\textbf{\large Resumen}\\
 {\large {}En este caso de uso el actor podrá conocer los problemas
que se han detectado tanto en las materias primas, como en los productos.
Posteriormente, si asi lo desea, podrá generar un reporte.}\\


% = = = = = = = = = = = = = = = = = = = = = = = = =
%	INICIO DE TABLA
% = = = = = = = = = = = = = = = = = = = = = = = = = 
\begin{table}[!ht]
\begin{centering}
\begin{tabular}{|c||c|l|}
\hline 
\multicolumn{2}{|c|}{Caso de Uso:} & CUCC 1.0: Visualizar problemas\tabularnewline
\hline 
\multicolumn{3}{|>{\columncolor[gray]{0.7}}c}{Resumen de Atributos}\tabularnewline
\hline 
\multicolumn{2}{|c|}{Autor} & \multicolumn{1}{p{10cm}||}{Pérez Alvarez Juan Carlos}\tabularnewline
\hline 
\multicolumn{2}{|c|}{Actor} & \multicolumn{1}{p{10cm}||}{Personal de CC}\tabularnewline
\hline 
\multicolumn{2}{|c|}{Propósito} & \multicolumn{1}{p{10cm}||}{Visualizar los problemas que se han detectado.}\tabularnewline
\hline 
\multicolumn{2}{|c|}{Entradas} & \multicolumn{1}{p{10cm}||}{
-Status: Todos los reportes, Pendiente o Solucionado \newline
-Intervalo de tiemp: Todos los reportes o definido por una fecha de inicio y una fecha de fin \newline
-Etapa en el proceso de producción.}\tabularnewline
\hline 
\multicolumn{2}{|c|}{Salidas} & \multicolumn{1}{p{10cm}||}{Listado con los reportes, de acuerdo a los criterios especificados}\tabularnewline
\hline 
\multicolumn{2}{|c|}{Pre-condiciones} & \multicolumn{1}{p{10cm}||}{Estar identificado como un usuario del area de control de calidad.}\tabularnewline
\hline 
\multicolumn{2}{|c|}{Post-condiciones} & \multicolumn{1}{p{10cm}||}{El actor visualiza los problemas reportados}\tabularnewline
\hline 
\multicolumn{2}{|c|}{Errores} & \multicolumn{1}{p{10cm}||}{
MSG3 'Datos no encontrados'\newline
MSG4 'Error al conectar con la base de datos'\newline
MSG11 'La fecha de inicio es posterior a la fecha de fin'\newline
MSG12 'La fecha indicada es posterior a la fecha del dia de hoy'}\tabularnewline
\hline 
\multicolumn{2}{|c|}{Tipo} & \multicolumn{1}{p{10cm}||}{Primario}\tabularnewline
\hline 
\end{tabular}
\par\end{centering}

\caption{Caso de Uso 1.0: Visualizar problemas}


\label{tab:CasosdeUso:nombredecasodeuso} 
\end{table}


% = = = = = = = = = = = = = = = = = = = = = = = = =
%	FIN DE TABLA
% = = = = = = = = = = = = = = = = = = = = = = = = =
% = = = = = = = = = = = = = = = = = = = = = = = = =
%	INICIO DE TRAYECTORIA
% = = = = = = = = = = = = = = = = = = = = = = = = = 


\textbf{\large Trayectorias del CU}{\large \par}

\textit{\large Trayectoria Principal}{\large {} }{\large \par}

%Solo hay que cambiar el nombre de la imagen dependiendo de si es actor o sistema


\begin{tabular}{ccl}
 &  & \tabularnewline
1.  & \includegraphics{actor}  & \multicolumn{1}{p{12cm}}{ Oprime en la opción {[}Visualizar problemas{]} en la pantalla PCC1}\tabularnewline
2.  & \includegraphics{sistema}  & \multicolumn{1}{p{12cm}}{Despliega la pantalla PCC2 con el formulario correspondiente.}\tabularnewline
3.  & \includegraphics{actor}  & \multicolumn{1}{p{12cm}}{Selecciona el criterio de filtrado de acuerdo a la Etapa de Proceso: {[}Materia prima{]}, {[}Linea de Producción{]} ó {[}Producto{]}}\tabularnewline
4.  & \includegraphics{actor}  & \multicolumn{1}{p{12cm}}{Selecciona la opción {[}Todos los reportes{]} en el apartado de {[}Fecha{]}{[}Trayectoria Alternativa B{]}}\tabularnewline
5.  & \includegraphics{actor}  & \multicolumn{1}{p{12cm}}{Selecciona la opción {[}Todos los reportes{]} en el apartado de {[}Status{]}{[}Trayectoria Alternativa C{]}}\tabularnewline
6.  & \includegraphics{actor}  & \multicolumn{1}{p{12cm}}{Oprime el botón {[}Visualizar reportes{]}{[}Trayectoria Alternativa A{]}}\tabularnewline
7.  & \includegraphics{sistema}  & \multicolumn{1}{p{12cm}}{La pantalla muestra un listado con la información correspondiente en la sección correspondiente.{[}Trayectoria Alternativa D{]}{[}Trayectoria Alternativa G{]} }\tabularnewline
8.  & \includegraphics{actor}  & \multicolumn{1}{p{12cm}}{ Selecciona la opción {[}Ver Reporte{]} de un elemento de la lista
de problemas desplegada }\tabularnewline
 &  & \multicolumn{1}{p{12cm}}{.... Fin del caso de uso}\tabularnewline
\end{tabular}

\newpage{}

\textit{Trayectoria Alternativa A}

Condición: El actor oprime el botón de cancelar

\begin{tabular}{ccl}
 &  & \tabularnewline
1.  & \includegraphics{actor}  & \multicolumn{1}{p{12cm}}{Oprime el botón de {[}Cancelar{]}}\tabularnewline
2.  & \includegraphics{sistema}  & \multicolumn{1}{p{12cm}}{Despliega la pantalla PCC1}\tabularnewline
 &  & \multicolumn{1}{p{12cm}}{.... Fin del caso de uso}\tabularnewline
\end{tabular}
\newline
\textit{Trayectoria Alternativa B}

Condición: El actor eligió la opción {[}Periodo de tiempo{]}

\begin{tabular}{ccl}
 &  & \tabularnewline
1.  & \includegraphics{sistema}  & \multicolumn{1}{p{12cm}}{ Desbloquea los campos {[}Fecha de inicio{]} y {[}Fecha de fin{]}
{[}Trayectoria Alternativa E{]} {[}Trayectoria Alternativa F{]}}\tabularnewline
2.  & \includegraphics{actor}  & \multicolumn{1}{p{12cm}}{ Ingresa los valores de {[}Fecha de inicio{]} y {[}Fecha de fin{]} }\tabularnewline
3.  & \includegraphics{sistema}  & \multicolumn{1}{p{12cm}}{Continua con el paso 5 de la Trayectoria principal}\tabularnewline
 &  & \multicolumn{1}{p{12cm}}{.... Fin de la trayectoria}\tabularnewline
\end{tabular}
\newline
\textit{Trayectoria Alternativa C}

Condición: El actor eligió la opción {[}Status{]}

\begin{tabular}{ccl}
 &  & \tabularnewline
1.  & \includegraphics{sistema}  & \multicolumn{1}{p{12cm}}{ Desbloquea el campo {[}Status{]}}\tabularnewline
2.  & \includegraphics{actor}  & \multicolumn{1}{p{12cm}}{ Selecciona una opción para el campo {[}Status{]} }\tabularnewline
3.  & \includegraphics{sistema}  & \multicolumn{1}{p{12cm}}{Continua con el paso 6 de la Trayectoria principal}\tabularnewline
 &  & \multicolumn{1}{p{12cm}}{.... Fin de la trayectoria}\tabularnewline
\end{tabular}
\newline
\textit{Trayectoria Alternativa D}

Condición: No hay reportes por mostrar

\begin{tabular}{ccl}
 &  & \tabularnewline
1.  & \includegraphics{sistema}  & \multicolumn{1}{p{12cm}}{ Muestra el mensaje de error MSG3 'Datos no encontrados'}\tabularnewline
2.  & \includegraphics{sistema}  & \multicolumn{1}{p{12cm}}{ Continua con el paso 3 de la Trayectoria principal }\tabularnewline
 &  & \multicolumn{1}{p{12cm}}{.... Fin de la trayectoria}\tabularnewline
\end{tabular}
\newline
\textit{Trayectoria Alternativa E}

Condición: La fecha de inicio es posterior a la fecha de fin

\begin{tabular}{ccl}
 &  & \tabularnewline
1.  & \includegraphics{sistema}  & \multicolumn{1}{p{12cm}}{ Muestra el mensaje de error MSG11 'La fecha de inicio es posterior
a la fecha de fin'}\tabularnewline
2.  & \includegraphics{sistema}  & \multicolumn{1}{p{12cm}}{ Continua con el paso 2 de la Trayectoria alternativa B }\tabularnewline
 &  & \multicolumn{1}{p{12cm}}{.... Fin de la trayectoria}\tabularnewline
\end{tabular}
\newline
\textit{Trayectoria Alternativa F}

Condición: La fecha de inicio o la fecha de fin son posteriores a
la fecha actual

\begin{tabular}{ccl}
 &  & \tabularnewline
1.  & \includegraphics{sistema}  & \multicolumn{1}{p{12cm}}{ Muestra el mensaje de error MSG12 'La fecha indicada es posterior
a la fecha del dia de hoy'}\tabularnewline
2.  & \includegraphics{sistema}  & \multicolumn{1}{p{12cm}}{ Continua con el paso 2 de la Trayectoria alternativa B }\tabularnewline
 &  & \multicolumn{1}{p{12cm}}{.... Fin de la trayectoria}\tabularnewline
\end{tabular}
\newline
\textit{Trayectoria Alternativa G}

Condición: Error de conexión a base de datos

\begin{tabular}{ccl}
 &  & \tabularnewline
1.  & \includegraphics{sistema}  & \multicolumn{1}{p{12cm}}{ Muestra el mensaje de error MSG4 'Error al conectar con la base de datos'}\tabularnewline
2.  & \includegraphics{sistema}  & \multicolumn{1}{p{12cm}}{ Continua con el paso 2 de la Trayectoria alternativa B }\tabularnewline
 &  & \multicolumn{1}{p{12cm}}{.... Fin de la trayectoria}\tabularnewline
\end{tabular}

\newpage{}


\subsubsection{CUPCC1.1: Ver detalles del problema}

\textbf{\large Resumen}\\
 {\large {}En este caso de uso el actor puede visualizar los detalles
de un problema relacionado con Control de Calidad reportado en el sistema}\\


% = = = = = = = = = = = = = = = = = = = = = = = = =
%	INICIO DE TABLA
% = = = = = = = = = = = = = = = = = = = = = = = = = 
\begin{table}[!ht]
\begin{centering}
\begin{tabular}{|c||c|l|}
\hline 
\multicolumn{2}{|c|}{Caso de Uso:} & CU CC 1.1: Ver detalles del problema\tabularnewline
\hline 
\multicolumn{3}{|>{\columncolor[gray]{0.7}}c}{Resumen de Atributos}\tabularnewline
\hline 
\multicolumn{2}{|c|}{Autor} & \multicolumn{1}{p{10cm}||}{Pérez Alvarez Juan Carlos}\tabularnewline
\hline 
\multicolumn{2}{|c|}{Actor} & \multicolumn{1}{p{10cm}||}{Personal de CC}\tabularnewline
\hline 
\multicolumn{2}{|c|}{Propósito} & \multicolumn{1}{p{10cm}||}{Visualizar detalles de un problema.}\tabularnewline
\hline 
\multicolumn{2}{|c|}{Entradas} & \multicolumn{1}{p{10cm}||}{Ninguna}\tabularnewline
\hline 
\multicolumn{2}{|c|}{Salidas} & \multicolumn{1}{p{10cm}||}{Información acerca del problema}\tabularnewline
\hline 
\multicolumn{2}{|c|}{Pre-condiciones} & \multicolumn{1}{p{10cm}||}{
-Estar identificado como un usuario del area de control de calidad.\newline
-Haber seleccionado una opción de la PCC2.}\tabularnewline
\hline 
\multicolumn{2}{|c|}{Post-condiciones} & \multicolumn{1}{p{10cm}||}{El actor visualiza los detalles del problema}\tabularnewline
\hline 
\multicolumn{2}{|c|}{Errores} & \multicolumn{1}{p{10cm}||}{MSG4 'Error al conectar con la base de datos'}\tabularnewline
\hline 
\multicolumn{2}{|c|}{Tipo} & \multicolumn{1}{p{10cm}||}{Secundario}\tabularnewline
\hline 
\multicolumn{2}{|c|}{Fuente} & \multicolumn{1}{p{10cm}||}{CU CC 17.1}\tabularnewline
\hline 
\end{tabular}
\par\end{centering}

\caption{Caso de Uso 17.2: Ver detalles del problema}


\label{tab:CasosdeUso:nombredecasodeuso} 
\end{table}


% = = = = = = = = = = = = = = = = = = = = = = = = =
%	FIN DE TABLA
% = = = = = = = = = = = = = = = = = = = = = = = = =
% = = = = = = = = = = = = = = = = = = = = = = = = =
%	INICIO DE TRAYECTORIA
% = = = = = = = = = = = = = = = = = = = = = = = = = 


\textbf{\large Trayectorias del CU}{\large \par}

\textit{\large Trayectoria Principal}{\large {} }{\large \par}

%Solo hay que cambiar el nombre de la imagen dependiendo de si es actor o sistema


\begin{tabular}{ccl}
 &  & \tabularnewline
1.  & \includegraphics{actor}  & \multicolumn{1}{p{12cm}}{ Oprime en la opción {[}Ver reporte{]} en la pantalla PCC2}\tabularnewline
2.  & \includegraphics{sistema}  & \multicolumn{1}{p{12cm}}{ Carga la información del problema seleccionado.{[}Trayectoria Alternativa C{]}}\tabularnewline
3.  & \includegraphics{actor}  & \multicolumn{1}{p{12cm}}{Oprime el botón de {[}Atrás{]} {[}Trayectoria Alternativa A{]}{[}Trayectoria Alternativa B{]}}\tabularnewline
4.  & \includegraphics{sistema}  & \multicolumn{1}{p{12cm}}{Despliega la pantalla PCC3}\tabularnewline
 &  & \multicolumn{1}{p{12cm}}{.... Fin del caso de uso}\tabularnewline
\end{tabular}

\newpage{}\textit{Trayectoria Alternativa A}

Condición: El actor oprime el botón de Imprimir

\begin{tabular}{ccl}
 &  & \tabularnewline
1.  & \includegraphics{actor}  & \multicolumn{1}{p{12cm}}{Oprime el botón de {[}Imprimir{]}}\tabularnewline
2.  & \includegraphics{sistema}  & \multicolumn{1}{p{12cm}}{Despliega abre el archivo en formato PDF}\tabularnewline
3.  & \includegraphics{sistema}  & \multicolumn{1}{p{12cm}}{Continua con el paso 3 de la Trayectoria principal}\tabularnewline
 &  & \multicolumn{1}{p{12cm}}{.... Fin de la trayectoria}\tabularnewline
\end{tabular}
\newline
\textit{Trayectoria Alternativa B}

Condición: El actor oprime el botón de Levantar Reporte

\begin{tabular}{ccl}
 &  & \tabularnewline
1.  & \includegraphics{actor}  & \multicolumn{1}{p{12cm}}{Oprime el botón de {[}Levantar Reporte{]}}\tabularnewline
2.  & \includegraphics{sistema}  & \multicolumn{1}{p{12cm}}{Despliega la pantalla PG3 'Enviar notificación'}\tabularnewline
3.  & \includegraphics{actor}  & \multicolumn{1}{p{12cm}}{ Interactua con el caso de uso CUG1.1 'Enviar notificación'}\tabularnewline
4.  & \includegraphics{sistema}  & \multicolumn{1}{p{12cm}}{Continua con el paso 3 de la Trayectoria principal}\tabularnewline
 &  & \multicolumn{1}{p{12cm}}{.... Fin de la trayectoria}\tabularnewline
\end{tabular}
\newline
\textit{Trayectoria Alternativa C}

Condición: Error de conexión a base de datos

\begin{tabular}{ccl}
 &  & \tabularnewline
1.  & \includegraphics{sistema}  & \multicolumn{1}{p{12cm}}{ Muestra el mensaje de error MSG4 'Error al conectar con la base de datos'}\tabularnewline
2.  & \includegraphics{sistema}  & \multicolumn{1}{p{12cm}}{ Continua con el paso 2 de la Trayectoria alternativa B }\tabularnewline
 &  & \multicolumn{1}{p{12cm}}{.... Fin de la trayectoria}\tabularnewline
\end{tabular}
\newpage{}

% = = = = = = = = = = = = = = = = = = = = = = = = =
%	
%		INICIO CASO DE USO 18.1
% = = = = = = = = = = = = = = = = = = = = = = = = = 



\subsection{CUCC 2.0: Rastrear producto}

\textbf{\large Resumen}\\
 {\large {}En este caso de uso el actor puede localizar un producto,
a través del lote de materia prima, o por el lote de producto, para conocer su ubicación actual}\\

% = = = = = = = = = = = = = = = = = = = = = = = = =
%	INICIO DE TABLA
% = = = = = = = = = = = = = = = = = = = = = = = = = 
\begin{table}[!ht]
\begin{centering}
\begin{tabular}{|c||c|l|}
\hline 
\multicolumn{2}{|c|}{Caso de Uso:} & CU CC 18.1: Rastrear producto\tabularnewline
\hline 
\multicolumn{3}{|>{\columncolor[gray]{0.7}}c}{Resumen de Atributos}\tabularnewline
\hline 
\multicolumn{2}{|c|}{Autor} & \multicolumn{1}{p{10cm}||}{Pérez Alvarez Juan Carlos}\tabularnewline
\hline 
\multicolumn{2}{|c|}{Actor} & \multicolumn{1}{p{10cm}||}{Personal de CC}\tabularnewline
\hline 
\multicolumn{2}{|c|}{Propósito} & \multicolumn{1}{p{10cm}||}{Localizar la ubicación actual de un producto.}\tabularnewline
\hline 
\multicolumn{2}{|c|}{Entradas} & \multicolumn{1}{p{10cm}||}{Se solicitara al usuario el tipo de Número de lote y el Número de lote de materia prima, o el Número de lote de producto}\tabularnewline
\hline 
\multicolumn{2}{|c|}{Salidas} & \multicolumn{1}{p{10cm}||}{Informe de la ubicación del producto}\tabularnewline
\hline 
\multicolumn{2}{|c|}{Pre-condiciones} & \multicolumn{1}{p{10cm}||}{Estar identificado como un usuario del area de control de calidad.}\tabularnewline
\hline 
\multicolumn{2}{|c|}{Post-condiciones} & \multicolumn{1}{p{10cm}||}{El actor visualiza la ubicación del producto}\tabularnewline
\hline 
\multicolumn{2}{|c|}{Errores} & \multicolumn{1}{p{10cm}||}{
MSG10 'Registro inexistente'\newline
MSG4 'Error al conectar con la base de datos'\newline
MSG7 'Formato incorrecto de los datos'
}\tabularnewline
\hline 
\multicolumn{2}{|c|}{Tipo} & \multicolumn{1}{p{10cm}||}{Primario}\tabularnewline
\end{tabular}
\par\end{centering}
\caption{Caso de Uso 18.1: Rastrear producto}


\label{tab:CasosdeUso:nombredecasodeuso} 
\end{table}


% = = = = = = = = = = = = = = = = = = = = = = = = =
%	FIN DE TABLA
% = = = = = = = = = = = = = = = = = = = = = = = = =
% = = = = = = = = = = = = = = = = = = = = = = = = =
%	INICIO DE TRAYECTORIA
% = = = = = = = = = = = = = = = = = = = = = = = = = 


\textbf{\large Trayectorias del CU}{\large \par}

\textit{\large Trayectoria Principal}{\large {} }{\large \par}

%Solo hay que cambiar el nombre de la imagen dependiendo de si es actor o sistema


\begin{tabular}{ccl}
 &  & \tabularnewline
1.  & \includegraphics{actor}  & \multicolumn{1}{p{12cm}}{ Oprime en la opción {[}Seguimiento del producto{]} en la pantalla PCC1}\tabularnewline
2.  & \includegraphics{sistema}  & \multicolumn{1}{p{12cm}}{Despliega la pantalla PCC4 con el formulario.}\tabularnewline
3.  & \includegraphics{actor}  & \multicolumn{1}{p{12cm}}{Selecciona el criterio de búsqueda: Rastrear por número de lote de
materia prima, o por número de lote de producto}\tabularnewline
4.  & \includegraphics{actor}  & \multicolumn{1}{p{12cm}}{Ingresa el número de lote en el campo {[}Número de lote{]}}\tabularnewline
5.  & \includegraphics{actor}  & \multicolumn{1}{p{12cm}}{Oprime el botón {[}Rastrear producto{]} {[}Trayectoria Alternativa
A{]}}\tabularnewline
6.  & \includegraphics{sistema}  & \multicolumn{1}{p{12cm}}{Muestra en la pantalla PCC4 la información correspondiente. {[}Trayectoria Alternativa B{]}{[}Trayectoria Alternativa C{]}{[}Trayectoria Alternativa D{]}}\tabularnewline
 &  & \multicolumn{1}{p{12cm}}{.... Fin del caso de uso}\tabularnewline
\end{tabular}
\newline
\newpage{}\textit{Trayectoria Alternativa A}

Condición: El actor oprime el botón de cancelar

\begin{tabular}{ccl}
 &  & \tabularnewline
1.  & \includegraphics{actor}  & \multicolumn{1}{p{12cm}}{Oprime el botón de {[}Cancelar{]}}\tabularnewline
2.  & \includegraphics{sistema}  & \multicolumn{1}{p{12cm}}{Despliega la pantalla PCC1}\tabularnewline
 &  & \multicolumn{1}{p{12cm}}{.... Fin del caso de uso}\tabularnewline
\end{tabular}
\newline
\textit{Trayectoria Alternativa B}

Condición: El número de lote no existe

\begin{tabular}{ccl}
 &  & \tabularnewline
1.  & \includegraphics{sistema}  & \multicolumn{1}{p{12cm}}{ Muestra el mensaje de error MSG10 'Registro inexistente'}\tabularnewline
2.  & \includegraphics{sistema}  & \multicolumn{1}{p{12cm}}{Continua con el paso 3 de la Trayectoria principal}\tabularnewline
 &  & \multicolumn{1}{p{12cm}}{.... Fin de la trayectoria}\tabularnewline
\end{tabular}
\newline
\textit{Trayectoria Alternativa C}

Condición: Error de conexión a base de datos

\begin{tabular}{ccl}
 &  & \tabularnewline
1.  & \includegraphics{sistema}  & \multicolumn{1}{p{12cm}}{ Muestra el mensaje de error MSG4 'Error al conectar con la base de datos'}\tabularnewline
2.  & \includegraphics{sistema}  & \multicolumn{1}{p{12cm}}{ Continua con el paso 2 de la Trayectoria alternativa B }\tabularnewline
 &  & \multicolumn{1}{p{12cm}}{.... Fin de la trayectoria}\tabularnewline
\end{tabular}
\newline
\textit{Trayectoria Alternativa D}

Condición: El formato de número de lote no es correcto de acuerdo a la Regla de Negocio {[}BRP5 Formato de lote de producto{]}

\begin{tabular}{ccl}
 &  & \tabularnewline
1.  & \includegraphics{sistema}  & \multicolumn{1}{p{12cm}}{ Muestra el mensaje de error MSG7 'Formato incorrecto de los datos'}\tabularnewline
2.  & \includegraphics{sistema}  & \multicolumn{1}{p{12cm}}{ Continua con el paso 2 de la Trayectoria alternativa B }\tabularnewline
 &  & \multicolumn{1}{p{12cm}}{.... Fin de la trayectoria}\tabularnewline
\end{tabular}

% = = = = = = = = = = = = = = = = = = = = = = = = =
%	FIN DEL DOCUMENTO
% = = = = = = = = = = = = = = = = = = = = = = = = =

\end{document}
