\documentclass[10pt,spanish]{article}
\usepackage[utf8x]{inputenc}
\usepackage[letterpaper]{geometry}
\geometry{verbose}
\usepackage{amsmath}
\usepackage{amssymb}
\usepackage{graphicx}

\makeatletter

\providecommand{\tabularnewline}{\\}

\usepackage{ucs}\usepackage[spanish]{babel}
\usepackage{amsfonts}\usepackage{colortbl}
% = = = = = = = = = = = = = = = = = = = = = = = = =
% INICIO EL DOCUMENTO
% = = = = = = = = = = = = = = = = = = = = = = = = =


\makeatother

\usepackage{babel}
\addto\shorthandsspanish{\spanishdeactivate{~<>}}

\begin{document}

	\section{Documentación de Casos de Uso}

		\subsection{CUP1: Consultar estado del almacén}

		\textbf{\large Resumen}{\large }
		{Este CU tiene como objetivo permitir al usuario de producción consultar el estado del almacén con el fin de saber si empezar o no la producción.}\\
		{\large \par}

		% = = = = = = = = = = = = = = = = = = = = = = = = =
		%	INICIO DE TABLA
		% = = = = = = = = = = = = = = = = = = = = = = = = = 
		\begin{table}[!ht]
		\begin{centering}
		\begin{tabular}{|c||c|l|}
		\hline 
		\multicolumn{2}{|c|}{Caso de Uso:} & CUP1: Consultar estado del almacén\tabularnewline
		\hline 
		\multicolumn{3}{|>{\columncolor[gray]{0.7}}c}{Resumen de Atributos}\tabularnewline
		\hline 
		\multicolumn{2}{|c|}{Actor} & Usuario de Producción\tabularnewline
		\hline 
		\multicolumn{2}{|c|}{Propósito} & Permitir al usuario consultar el estado del almacén.\tabularnewline
		\hline 
		\multicolumn{2}{|c|}{Entradas} & Ninguna\tabularnewline
		\hline 
		\multicolumn{2}{|c|}{Salidas} & Se mostrara en pantalla el mensaje MSGP1: Disponibilidad en Almacén.\tabularnewline
		\hline 
		\multicolumn{2}{|c|}{Pre-condiciones} &  \multicolumn{1}{p{12cm}|}{Que existan registros de nivel de capacidad en el almacén.\newline
		Debe existir conexion con la BD.}\tabularnewline
		\hline 
		\multicolumn{2}{|c|}{Pos-condiciones} & Se mostrara en pantalla el mensaje MSGP1: Disponibilidad en Almacén.\tabularnewline
		\hline 
		\multicolumn{2}{|c|}{Errores} & \multicolumn{1}{p{12cm}|}{Se mostrara el mensaje MSGP2: Información de Almacén no disponible, cuando no se tengan registros del almacén.\newline
		MSG4 Error al obtener conexión con la base de datos}\tabularnewline
		\hline 
		\multicolumn{2}{|c|}{Tipo} & Principal\tabularnewline
		\hline 
		\multicolumn{2}{|c|}{Fuente} & -\tabularnewline
		\hline 
		\multicolumn{2}{|c|}{RN Relacionada} & Este CU satisface la BR3 Limite de almacén\tabularnewline
		\hline 		
		\multicolumn{2}{|c|}{Autor} & Rueda López Miguel Angel\tabularnewline
		\hline 
		\multicolumn{2}{|c|}{Fecha} & 26 de febrero de 2013\tabularnewline
		\hline 				
		\end{tabular}
		\par\end{centering}
		
	\caption{CUP1 Consultar estado del almacén}
	\label{tab:CasosdeUso:nombredecasodeuso} 
	\end{table}


	% = = = = = = = = = = = = = = = = = = = = = = = = =
	%	FIN DE TABLA
	% = = = = = = = = = = = = = = = = = = = = = = = = =
	% = = = = = = = = = = = = = = = = = = = = = = = = =
	%	INICIO DE TRAYECTORIA
	% = = = = = = = = = = = = = = = = = = = = = = = = = 
	\textbf{\large Trayectorias del CU}{\large \par}

	\textit{\large Trayectoria Principal}{\large{} }{\large \par}

	\begin{tabular}{ccl}
	&  & \tabularnewline
	1. & \includegraphics{actor} & solicita consultar el estado del almacén mediante el botón [Consultar estado del\tabularnewline
	& & Almacén] de la pantalla PP2 Consultar estado del almacén.\tabularnewline
	2. & \includegraphics{sistema} &  busca los registros correspondientes del estado del almacén más reciente. \tabularnewline
	& & [Trayectoria A][Trayectoria B]\tabularnewline
	3. & \includegraphics{sistema} & muestra el mensaje MSGP1: Disponibilidad en el almacén.\tabularnewline
	 &  & .... Fin del caso de uso\tabularnewline \\
	\end{tabular}
	\newpage
	\textit{Trayectoria Alternativa A}\\
	Condición: No se encuentra la información de disponibilidad del almacén.\\
	\begin{tabular}{ccl}
	& & \tabularnewline
	A-1 & \includegraphics{sistema} & Muestra el mensaje MSGP2: Información de Almacén no disponible en la pantalla \tabularnewline
	& & PP1 Menú principal de producción.\tabularnewline
	A-2 & & Fin del Caso de Uso\tabularnewline
	\end{tabular}
	\newline
	\textit{Trayectoria Alternativa B}\\
	Condición: No se pudo establecer conexión con la BD.\\
	\begin{tabular}{ccl}
	& & \tabularnewline
	B-1 & \includegraphics{sistema} & Muestra el mensaje de error MSG4 'No se pudo establecer conexión con la base\tabularnewline
	& & de datos.\tabularnewline
	B-2 & \includegraphics{sistema} & Regresa a la pantalla PP1 Menú principal de producción.\tabularnewline
	B-3 & & Fin del Caso de Uso\tabularnewline	
	\end{tabular}

% = = = = = = = = = = = = = = = = = = = = = = = = = = = = = =
% Consultar pedidos en espera
% = = = = = = = = = = = = = = = = = = = = = = = = = = = = = =
		\newpage
		\subsection{CUP2: Consultar pedidos en espera.}

		\textbf{\large Resumen}{\large }\\
		{Este CU tiene como objetivo permitir al usuario de producción consultar los pedidos que se tienen pendientes.}\\
		{\large \par}

		% = = = = = = = = = = = = = = = = = = = = = = = = =
		%	INICIO DE TABLA
		% = = = = = = = = = = = = = = = = = = = = = = = = = 
		\begin{table}[!ht]
		\begin{centering}
		\begin{tabular}{|c||c|l|}
		\hline 
		\multicolumn{2}{|c|}{Caso de Uso:} & CUP2: Consultar pedidos en espera.\tabularnewline
		\hline 
		\multicolumn{3}{|>{\columncolor[gray]{0.7}}c}{Resumen de Atributos}\tabularnewline
		\hline 
		\multicolumn{2}{|c|}{Actor} & Usuario de Producción\tabularnewline
		\hline 
		\multicolumn{2}{|c|}{Propósito} & \multicolumn{1}{p{12cm}|}{Permitir al usuario consultar los pedidos que se tienen en espera por falta de producción o por motivo de cambio.}\tabularnewline
		\hline 
		\multicolumn{2}{|c|}{Entradas} & \multicolumn{1}{p{12cm}|}{Para llevar a cabo esta consulta, el usuario debe seleccionar el parámetro de búsqueda, que podría ser:\newline
•	Pedidos pendientes\newline
•	Pedidos para cambio\newline
En caso de seleccionar Pedidos para cambio, el usuario ingresara la fecha en que se realizó esa venta, para comprobar la congruencia del pedido en espera.
}\tabularnewline
		\hline 
		\multicolumn{2}{|c|}{Salidas} & \multicolumn{1}{p{12cm}|}{Se mostrara en pantalla una tabla que contenga información del pedido, acorde a la BRP2 Contenido del pedido:\newline
•	Numero de pedido\newline
•	Nombre del cliente\newline
•	Descripción del pedido\newline
Cuando no se cuente con pedidos pendientes se desplegara el mensaje MSGP3: No existen pedidos pendientes.
}\tabularnewline
		\hline 
		\multicolumn{2}{|c|}{Pre-condiciones} & \multicolumn{1}{p{12cm}|}{Que existan registros de pedidos pendientes.\newline
Que exista dentro del catálogo de parámetros, el parámetro solicitado.\newline
		Debe existir conexion con la BD.}\tabularnewline
		\hline 
		\multicolumn{2}{|c|}{Pos-condiciones} & Se mostrara en pantalla el mensaje MSGP3: No existen pedidos pendientes.\tabularnewline
		\hline 
		\multicolumn{2}{|c|}{Errores} & \multicolumn{1}{p{12cm}|}{Se mostrara el mensaje MSG3: Datos no encontrados, cuando no existan registros sobre pedidos.\newline
		MSG4 Error al obtener conexión con la base de datos}\tabularnewline
		\hline 
		\multicolumn{2}{|c|}{Tipo} & Principal\tabularnewline
		\hline 
		\multicolumn{2}{|c|}{Fuente} & Basado en el funcionamiento del CUV1.1 Registrar Venta\tabularnewline
		\hline 
		\multicolumn{2}{|c|}{RN Relacionada} & \multicolumn{1}{p{12cm}|}{Este CU satisface la BR4 Cliente de prioritario, en relación con los cambios en los pedidos.}\tabularnewline
		\hline 		
		\multicolumn{2}{|c|}{Autor} & Rueda López Miguel Angel\tabularnewline
		\hline 
		\multicolumn{2}{|c|}{Fecha} & 26 de febrero de 2013\tabularnewline
		\hline 				
		\end{tabular}
		\par\end{centering}
		
	\caption{CUP2 Consultar pedidos en espera}
	\label{tab:CasosdeUso:nombredecasodeuso} 
	\end{table}


	% = = = = = = = = = = = = = = = = = = = = = = = = =
	%	FIN DE TABLA
	% = = = = = = = = = = = = = = = = = = = = = = = = =
	% = = = = = = = = = = = = = = = = = = = = = = = = =
	%	INICIO DE TRAYECTORIA
	% = = = = = = = = = = = = = = = = = = = = = = = = = 
	\newpage
	\textbf{\large Trayectorias del CU}{\large \par}
	\textit{\large Trayectoria Principal}{\large{} }{\large \par}
	\begin{tabular}{ccl}
	 &  & \tabularnewline
	1. & \includegraphics{actor} & selecciona la opción ‘Pedidos pendientes’ de la lista ‘Seleccionar un parámetro de\tabularnewline
	& &  búsqueda’ de la pantalla PP3 Consultar pedidos en espera. [Trayectoria A]\tabularnewline
\tabularnewline
	2. & \includegraphics{actor} &  solicita consultar pedidos en espera mediante el botón [Consultar Pedidos]\tabularnewline
	3. & \includegraphics{sistema} & verifica que el parámetro seleccionado sea válido según lo especificado en la BRP1\tabularnewline 
	& & Pedidos en espera. [Trayectoria B]\tabularnewline
	4. & \includegraphics{sistema} & busca los registros correspondientes al parámetro de búsqueda seleccionado. \tabularnewline
	& &  [Trayectoria C][Trayectoria D]\tabularnewline
	5. & \includegraphics{sistema} & despliega una tabla mostrando los datos correspondientes de los pedidos según lo\tabularnewline
	& & especificado en la BRP2 Contenido del Pedido.\tabularnewline
	 &  & .... Fin del caso de uso\tabularnewline \\
	\end{tabular}

	\textit{Trayectoria Alternativa A}\\
	Condición: El usuario selecciona la opción ‘Pedidos para cambio’ de la lista ‘Seleccionar parámetro de búsqueda.\\
	\begin{tabular}{ccl}
	& & \tabularnewline
	A-1 & \includegraphics{actor} & Ingresa la fecha de Venta en el campo ‘Seleccionar fecha de Venta’ en la pantalla\tabularnewline
	& & PP3 Consultar pedidos en espera. [Trayectoria A.1] \tabularnewline
	A-2 & \includegraphics{sistema} & Continúa en el paso 2 de la trayectoria principal.\tabularnewline	
	 & & .... Fin del Caso de Uso\tabularnewline
	\end{tabular}

	\textit{Trayectoria Alternativa A.1}\\
	Condición: La fecha ingresada de venta no es valida.\\
	\begin{tabular}{ccl}
	& & \tabularnewline
	A.1-1 & \includegraphics{sistema} & Muestra en la pantalla PP3 Consultar pedidos en espera el MSG8 Fecha\tabularnewline
	& & Incorrecta.\tabularnewline 
	A.1-2 & \includegraphics{sistema} & Solicita verificar la fecha ingresada por el usuario.\tabularnewline	
	 & & .... Fin del Caso de Uso\tabularnewline
	\end{tabular}

	\textit{Trayectoria Alternativa B}\\
	Condición: No existen elementos en el catálogo de parámetros ‘Seleccionar parámetro de búsqueda.\\
	\begin{tabular}{ccl}
	& & \tabularnewline
	B-1 & \includegraphics{sistema} & Muestra en la pantalla PP3 Consultar pedidos en espera el MSG3 Datos no\tabularnewline
	& &  encontrados\tabularnewline
	B-2 & & .... Fin del Caso de Uso\tabularnewline
	\end{tabular}
\newpage
	\textit{Trayectoria Alternativa C}\\
	Condición: No existen registros de pedidos pendientes en espera.\\
	\begin{tabular}{ccl}
	& & \tabularnewline
	C-1 & \includegraphics{sistema} & Muestra en la pantalla PP3 Consultar pedidos en espera el MSGP3 No existen\tabularnewline
	& & pedidos pendientes.\tabularnewline
	C-2 & & .... Fin del Caso de Uso\tabularnewline
	\end{tabular}
	
	\textit{Trayectoria Alternativa D}\\
	Condición: No se pudo establecer conexión con la BD.\\
	\begin{tabular}{ccl}
	& & \tabularnewline
	D-1 & \includegraphics{sistema} & Muestra el mensaje de error MSG4 'No se pudo establecer conexión con la base\tabularnewline
	& & de datos.\tabularnewline
	D-2 & \includegraphics{sistema} & Regresa a la pantalla PP1 Menú principal de producción.\tabularnewline
	D-3 & & Fin del Caso de Uso\tabularnewline	
	\end{tabular}


% = = = = = = = = = = = = = = = = = = = = = = = = = = = = = =
% Consultar disponibilidad de Ingredientes
% = = = = = = = = = = = = = = = = = = = = = = = = = = = = = =
\newpage
		\subsection{CUP3: Consultar disponibilidad de ingredientes.}

		\textbf{\large Resumen}{\large }\\
		{Este CU tiene como objetivo permitir al usuario de producción consultar la disponibilidad de los ingredientes utilizados en la producción.}\\
		{\large \par}
		\begin{table}[!ht]
		\begin{centering}
		\begin{tabular}{|c||c|l|}
		\hline 
		\multicolumn{2}{|c|}{Caso de Uso:} & CUP3: Consultar disponibilidad de ingredientes.\tabularnewline
		\hline 
		\multicolumn{3}{|>{\columncolor[gray]{0.7}}c}{Resumen de Atributos}\tabularnewline
		\hline 
		\multicolumn{2}{|c|}{Actor} & Usuario de Producción\tabularnewline
		\hline 
		\multicolumn{2}{|c|}{Propósito} & \multicolumn{1}{p{12cm}|}{Permitir al usuario consultar la disponibilidad de un cierto ingrediente utilizado en la producción.}\tabularnewline
		\hline 
		\multicolumn{2}{|c|}{Entradas} & \multicolumn{1}{p{12cm}|}{El usuario seleccionara mediante un elemento checkbox el criterio de consulta que puede ser de 2 tipos:\newline
\begin{itemize}
\item Consulta por producto
\item Consulta por búsqueda
\end{itemize}
Al seleccionar el criterio por producto seleccionara en 1 catalogo el ingrediente deseado.\newline
Al seleccionar el criterio de búsqueda ingresara mediante el teclado el nombre del ingrediente a consultar.}\tabularnewline
		\hline 
		\multicolumn{2}{|c|}{Salidas} & \multicolumn{1}{p{12cm}|}{Se mostrara una tabla con resultados de acuerdo a la BRP3 Búsqueda de ingredientes.\newline
}\tabularnewline
		\hline 
		\multicolumn{2}{|c|}{Pre-condiciones} & \multicolumn{1}{p{12cm}|}{Debe existir un catalogo de ingredientes.\newline
		Debe existir el ingrediente a buscar.\newline
		Debe existir conexión con la BD.}\tabularnewline
		\hline 
		\multicolumn{2}{|c|}{Pos-condiciones} & \multicolumn{1}{p{12cm}|}{Se desplegara una tabla con los resultados de la consulta ajustándose a la BRP3 Búsqueda de ingredientes}\tabularnewline
		\hline 
		\multicolumn{2}{|c|}{Errores} & \multicolumn{1}{p{12cm}|}{Se mostrara el mensaje MSG3 Datos no encontrados.\newline
		MSG4 Error al obtener conexión con la base de datos}\tabularnewline
		\hline 
		\multicolumn{2}{|c|}{Tipo} & Principal\tabularnewline
		\hline 
		\multicolumn{2}{|c|}{Fuente} & \multicolumn{1}{p{12cm}|}{-}\tabularnewline
		\hline 
		\multicolumn{2}{|c|}{RN Relacionada} & \multicolumn{1}{p{12cm}|}{-}\tabularnewline
		\hline 		
		\multicolumn{2}{|c|}{Autor} & Rueda López Miguel Angel\tabularnewline
		\hline 
		\multicolumn{2}{|c|}{Fecha} & 26 de febrero de 2013\tabularnewline
		\hline 				
		\end{tabular}
		\par\end{centering}
		
	\caption{CUP3 Consultar disponibilidad de ingredientes}
	\label{tab:CasosdeUso:nombredecasodeuso} 
	\end{table}
	\newpage
	\textbf{\large Trayectorias del CU}{\large \par}
	\textit{\large Trayectoria Principal}{\large{} }{\large \par}
	\begin{tabular}{ccl}
	 &  & \tabularnewline
	1. & \includegraphics{actor} & selecciona la opción ‘Consulta por ingrediente’ del campo 'Tipo de consulta' \tabularnewline
	& &  de la pantalla PP4 Consultar disponibilidad de ingredientes. [Trayectoria A]\tabularnewline
\tabularnewline
	2. & \includegraphics{sistema} &  carga el catalogo de ingredientes de acuerdo al registro mas reciente.[Trayectoria B]\tabularnewline
	3. & \includegraphics{actor} &  selecciona un ingrediente del catalogo de ingredientes.\tabularnewline
	4. & \includegraphics{actor} & solicita consultar disponibilidad mediante el botón [Consultar disponibilidad]\tabularnewline 
	5. & \includegraphics{sistema} & busca los registros correspondientes al parámetro de búsqueda seleccionado. \tabularnewline
	6. & \includegraphics{sistema} & despliega una tabla mostrando los datos correspondientes del ingrediente según lo\tabularnewline
	& & especificado en la BRP3 Búsqueda de ingrediente. [Trayectoria C]\tabularnewline
	 &  & .... Fin del caso de uso\tabularnewline \\
	\end{tabular}

	\textit{Trayectoria Alternativa A}\\
	Condición: El usuario selecciona la opción 'Consulta por búsqueda' en la pantalla PP4 Consultar disponibilidad de ingredientes.\\
	\begin{tabular}{ccl}
	& & \tabularnewline
	A-1 & \includegraphics{actor} & Ingresa el nombre del ingrediente a buscar en el campo 'Ingresa el nombre del ingrediente'\tabularnewline
	& & [Trayectoria A.1]\tabularnewline
	A-2 & \includegraphics{sistema} & Continúa en el paso 3 de la trayectoria principal.\tabularnewline	
	 & & .... Fin del Caso de Uso\tabularnewline
	\end{tabular}
	
	\textit{Trayectoria Alternativa A.1}\\
	Condición: El nombre de producto ingresado no existe.\\
	\begin{tabular}{ccl}
	& & \tabularnewline
	A.1-1 & \includegraphics{sistema} & Muestra en la pantalla PP4 Consultar disponibilidad de ingredientes el MSG3\tabularnewline
	& &  Datos no encontrados\tabularnewline 
	A.1-2 & \includegraphics{sistema} & Solicita ingresar nuevamente el nombre del ingrediente.\tabularnewline	
	 & & .... Fin del Caso de Uso\tabularnewline
	\end{tabular}
	
	\textit{Trayectoria Alternativa B}\\
	Condición: No se pudo establecer conexión con la BD.\\
	\begin{tabular}{ccl}
	& & \tabularnewline
	B-1 & \includegraphics{sistema} & Muestra el mensaje de error MSG4 'No se pudo establecer conexión con la base\tabularnewline
	& & de datos.\tabularnewline
	B-2 & \includegraphics{sistema} & Regresa a la pantalla PP1 Menú principal de producción.\tabularnewline
	B-3 & & Fin del Caso de Uso\tabularnewline	
	\end{tabular}
	
	\textit{Trayectoria Alternativa C}\\
	Condición: No existen registros del ingrediente solicitado.\\
	\begin{tabular}{ccl}
	& & \tabularnewline
	C-1 & \includegraphics{sistema} & Muestra en la pantalla PP4 Consultar disponibilidad de ingredientes el MSG3\tabularnewline
	& &  Datos no encontrados.\tabularnewline
	C-2 & & .... Fin del Caso de Uso\tabularnewline
	\end{tabular}
	

% = = = = = = = = = = = = = = = = = = = = = = = = = = = = = =
% ASIGNAR LINEAS DE PRODUCCION
% = = = = = = = = = = = = = = = = = = = = = = = = = = = = = =
\newpage
\subsection{CUP4: Asignar Linea de Producción.}

	\textbf{\large Resumen}{\large }\\
	{Este CU tiene como objetivo asociar una linea de producción directamente con las unidades de un producto a realizar.}\\
	{\large \par}
	\begin{table}[!ht]
		\begin{centering}
		\begin{tabular}{|c||c|l|}
		\hline 
		\multicolumn{2}{|c|}{Caso de Uso:} & CUP4: Asignar Linea de Producción.\tabularnewline
		\hline 
		\multicolumn{3}{|>{\columncolor[gray]{0.7}}c}{Resumen de Atributos}\tabularnewline
		\hline 
		\multicolumn{2}{|c|}{Actor} & Usuario de Producción, Administrador\tabularnewline
		\hline 
		\multicolumn{2}{|c|}{Propósito} & \multicolumn{1}{p{12cm}|}{Asociar una linea de producción a un producto especifico.}\tabularnewline
		\hline 
		\multicolumn{2}{|c|}{Entradas} & \multicolumn{1}{p{12cm}|}{Para llevar a cabo la asignación de lineas de producción es necesario llenar un formulario que debe solicitar la siguiente información:\newline
		\begin{itemize}
		\item Fecha de elaboración
		\item Linea de Producción
		\item Producto Asociado
		\item Cantidad de producto
		\item Encargado de producción
		\end{itemize}}\tabularnewline
		\hline 
		\multicolumn{2}{|c|}{Salidas} & \multicolumn{1}{p{12cm}|}{Se mostrara el MSG1 Mensaje de confirmación.}\tabularnewline
		\hline 
		\multicolumn{2}{|c|}{Pre-condiciones} & \multicolumn{1}{p{12cm}|}{La linea de producción debe esta habilitada.\newline
		Debe haber espacio en el almacén para la producción.\newline
		El encargado a seleccionar debe estar disponible.\newline
		Debe existir conexión con la BD.}\tabularnewline
		\hline 
		\multicolumn{2}{|c|}{Pos-condiciones} & \multicolumn{1}{p{12cm}|}{Actualizar el registro de producción. \newline
		Establecer la producción en la linea especificada.}\tabularnewline
		\hline 
		\multicolumn{2}{|c|}{Errores} & \multicolumn{1}{p{12cm}|}{MSG2 Mensaje de Error\newline
		MSG4 Error al obtener conexión con la base de datos}\tabularnewline		
		\hline 
		\multicolumn{2}{|c|}{Tipo} & Primario\tabularnewline
		\hline 
		\multicolumn{2}{|c|}{Fuente} & \multicolumn{1}{p{12cm}|}{-}\tabularnewline
		\hline 
		\multicolumn{2}{|c|}{RN Relacionada} & \multicolumn{1}{p{12cm}|}{Relacionado con las BR1 y BR3}\tabularnewline
		\hline 		
		\multicolumn{2}{|c|}{Autor} & Rueda López Miguel Angel\tabularnewline
		\hline 
		\multicolumn{2}{|c|}{Fecha} & 26 de febrero de 2013\tabularnewline
		\hline 				
		\end{tabular}
		\par\end{centering}
		
	\caption{CUP4 Asignar linea de producción}
	\label{tab:CasosdeUso:nombredecasodeuso} 
	\end{table}
	
	\newpage
	\textbf{\large Trayectorias del CU}{\large \par}
	\textit{\large Trayectoria Principal}{\large{} }{\large \par}
	\begin{tabular}{ccl}
	 &  & \tabularnewline
	1. & \includegraphics{actor} & selecciona la fecha de asignación en el campo 'Ingresa la fecha de elaboración'\tabularnewline
	& &   de la pantalla PP5 Asignar Linea de Producción. \tabularnewline
	2. & \includegraphics{actor} & selecciona la linea de producción a asignar en el catalogo 'Selecciona la linea\tabularnewline 
	& &  de producción asociada' [Trayectoria A] \tabularnewline
	3. & \includegraphics{sistema} & realiza una carga del catalogo de Encargados de la linea de producción.\tabularnewline
	& & [Trayectoria B]\tabularnewline
	4. & \includegraphics{actor} & selecciona al encargado de la linea en el catalogo 'Selecciona el encargado de\tabularnewline
	& &  producción'.\tabularnewline		
	5. & \includegraphics{actor} & selecciona el producto que se va a asociar en el catalogo 'Selecciona el producto\tabularnewline
	& &  asociado' \tabularnewline
	6. & \includegraphics{sistema} & realiza una carga del catalogo de unidades de producción por producto.\tabularnewline
	& & [Trayectoria B]\tabularnewline
	7. & \includegraphics{actor} & selecciona la cantidad de unidades de producto en el catalogo 'Selecciona la cantidad\tabularnewline
	& & de producto'\tabularnewline
	8. & \includegraphics{sistema} & verifica que se hayan ingresado todos los datos solicitados en los campos requeridos.\tabularnewline
	& & [Trayectoria C]\tabularnewline
	9. & \includegraphics{sistema} & regresa a la pantalla PP1 Menú Principal de Producción y muestra el mensaje MSG1\tabularnewline
	& &  Mensaje de Confirmación. [Trayectoria D]\tabularnewline	 
	 &  & .... Fin del caso de uso\tabularnewline \\
	\end{tabular}			
	
	\textit{Trayectoria Alternativa A}\\
	Condición: No se encontraron elementos registrados en el catalogo de lineas de producción.\\
	\begin{tabular}{ccl}
	& & \tabularnewline
	A-1 & \includegraphics{sistema} & Muestra el mensaje MSG3 Datos no encontrados en la pantalla PP1 Menú  \tabularnewline
	& & principal de Producción.\tabularnewline	
	A-2 & & .... Fin del Caso de Uso\tabularnewline
	\end{tabular}	
	\newpage
	\textit{Trayectoria Alternativa B}\\
	Condición: No se pudo establecer conexión con la BD.\\
	\begin{tabular}{ccl}
	& & \tabularnewline
	B-1 & \includegraphics{sistema} & Muestra el mensaje de error MSG4 'No se pudo establecer conexión con la base\tabularnewline
	& & de datos.\tabularnewline
	B-2 & \includegraphics{sistema} & Regresa a la pantalla PP1 Menú principal de producción.\tabularnewline
	B-3 & & Fin del Caso de Uso\tabularnewline	
	\end{tabular}	

	
	\textit{Trayectoria Alternativa C}\\
	Condición: El usuario no selecciono una opción en un campo solicitado.\\
	\begin{tabular}{ccl}
	& & \tabularnewline
	C-1 & \includegraphics{sistema} & muestra el mensaje MSGP4 Campo Requerido.\tabularnewline	
	C-2 & \includegraphics{sistema} & solicita ingresar el o los campo(s) faltante(s).\tabularnewline		
	C-3 & & continua en el paso 2 de la trayectoria principal\tabularnewline
	\end{tabular}	
	
	\textit{Trayectoria Alternativa D}\\
	Condición: No se pudo asignar la linea de producción.\\
	\begin{tabular}{ccl}
	& & \tabularnewline
	D-1 & \includegraphics{sistema} & muestra el mensaje MSG2 Mensaje de Error.\tabularnewline	
	D-2 & \includegraphics{sistema} & regresa a la pantalla PP1 Menú principal de Producción.\tabularnewline		
	D-3 & & .... Fin del Caso de Uso\tabularnewline
	\end{tabular}				
% = = = = = = = = = = = = = = = = = = = = = = = = = = = = = =
% GESTIONAR LOTES 5
% = = = = = = = = = = = = = = = = = = = = = = = = = = = = = 
\newpage
\subsection{CUP5: Gestionar Lotes}

	\textbf{\large Resumen}{\large }\\
	{Este CU tiene como objetivo permitir al usuario de producción o administrador mostrar un menú con opciones para consultar o registrar un lote.}\\
	{\large \par}
	\begin{table}[!ht]
		\begin{centering}
		\begin{tabular}{|c||c|l|}
		\hline 
		\multicolumn{2}{|c|}{Caso de Uso:} & CUP5: Gestionar Lotes.\tabularnewline
		\hline 
		\multicolumn{3}{|>{\columncolor[gray]{0.7}}c}{Resumen de Atributos}\tabularnewline
		\hline 
		\multicolumn{2}{|c|}{Actor} & Usuario de Producción, Administrador\tabularnewline
		\hline 
		\multicolumn{2}{|c|}{Propósito} & \multicolumn{1}{p{12cm}|}{Mostrar al usuario un menú para consultar o registrar un lote.}\tabularnewline
		\hline 
		\multicolumn{2}{|c|}{Entradas} & \multicolumn{1}{p{12cm}|}{Ninguna}\tabularnewline
		\hline 
		\multicolumn{2}{|c|}{Salidas} & \multicolumn{1}{p{12cm}|}{Ninguna}\tabularnewline
		\hline 
		\multicolumn{2}{|c|}{Pre-condiciones} & \multicolumn{1}{p{12cm}|}{Ninguna}
\tabularnewline
		\hline 
		\multicolumn{2}{|c|}{Pos-condiciones} & \multicolumn{1}{p{12cm}|}{Realizar cambios en el registro de lotes}\tabularnewline
		\hline 
		\multicolumn{2}{|c|}{Errores} & \multicolumn{1}{p{12cm}|}{-}\tabularnewline
		\hline 
		\multicolumn{2}{|c|}{Tipo} & Primario\tabularnewline
		\hline 
		\multicolumn{2}{|c|}{Fuente} & \multicolumn{1}{p{12cm}|}{-}\tabularnewline
		\hline 
		\multicolumn{2}{|c|}{RN Relacionada} & \multicolumn{1}{p{12cm}|}{Relacionado con las BR1, BR3 y BRP4}\tabularnewline
		\hline 		
		\multicolumn{2}{|c|}{Autor} & Rueda López Miguel Angel\tabularnewline
		\hline 
		\multicolumn{2}{|c|}{Fecha} & 26 de febrero de 2013\tabularnewline
		\hline 				
		\end{tabular}
		\par\end{centering}
		
	\caption{CUP5 Gestionar Lotes}
	\label{tab:CasosdeUso:nombredecasodeuso} 
	\end{table}
	\textbf{\large Trayectorias del CU}{\large \par}
	\textit{\large Trayectoria Principal}{\large{} }{\large \par}
	\begin{tabular}{ccl}
	 &  & \tabularnewline
	1. & \includegraphics{sistema} & Muestra la pantalla PP6 Gestionar Lotes \tabularnewline
	 &  & .... Fin del caso de uso\tabularnewline \\
	\end{tabular}		
	\newpage
% = = = = = = = = = = = = = = = = = = = = = = = = = = = = = =
% REGISTRAR LOTE 5.1
% = = = = = = = = = = = = = = = = = = = = = = = = = = = = = =
	\subsection{CUP5.1: Registrar Lote.}

	\textbf{\large Resumen}{\large }\\
	{Este CU tiene como objetivo permitir al usuario de producción o administrador registrar la producción de un nuevo lote.}\\
	{\large \par}
	\begin{table}[!ht]
		\begin{centering}
		\begin{tabular}{|c||c|l|}
		\hline 
		\multicolumn{2}{|c|}{Caso de Uso:} & CUP5.1: Registrar Lote.\tabularnewline
		\hline 
		\multicolumn{3}{|>{\columncolor[gray]{0.7}}c}{Resumen de Atributos}\tabularnewline
		\hline 
		\multicolumn{2}{|c|}{Actor} & Usuario de Producción, Administrador\tabularnewline
		\hline 
		\multicolumn{2}{|c|}{Propósito} & \multicolumn{1}{p{12cm}|}{Permitir al usuario registrar un nuevo lote.}\tabularnewline
		\hline 
		\multicolumn{2}{|c|}{Entradas} & \multicolumn{1}{p{12cm}|}{Para llevar a cabo el registro de un nuevo lote el usuario deberá ingresar la siguiente información:\newline
		\begin{itemize}
		\item Producto asociado al Lote
		\item Fecha de elaboración del Lote
		\item Linea de producción asociada
		\end{itemize}
		}\tabularnewline
		\hline 
		\multicolumn{2}{|c|}{Salidas} & \multicolumn{1}{p{12cm}|}{Se mostrara el MSG1 Mensaje de Confirmación cuando se haya completado el registro.
}\tabularnewline
		\hline 
		\multicolumn{2}{|c|}{Pre-condiciones} & \multicolumn{1}{p{12cm}|}{Debe haber terminado completamente la producción del lote.}
\tabularnewline
		\hline 
		\multicolumn{2}{|c|}{Pos-condiciones} & \multicolumn{1}{p{12cm}|}{Se agregara un nuevo registro con la información del lote creado.\newline
		Se mostrara el MSG1 Mensaje de Confirmación.\newline
		Se generara un numero identificador único para el lote.\newline
		Se actualizara la información de los lotes.}\tabularnewline
		\hline 
		\multicolumn{2}{|c|}{Errores} & \multicolumn{1}{p{12cm}|}{Se mostrara el mensaje MSG2 Mensaje de Error.}\tabularnewline
		\hline 
		\multicolumn{2}{|c|}{Tipo} & Secundario\tabularnewline
		\hline 
		\multicolumn{2}{|c|}{Fuente} & \multicolumn{1}{p{12cm}|}{Basado en el funcionamiento del CUP5 Gestionar Lotes.}\tabularnewline
		\hline 
		\multicolumn{2}{|c|}{RN Relacionada} & \multicolumn{1}{p{12cm}|}{Relacionado con las BR1, BR3 y BRP4}\tabularnewline
		\hline 		
		\multicolumn{2}{|c|}{Autor} & Rueda López Miguel Angel\tabularnewline
		\hline 
		\multicolumn{2}{|c|}{Fecha} & 26 de febrero de 2013\tabularnewline
		\hline 				
		\end{tabular}
		\par\end{centering}
		
	\caption{CUP5.1 Registrar Lote}
	\label{tab:CasosdeUso:nombredecasodeuso} 
	\end{table}
	\newpage
	\textbf{\large Trayectorias del CU}{\large \par}
	\textit{\large Trayectoria Principal}{\large{} }{\large \par}
	\begin{tabular}{ccl}
	 &  & \tabularnewline
	1. & \includegraphics{actor} & selecciona el producto asociado al lote en el catalogo 'Selecciona el producto asociado'\tabularnewline
	& &   de la pantalla PP7 Registrar Lote [Trayectoria A] \tabularnewline
	2. & \includegraphics{actor} & ingresa la fecha mediante el campo 'Ingresa la fecha de elaboración'.\tabularnewline
	3. & \includegraphics{actor} & selecciona la linea de producción en el catalogo 'Selecciona la linea de producción asociada' \tabularnewline
	4. & \includegraphics{actor} & solicita registrar el lote mediante el botón [Registrar Lote]\tabularnewline
	5. & \includegraphics{sistema} & verifica que se hayan ingresado todos los datos solicitados en los campos requeridos.\tabularnewline
	& & [Trayectoria B]\tabularnewline
	6. & \includegraphics{sistema} & regresa a la pantalla PP6 Gestionar Lotes y muestra el mensaje MSG1 Mensaje de \tabularnewline
	& & Confirmación. [Trayectoria C]\tabularnewline
	 &  & .... Fin del caso de uso\tabularnewline \\
	\end{tabular}		
	
	\textit{Trayectoria Alternativa A}\\
	Condición: No se encontraron elementos registrados en el catalogo de productos.\\
	\begin{tabular}{ccl}
	& & \tabularnewline
	A-1 & \includegraphics{sistema} & Muestra el mensaje MSG3 Datos no encontrados en la pantalla PP6 Gestionar Lotes\tabularnewline	
	A-2 & & .... Fin del Caso de Uso\tabularnewline
	\end{tabular}	
	
	\textit{Trayectoria Alternativa B}\\
	Condición: El usuario no selecciono una opción en un campo solicitado.\\
	\begin{tabular}{ccl}
	& & \tabularnewline
	B-1 & \includegraphics{sistema} & muestra el mensaje MSGP4 Campo Requerido.\tabularnewline	
	B-2 & \includegraphics{sistema} & solicita ingresar el o los campo(s) faltante(s).\tabularnewline		
	B-3 & & continua en el paso 1 de la trayectoria principal\tabularnewline
	\end{tabular}	
	
	\textit{Trayectoria Alternativa C}\\
	Condición: No se pudo registrar el nuevo lote.\\
	\begin{tabular}{ccl}
	& & \tabularnewline
	C-1 & \includegraphics{sistema} & muestra el mensaje MSG2 Mensaje de Error.\tabularnewline	
	C-2 & \includegraphics{sistema} & regresa a la pantalla PP6 Gestionar Lotes.\tabularnewline		
	C-3 & & .... Fin del Caso de Uso\tabularnewline
	\end{tabular}			
% = = = = = = = = = = = = = = = = = = = = = = = = = = = = = =
% CONSULTAR LOTE 5.2
% = = = = = = = = = = = = = = = = = = = = = = = = = = = = = =
	\newpage
	\subsection{CUP5.2: Consultar Lote.}

	\textbf{\large Resumen}{\large }\\
	{Este CU tiene como objetivo permitir al usuario de producción o administrador consultar la información relacionada con un determinado lote.}\\
	{\large \par}
	\begin{table}[!ht]
		\begin{centering}
		\begin{tabular}{|c||c|l|}
		\hline 
		\multicolumn{2}{|c|}{Caso de Uso:} & CUP5.2: Consultar Lote.\tabularnewline
		\hline 
		\multicolumn{3}{|>{\columncolor[gray]{0.7}}c}{Resumen de Atributos}\tabularnewline
		\hline 
		\multicolumn{2}{|c|}{Actor} & Usuario de Producción, Administrador\tabularnewline
		\hline 
		\multicolumn{2}{|c|}{Propósito} & \multicolumn{1}{p{12cm}|}{Permitir al usuario consultar la información de un lote a través del identificador único de cada lote según se especifica en la BRP4 Información de lote.}\tabularnewline
		\hline 
		\multicolumn{2}{|c|}{Entradas} & \multicolumn{1}{p{12cm}|}{El usuario ingresara en un campo de texto el identificador asociado al lote del cual se desea conocer información.}\tabularnewline
		\hline 
		\multicolumn{2}{|c|}{Salidas} & \multicolumn{1}{p{12cm}|}{Se mostrara una tabla con resultados de acuerdo a la BRP4 Información de lote.
}\tabularnewline
		\hline 
		\multicolumn{2}{|c|}{Pre-condiciones} & \multicolumn{1}{p{12cm}|}{Debe haber sido registrado previamente el lote.\newline
		Debe existir conexión con la BD.}\tabularnewline		
		\hline 
		\multicolumn{2}{|c|}{Pos-condiciones} & \multicolumn{1}{p{12cm}|}{Se desplegara una tabla con los resultados de la consulta ajustándose a la BRP4 Información de lote}\tabularnewline
		\hline 
		\multicolumn{2}{|c|}{Errores} & \multicolumn{1}{p{12cm}|}{Se mostrara el mensaje MSG3 Datos no encontrados.\newline
		MSG4 Error al obtener conexión con la base de datos}\tabularnewline		
		\hline 
		\multicolumn{2}{|c|}{Tipo} & Secundario\tabularnewline
		\hline 
		\multicolumn{2}{|c|}{Fuente} & \multicolumn{1}{p{12cm}|}{Basado en el funcionamiento del CUP5 Gestionar Lotes.}\tabularnewline
		\hline 
		\multicolumn{2}{|c|}{RN Relacionada} & \multicolumn{1}{p{12cm}|}{Relacionado con las BR1, BR3 y BRP4}\tabularnewline
		\hline 		
		\multicolumn{2}{|c|}{Autor} & Rueda López Miguel Angel\tabularnewline
		\hline 
		\multicolumn{2}{|c|}{Fecha} & 26 de febrero de 2013\tabularnewline
		\hline 				
		\end{tabular}
		\par\end{centering}
		
	\caption{CUP5.2 Consultar Lote}
	\label{tab:CasosdeUso:nombredecasodeuso} 
	\end{table}
	
	\textbf{\large Trayectorias del CU}{\large \par}
	\textit{\large Trayectoria Principal}{\large{} }{\large \par}
	\begin{tabular}{ccl}
	 &  & \tabularnewline
	1. & \includegraphics{actor} & ingresa el numero de lote en el campo 'Ingresa el \# de lote' de la pantalla PP8\tabularnewline
	& &  Consultar Lote \tabularnewline
	2. & \includegraphics{actor} & selecciona consultar lote mediante el botón [Consultar Lote]\tabularnewline 
	3. & \includegraphics{sistema} & busca los registros correspondientes al parámetro de búsqueda ingresado. \tabularnewline
	& & [Trayectoria A][Trayectoria B]\tabularnewline
	4. & \includegraphics{sistema} & despliega una tabla mostrando los datos correspondientes del lote solicitado según lo\tabularnewline
	& & especificado en la BRP4 Información de lote. [Trayectoria C]\tabularnewline
	 &  & .... Fin del caso de uso\tabularnewline \\
	\end{tabular}	
	\newpage
	\textit{Trayectoria Alternativa A}\\
	Condición: El numero de lote ingresado no cumple con el formato especificado en la BRP4 Información de lote.\\
	\begin{tabular}{ccl}
	& & \tabularnewline
	A-1 & \includegraphics{sistema} & Muestra el mensaje MSG7 Formato incorrecto de los datos\tabularnewline	
	A-2 & & .... Fin del Caso de Uso\tabularnewline
	\end{tabular}	

	\textit{Trayectoria Alternativa B}\\
	Condición: No se pudo establecer conexión con la BD.\\
	\begin{tabular}{ccl}
	& & \tabularnewline
	B-1 & \includegraphics{sistema} & Muestra el mensaje de error MSG4 'No se pudo establecer conexión con la base\tabularnewline
	& & de datos.\tabularnewline
	B-2 & \includegraphics{sistema} & Regresa a la pantalla PP1 Menú principal de producción.\tabularnewline
	B-3 & & Fin del Caso de Uso\tabularnewline	
	\end{tabular}	
	
	\textit{Trayectoria Alternativa C}\\
	Condición: El numero de lote no coincide con ningún otro registrado.\\
	\begin{tabular}{ccl}
	& & \tabularnewline
	C-1 & \includegraphics{sistema} & Muestra el mensaje MSG3 Datos no encontrados\tabularnewline	
	C-2 & & .... Fin del Caso de Uso\tabularnewline
	\end{tabular}	

% = = = = = = = = = = = = = = = = = = = = = = = = = = = = = =
% CUP6. Crear Reporte
% = = = = = = = = = = = = = = = = = = = = = = = = = = = = = =
\newpage
		\subsection{CUP6: Crear Reporte de Producción.}

		\textbf{\large Resumen}{\large }\\
		{Este caso de uso tiene como objetivo permitir al usuario del área de Producción generar reportes personalizados de los Lotes producidos  y de la Materia Prima utilizada en la elaboración de los mismos. La personalización incluye la posibilidad de indicar el lapso de tiempo que abarcará el reporte y la forma en que se ordenará la información dentro del mismo. }\\
		{\large \par}

		% = = = = = = = = = = = = = = = = = = = = = = = = =
		%	INICIO DE TABLA
		% = = = = = = = = = = = = = = = = = = = = = = = = = 
		\begin{table}[!ht]
		\begin{centering}
		\begin{tabular}{|c||c|l|}
		\hline 
		\multicolumn{2}{|c|}{Caso de Uso:} & CUP6 Crear Reporte de Producción   \tabularnewline
		\hline 
		\multicolumn{3}{|>{\columncolor[gray]{0.7}}c}{Resumen de Atributos}\tabularnewline
		\hline 
		\multicolumn{2}{|c|}{Actor} & Usuario de Producción\tabularnewline
		\hline 
		\multicolumn{2}{|c|}{Propósito} & \multicolumn{1}{p{12cm}|}{Permitir al usuario crear reportes personalizados referentes al área de Producción}\tabularnewline
		\hline 
		\multicolumn{2}{|c|}{Entradas} & \multicolumn{1}{p{12cm}|}{
		Para generar un reporte, el usuario deberá:\newline		
	- Seleccionar con el mouse el tipo de reporte\newline
	- Seleccionar con el mouse por medio de un Datepicker la fecha desde la cual empezará el 		 reporte\newline
	- Seleccionar con el mouse por medio de un Datepicker la fecha hasta la cual abarcará el reporte\newline
	- Seleccionar con el mouse la forma en que se ordenarán los datos del reporte
}\tabularnewline
		\hline 
		\multicolumn{2}{|c|}{Salidas} & \multicolumn{1}{p{12cm}|}{

    - Se mostrará en pantalla el mensaje MSG1: Mensaje de Confirmación.\newline
	- Además, se mostrará el reporte correspondiente en formato PDF según la estructura
	especificada en la regla de negocio BR5: Formato de Reportes.
}\tabularnewline
\hline 
		\multicolumn{2}{|c|}{Pre-condiciones} & \multicolumn{1}{p{12cm}|}{

    - Deben existir lotes registrados en la base de datos.\newline
	- El usuario debe haber iniciado sesión previamente como usuario de Producción.
}\tabularnewline				
		\hline 
		\multicolumn{2}{|c|}{Post-condiciones} & Se limpia el formulario para permitir la generación de un nuevo reporte.\tabularnewline
\hline 
		\multicolumn{2}{|c|}{Errores} & \multicolumn{1}{p{12cm}|}{

    - El sistema muestra en pantalla los mensajes de error MSG11 o MSG12 referentes a errores
    en las fechas cuando el usuario ingrese datos erróneos en los campos de fecha.\newline
	- El sistema muestra en pantalla el mensaje de error MSG3: Datos no encontrados, cuando al acceder a la base de datos, ésta se encuentre vacía o no contenga los datos requeridos por el usuario.			
}\tabularnewline						
		
		\hline 		
		\multicolumn{2}{|c|}{Autor} & Durán Pineda Mario Ángel\tabularnewline
		\hline 
		\end{tabular}
		\par\end{centering}
		
	
	\label{tab:CasosdeUso:nombredecasodeuso} 
	\end{table}


	% = = = = = = = = = = = = = = = = = = = = = = = = =
	%	FIN DE TABLA
	% = = = = = = = = = = = = = = = = = = = = = = = = =
	% = = = = = = = = = = = = = = = = = = = = = = = = =
	%	INICIO DE TRAYECTORIA
	% = = = = = = = = = = = = = = = = = = = = = = = = = 
	\newpage
	\textbf{\large Trayectorias del CU\\}{\large \par}
	\textit{\large Trayectoria Principal}{\large{} }{\large \par}
	\begin{tabular}{ccl}
	 &  & \tabularnewline
	1. & \includegraphics{actor} & El usuario selecciona la opción Crear Reporte dando clic al botón [Crear Reporte] \tabularnewline
	& &  de la pantalla PP1: Menú Principal de Producción.\tabularnewline
\tabularnewline

	2. & \includegraphics{sistema} &  El sistema despliega la pantalla PP9: Crear Reporte de Producción.\tabularnewline
	
	3. & \includegraphics{actor} & El usuario selecciona la opción de reporte 'Reporte de Lotes'. [Trayectoria A] \tabularnewline 

	4. & \includegraphics{sistema} & El sistema carga las opciones con las que se puede ordenar un 'Reporte de Lotes', \tabularnewline
	& & como se muestra en la pantalla PP9.1: Opciones del Reporte de Lotes \tabularnewline
	
	5. & \includegraphics{actor} & El usuario ingresa la fecha a partir de la cual abarcará el reporte y la fecha de término \tabularnewline
	&  & del mismo.\tabularnewline
	
	6. & \includegraphics{actor} & El usuario selecciona la forma en que estará ordenado su reporte (Por Fecha de \tabularnewline
	& & producción, Línea de Producción o Producto). \tabularnewline
	
	7. & \includegraphics{actor} & El usuario da clic en el botón [Generar Reporte]. \tabularnewline 
	
	8. & \includegraphics{sistema} & El sistema valida los datos ingresados por el usuario. [Trayectoria B] \tabularnewline
	
	9. & \includegraphics{sistema} & El sistema genera el reporte correspondiente y lo muestra en pantalla al usuario. \tabularnewline
	
	
	\tabularnewline
	 &  & - - - - Fin del caso de uso\tabularnewline \\
	\end{tabular}

\newpage
	\textit{\large \\Trayectoria Alternativa A: Reporte de Materia Prima.}
	
	\begin{tabular}{ccl}
	& & \tabularnewline
	
	A.1 & \includegraphics{actor} & El usuario selecciona la opción de reporte 'Reporte de Materia Prima'.\tabularnewline
		
	A.2 & \includegraphics{sistema} & El sistema carga las opciones con las que se puede ordenar un 'Reporte de Materia Prima', \tabularnewline
	& & como se muestra en la pantalla PP9.2: Opciones del Reporte de Materia Prima. \tabularnewline
	
	A.3 & \includegraphics{actor} & El usuario ingresa la fecha a partir de la cual abarcará el reporte y la fecha de \tabularnewline
& &	término del mismo.\tabularnewline

    A.4 & \includegraphics{actor} & El usuario selecciona la forma en que estará ordenado su reporte (Por Materia Prima,  	\tabularnewline
	& &  por Proveedor o por Línea de Producción).\tabularnewline\tabularnewline
	
	A.5 & & Regresar a la trayectoria principal en el punto 7.\tabularnewline

	
	\tabularnewline\tabularnewline	
	\end{tabular}

\textit{\large \\Trayectoria Alternativa B: Datos incorrectos.}

\begin{tabular}{ccl}
	& & \tabularnewline
	
	A.1 & \includegraphics{sistema} & El sistema detecta incongruencias en las fechas indicadas para el reporte\tabularnewline
		
	A.2 & \includegraphics{sistema} & El sistema despliega en pantalla el mensaje MSG11 o MSG12 relacionados \tabularnewline
	& & con los errores de fechas según el error encontrado. 	\tabularnewline	
	
	A.3 & \includegraphics{actor} & El usuario ingresa nuevamente la fecha a partir de la cual abarcará el reporte   \tabularnewline
& &	y la fecha de término del mismo.\tabularnewline    \tabularnewline    
	
	A.4 & & Regresar a la trayectoria principal en el punto 7.\tabularnewline

	
	\tabularnewline\tabularnewline	
	\end{tabular}

	% = = = = = = = = = = = = = = = = = = = = = = = = =
	%	FIN DEL DOCUMENTO
	% = = = = = = = = = = = = = = = = = = = = = = = = =

\end{document}
