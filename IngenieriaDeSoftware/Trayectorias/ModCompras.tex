%% LyX 2.0.3 created this file.  For more info, see http://www.lyx.org/.
%% Do not edit unless you really know what you are doing.
\documentclass[10pt,spanish]{article}
\usepackage[utf8x]{inputenc}
\usepackage[letterpaper]{geometry}
\geometry{verbose}
\usepackage{amsmath}
\usepackage{amssymb}
\usepackage{graphicx}

\makeatletter

%%%%%%%%%%%%%%%%%%%%%%%%%%%%%% LyX specific LaTeX commands.
%% Because html converters don't know tabularnewline
\providecommand{\tabularnewline}{\\}

%%%%%%%%%%%%%%%%%%%%%%%%%%%%%% User specified LaTeX commands.

\usepackage{ucs}\usepackage[spanish]{babel}
\usepackage{amsfonts}\usepackage{colortbl}% = = = = = = = = = = = = = = = = = = = = = = = = =
% INICIO EL DOCUMENTO
% = = = = = = = = = = = = = = = = = = = = = = = = =



\usepackage{babel}
\addto\shorthandsspanish{\spanishdeactivate{~<>}}





\usepackage{babel}
\addto\shorthandsspanish{\spanishdeactivate{~<>}}



\usepackage{babel}
\addto\shorthandsspanish{\spanishdeactivate{~<>}}





\usepackage{babel}
\addto\shorthandsspanish{\spanishdeactivate{~<>}}





\usepackage{babel}
\addto\shorthandsspanish{\spanishdeactivate{~<>}}





\usepackage{babel}
\addto\shorthandsspanish{\spanishdeactivate{~<>}}





\usepackage{babel}
\addto\shorthandsspanish{\spanishdeactivate{~<>}}





\usepackage{babel}
\addto\shorthandsspanish{\spanishdeactivate{~<>}}





\usepackage{babel}
\addto\shorthandsspanish{\spanishdeactivate{~<>}}



\usepackage{babel}
\addto\shorthandsspanish{\spanishdeactivate{~<>}}





\usepackage{babel}
\addto\shorthandsspanish{\spanishdeactivate{~<>}}





\usepackage{babel}
\addto\shorthandsspanish{\spanishdeactivate{~<>}}





\usepackage{babel}
\addto\shorthandsspanish{\spanishdeactivate{~<>}}





\usepackage{babel}
\addto\shorthandsspanish{\spanishdeactivate{~<>}}





\usepackage{babel}
\addto\shorthandsspanish{\spanishdeactivate{~<>}}





\usepackage{babel}
\addto\shorthandsspanish{\spanishdeactivate{~<>}}



\usepackage{babel}
\addto\shorthandsspanish{\spanishdeactivate{~<>}}

\makeatother

\usepackage{babel}
\addto\shorthandsspanish{\spanishdeactivate{~<>}}

\begin{document}
\tableofcontents{}

\pagebreak{}


\section{Módulo Compras}

\subsection{CUC1.0: Gestionar Materia Prima}

\textbf{\large Resumen}\\
 {\large {} Este Caso de uso tiene como objetivo realizar la compra
y el ingreso de la Materia Prima}\\


% = = = = = = = = = = = = = = = = = = = = = = = = =
%    INICIO DE TABLA
% = = = = = = = = = = = = = = = = = = = = = = = = = 
% Se utiliza para que no se salga la tabla de la hoja, porque sino se autodimensiona
%\multicolumn{1}{p{10cm}||}{Contenido}
\begin{table}[!ht]
\begin{centering}
\begin{tabular}{|c||c|l|}
\hline 
\multicolumn{2}{|c|}{Caso de Uso:} & CUC1.0: Gestionar Materia Prima\tabularnewline
\hline 
\multicolumn{3}{|>{\columncolor[gray]{0.7}}c}{Resumen de Atributos}\tabularnewline
\hline 
\multicolumn{2}{|c|}{Autor} & \multicolumn{1}{p{10cm}||}{Ventura Cruz Eduardo}\tabularnewline
\hline 
\multicolumn{2}{|c|}{Actor} & \multicolumn{1}{p{10cm}||}{Administrador}\tabularnewline
\hline 
\multicolumn{2}{|c|}{Propósito} & \multicolumn{1}{p{10cm}||}{Acceder a opciones de adquirir o ingresar Materia Prima. }\tabularnewline
\hline 
\multicolumn{2}{|c|}{Entradas} & \multicolumn{1}{p{10cm}||}{Ninguna}\tabularnewline
\hline 
\multicolumn{2}{|c|}{Salidas} & \multicolumn{1}{p{10cm}||}{Interfaz Correspondiente a gestionar Materia Prima}\tabularnewline
\hline 
\multicolumn{2}{|c|}{Pre-condiciones} & \multicolumn{1}{p{10cm}||}{El usuario de compras debe estar autentificado}\tabularnewline
\hline 
\multicolumn{2}{|c|}{Pos-condiciones} & \multicolumn{1}{p{10cm}||}{La base de datos debe estar disponible}\tabularnewline
\hline 
\multicolumn{2}{|c|}{Errores} & \multicolumn{1}{p{10cm}||}{MSG4 Error al conectar con la base de datos.}\tabularnewline
\hline 
\multicolumn{2}{|c|}{Tipo} & \multicolumn{1}{p{10cm}||}{Primario}\tabularnewline
\hline 
\multicolumn{2}{|c|}{Fuente} & \multicolumn{1}{p{10cm}||}{Basado en el funcionamiento etc.}\tabularnewline
\hline 
\end{tabular}
\par\end{centering}

\caption{Caso de Uso 1.0: Gestionar Materia Prima}


\label{tab:CasosdeUso:nombredecasodeuso} 
\end{table}


% = = = = = = = = = = = = = = = = = = = = = = = = =
%    FIN DE TABLA
% = = = = = = = = = = = = = = = = = = = = = = = = =
% = = = = = = = = = = = = = = = = = = = = = = = = =
%	INICIO DE TRAYECTORIA
% = = = = = = = = = = = = = = = = = = = = = = = = = 


\textbf{\large Trayectorias del CU}{\large \par}

\textit{\large Trayectoria Principal}{\large {} }{\large \par}

%Solo hay que cambiar el nombre de la imagen dependiendo de si es actor o sistema


\begin{tabular}{ccl}
 &  & \tabularnewline
1.  & \includegraphics{actor}  & \multicolumn{1}{p{12cm}}{ Presiona el botón {[}Gestión de Materia Prima{]} en la pantalla PC1}\tabularnewline
2.  & \includegraphics{sistema}  & \multicolumn{1}{p{12cm}}{Carga los datos de las compras que estan registradas en la Base de
datos. {[}Trayectoria Alternativa A{]}}\tabularnewline
3.  & \includegraphics{sistema}  & \multicolumn{1}{p{12cm}}{Despliega la pantalla PC2}\tabularnewline
 &  & \multicolumn{1}{p{12cm}}{.... Fin del caso de uso}\tabularnewline
\end{tabular}

\textit{Trayectoria Alternativa A}

Condición: No se pudieron cargar los datos de las Compras.

\begin{tabular}{ccl}
 &  & \tabularnewline
1.  & \includegraphics{sistema}  & \multicolumn{1}{p{12cm}}{ Muestra el mensaje de error MSG4 'No se pudo establecer conexión
con la base de datos'}\tabularnewline
2.  & \includegraphics{sistema}  & \multicolumn{1}{p{12cm}}{Regresa a la pantalla PC1}\tabularnewline
 &  & \multicolumn{1}{p{12cm}}{.... Fin del caso de uso}\tabularnewline
\end{tabular}


\subsubsection{CUC1.1:Realizar Compra}

\textbf{\large Resumen}\\
 {\large {} Este Caso de uso tiene como objetivo permitir al usuario
dar por finalizada la compra e ingresar la Materia Prima comprada.}\\


% = = = = = = = = = = = = = = = = = = = = = = = = =
%    INICIO DE TRAYECTORIA
% = = = = = = = = = = = = = = = = = = = = = = = = = 


\begin{table}[!ht]
\begin{centering}
\begin{tabular}{|c||c|l|}
\hline 
\multicolumn{2}{|c|}{Caso de Uso:} & CUC1.1: Realizar Compra\tabularnewline
\hline 
\multicolumn{3}{|>{\columncolor[gray]{0.7}}c}{Resumen de Atributos}\tabularnewline
\hline 
\multicolumn{2}{|c|}{Autor} & \multicolumn{1}{p{10cm}||}{Ventura Cruz Eduardo}\tabularnewline
\hline 
\multicolumn{2}{|c|}{Actor} & \multicolumn{1}{p{10cm}||}{Administrador}\tabularnewline
\hline 
\multicolumn{2}{|c|}{Propósito} & \multicolumn{1}{p{10cm}||}{Registrar una compra en el sistema}\tabularnewline
\hline 
\multicolumn{2}{|c|}{Entradas} & \multicolumn{1}{p{10cm}||}{Datos del formulario Realizar Compra}\tabularnewline
\hline 
\multicolumn{2}{|c|}{Salidas} & \multicolumn{1}{p{10cm}||}{Ninguna}\tabularnewline
\hline 
\multicolumn{2}{|c|}{Pre-condiciones} & \multicolumn{1}{p{10cm}||}{El usuario de compras debe estar autentificado.}\tabularnewline
\hline 
\multicolumn{2}{|c|}{Pos-condiciones} & \multicolumn{1}{p{10cm}||}{Ninguna}\tabularnewline
\hline 
\multicolumn{2}{|c|}{Errores} & \multicolumn{1}{p{10cm}||}{MSG2 Mensaje de error, MSG4 Error al conectar con la base de datos,
MSG7 Formato incorrecto de los datos }\tabularnewline
\hline 
\multicolumn{2}{|c|}{Tipo} & \multicolumn{1}{p{10cm}||}{Secundario}\tabularnewline
\hline 
\multicolumn{2}{|c|}{Fuente} & \multicolumn{1}{p{10cm}||}{Basado en el funcionamiento etc.}\tabularnewline
\hline 
\end{tabular}
\par\end{centering}

\caption{Caso de Uso 1.1: Realizar Compra}


\label{tab:CasosdeUso:nombredecasodeuso} 
\end{table}


% = = = = = = = = = = = = = = = = = = = = = = = = =
%	FIN DE TABLA
% = = = = = = = = = = = = = = = = = = = = = = = = =
% = = = = = = = = = = = = = = = = = = = = = = = = =
%	INICIO DE TRAYECTORIA
% = = = = = = = = = = = = = = = = = = = = = = = = = 


\textbf{\large Trayectorias del CU}{\large \par}

\textit{\large Trayectoria Principal}{\large {} }{\large \par}

%Solo hay que cambiar el nombre de la imagen dependiendo de si es actor o sistema


\begin{tabular}{ccl}
 &  & \tabularnewline
1.  & \includegraphics{actor}  & \multicolumn{1}{p{12cm}}{ Presiona el ícono {[}Realizar Compra{]} en la pantalla PC2}\tabularnewline
2.  & \includegraphics{sistema}  & \multicolumn{1}{p{12cm}}{Despliega la pantalla PC3 con el formulario a llenar.}\tabularnewline
3.  & \includegraphics{actor}  & \multicolumn{1}{p{12cm}}{Llena los datos del formulario}\tabularnewline
4.  & \includegraphics{actor}  & \multicolumn{1}{p{12cm}}{Da clic en el botón de Finalizar Compra{[}Trayectoria Alternativa
B{]}}\tabularnewline
5.  & \includegraphics{sistema}  & \multicolumn{1}{p{12cm}}{Verifica si los campos se llenaron correctamente{[}Trayectoria Alternativa
C{]}{[}Trayectoria Alternativa D{]}}\tabularnewline
6.  & \includegraphics{sistema}  & \multicolumn{1}{p{12cm}}{}\tabularnewline
 &  & \multicolumn{1}{p{12cm}}{.... Fin del caso de uso}\tabularnewline
\end{tabular}

\textit{Trayectoria Alternativa B}

Condición: Si el usuario da clic en el botón de cancelar.

\begin{tabular}{ccl}
 &  & \tabularnewline
1.  & \includegraphics{sistema}  & \multicolumn{1}{p{12cm}}{ Regresa a la pantalla PC2}\tabularnewline
 &  & \multicolumn{1}{p{12cm}}{.... Fin del caso de uso}\tabularnewline
\end{tabular}

\textit{Trayectoria Alternativa C}

Condición: Si la base de datos no esta disponible.

\begin{tabular}{ccl}
 &  & \tabularnewline
1.  & \includegraphics{sistema}  & \multicolumn{1}{p{12cm}}{ Muestra MSG4 Error al conectar con la base de datos}\tabularnewline
2.  & \includegraphics{sistema}  & \multicolumn{1}{p{12cm}}{Regresa a la pantalla PC2}\tabularnewline
 &  & \multicolumn{1}{p{12cm}}{.... Fin del caso de uso}\tabularnewline
\end{tabular}

\textit{Trayectoria Alternativa D}

Condición: Si el formato de los datos es incorrecto.

\begin{tabular}{ccl}
 &  & \tabularnewline
1.  & \includegraphics{sistema}  & \multicolumn{1}{p{12cm}}{ Muestra MSG7 Formato incorrecto de los datos}\tabularnewline
2.  & \includegraphics{sistema}  & \multicolumn{1}{p{12cm}}{Regresa al paso 3 de la trayectoria principal}\tabularnewline
 &  & \multicolumn{1}{p{12cm}}{.... Fin del caso de uso}\tabularnewline
\end{tabular}

% = = = = = = = = = = = = = = = = = = = = = = = = =
%	FIN DE TRAYECTORIA
% = = = = = = = = = = = = = = = = = = = = = = = = =


% = = = = = = = = = = = = = = = = = = = = = = = = =
%    INICIO DE TRAYECTORIA
% = = = = = = = = = = = = = = = = = = = = = = = = =
\subsection{CUC2.0: Gestionar Proveedores}

\textbf{\large Resumen}\\
 {\large {} {} Este caso de uso tiene como objetivo permitir al
usuario visualizar los proveedores que existen en el sistema, y le
muestra opciones para registrar, modificar, eliminar, desbloquear
y filtrar los resultados.}\\


% = = = = = = = = = = = = = = = = = = = = = = = = =
%    INICIO DE TABLA
% = = = = = = = = = = = = = = = = = = = = = = = = = 
\begin{table}[!ht]
\begin{centering}
\begin{tabular}{|c||c|l|}
\hline 
\multicolumn{2}{|c|}{Caso de Uso:} & CUC2.0: Gestionar Proveedores\tabularnewline
\hline 
\multicolumn{3}{|>{\columncolor[gray]{0.7}}c}{Resumen de Atributos}\tabularnewline
\hline 
\multicolumn{2}{|c|}{Autor} & \multicolumn{1}{p{10cm}||}{Jiménez López Carlos Adrian}\tabularnewline
\hline 
\multicolumn{2}{|c|}{Actor} & \multicolumn{1}{p{10cm}||}{Usuario de Compras}\tabularnewline
\hline 
\multicolumn{2}{|c|}{Propósito} & \multicolumn{1}{p{10cm}||}{Gestionar la información de los proveedores que hay en el sistema}\tabularnewline
\hline 
\multicolumn{2}{|c|}{Entradas} & \multicolumn{1}{p{10cm}||}{Ninguna}\tabularnewline
\hline 
\multicolumn{2}{|c|}{Salidas} & \multicolumn{1}{p{10cm}||}{Pantalla PC1}\tabularnewline
\hline 
\multicolumn{2}{|c|}{Pre-condiciones} & \multicolumn{1}{p{10cm}||}{Iniciar sesión como usuario de compras}\tabularnewline
\hline 
\multicolumn{2}{|c|}{Pos-condiciones} & \multicolumn{1}{p{10cm}||}{Ninguna}\tabularnewline
\hline 
\multicolumn{2}{|c|}{Errores} & \multicolumn{1}{p{10cm}||}{MSG4 Error al conectar con la base de datos.}\tabularnewline
\hline 
\multicolumn{2}{|c|}{Tipo} & \multicolumn{1}{p{10cm}||}{Primario}\tabularnewline
\hline 
\multicolumn{2}{|c|}{Fuente} & \multicolumn{1}{p{10cm}||}{}\tabularnewline
\hline 
\end{tabular}
\par\end{centering}

\caption{Caso de Uso Compras 2.0: Gestionar Proveedores}


\label{CUC1.0} 
\end{table}


% = = = = = = = = = = = = = = = = = = = = = = = = =
%	FIN DE TABLA
% = = = = = = = = = = = = = = = = = = = = = = = = =


\textbf{\large Trayectorias del CU}{\large \par}

\textit{\large Trayectoria Principal}{\large {} }{\large \par}

\begin{tabular}{ccl}
 &  & \tabularnewline
1.  & \includegraphics{actor}  & \multicolumn{1}{p{12cm}}{Presiona el menú {[}Gestión de Proveedores{]}.}\tabularnewline
2.  & \includegraphics{sistema}  & \multicolumn{1}{p{12cm}}{Carga los datos de los proveedores existentes en el sistema {[}Trayectoria
A{]}}\tabularnewline
3.  & \includegraphics{sistema}  & \multicolumn{1}{p{12cm}}{Despliega la pantalla PC1}\tabularnewline
4.  & \includegraphics{actor}  & \multicolumn{1}{p{12cm}}{Da clic en el botón {[}Registrar Proveedor{]} {[}Trayectoria B{]} {[}Trayectoria C{]} {[}Trayectoria D{]} {[}Trayectoria E{]}}\tabularnewline
5.  & \includegraphics{sistema}  & \multicolumn{1}{p{12cm}}{Extiende al CUC2.1 ``Registrar Proveedor"}\tabularnewline
 &  & \multicolumn{1}{p{12cm}}{.... Fin del caso de uso}\tabularnewline
\end{tabular}
\\


\textit{Trayectoria Alternativa A}

Condición: No se pudieron cargar los proveedores registrados

\begin{tabular}{ccl}
 &  & \tabularnewline
1.  & \includegraphics{sistema}  & \multicolumn{1}{p{12cm}}{ Muestra el mensaje de error MSG4 'No se pudo establecer conexión
con la base de datos'}\tabularnewline
 &  & \multicolumn{1}{p{12cm}}{.... Fin del caso de uso}\tabularnewline
\end{tabular}
\\

\textit{Trayectoria Alternativa B}

Condición: El actor escribe algún texto en el campo de búsqueda y presiona enter.

\begin{tabular}{ccl}
 &  & \tabularnewline
1.  & \includegraphics{sistema}  & \multicolumn{1}{p{12cm}}{Incluye al CUC2.0.1'}\tabularnewline
 &  & \multicolumn{1}{p{12cm}}{.... Fin del caso de uso}\tabularnewline
\end{tabular}

% = = = = = = = = = = = = = = = = = = = = = = = = =
%    FIN DE TRAYECTORIA
% = = = = = = = = = = = = = = = = = = = = = = = = = 


% = = = = = = = = = = = = = = = = = = = = = = = = =
%    INICIO DE TRAYECTORIA
% = = = = = = = = = = = = = = = = = = = = = = = = =
\subsubsection{CUC2.0.1: Buscar}

\textbf{\large Resumen}\\
 {\large {} {} Este caso de uso tiene como objetivo permitir al
usuario buscar un proveedor por medio palabra(s).}\\


% = = = = = = = = = = = = = = = = = = = = = = = = =
%    INICIO DE TABLA
% = = = = = = = = = = = = = = = = = = = = = = = = = 
\begin{table}[!ht]
\begin{centering}
\begin{tabular}{|c||c|l|}
\hline 
\multicolumn{2}{|c|}{Caso de Uso:} & CUC2.0.1: Buscar\tabularnewline
\hline 
\multicolumn{3}{|>{\columncolor[gray]{0.7}}c}{Resumen de Atributos}\tabularnewline
\hline 
\multicolumn{2}{|c|}{Autor} & \multicolumn{1}{p{10cm}||}{Jiménez López Carlos Adrian}\tabularnewline
\hline 
\multicolumn{2}{|c|}{Actor} & \multicolumn{1}{p{10cm}||}{Usuario de Compras}\tabularnewline
\hline 
\multicolumn{2}{|c|}{Propósito} & \multicolumn{1}{p{10cm}||}{Buscar los proveedores que coincidan con un criterio de búsqueda}\tabularnewline
\hline 
\multicolumn{2}{|c|}{Entradas} & \multicolumn{1}{p{10cm}||}{Palabra(s) Clave}\tabularnewline
\hline 
\multicolumn{2}{|c|}{Salidas} & \multicolumn{1}{p{10cm}||}{Lista de proveedores que coinciden con la entrada}\tabularnewline
\hline 
\multicolumn{2}{|c|}{Pre-condiciones} & \multicolumn{1}{p{10cm}||}{Iniciar sesión como usuario de compras}\tabularnewline
\hline 
\multicolumn{2}{|c|}{Pos-condiciones} & \multicolumn{1}{p{10cm}||}{Ninguna}\tabularnewline
\hline 
\multicolumn{2}{|c|}{Errores} & \multicolumn{1}{p{10cm}||}{MSG3 Datos no encontrados, MSG4 Error al conectar con la base de datos.}\tabularnewline
\hline 
\multicolumn{2}{|c|}{Tipo} & \multicolumn{1}{p{10cm}||}{Secundario}\tabularnewline
\hline 
\multicolumn{2}{|c|}{Fuente} & \multicolumn{1}{p{10cm}||}{CUC2.0: Gestionar Proveedores}\tabularnewline
\hline 
\end{tabular}
\par\end{centering}

\caption{Caso de Uso Compras 2.0.1: Buscar}


\label{CUC1.0} 
\end{table}


% = = = = = = = = = = = = = = = = = = = = = = = = =
%    FIN DE TABLA
% = = = = = = = = = = = = = = = = = = = = = = = = =


\textbf{\large Trayectorias del CU}{\large \par}

\textit{\large Trayectoria Principal}{\large {} }{\large \par}

\begin{tabular}{ccl}
 &  & \tabularnewline
1.  & \includegraphics{sistema}  & \multicolumn{1}{p{12cm}}{Verifica si la entrada no es vacía {[}Trayectoria
A{]}}\tabularnewline
2.  & \includegraphics{sistema}  & \multicolumn{1}{p{12cm}}{Consulta a los proveedores que cumplen con el criterio.{[}Trayectoria B{]}}\tabularnewline
3.  & \includegraphics{sistema}  & \multicolumn{1}{p{12cm}}{Muestra el resultado de la búsqueda{[}Trayectoria C{]}}\tabularnewline
 &  & \multicolumn{1}{p{12cm}}{.... Fin del caso de uso}\tabularnewline
\end{tabular}
\\

\textit{Trayectoria Alternativa A}

Condición: La entrada es vacía.

\begin{tabular}{ccl}
 &  & \tabularnewline
1.  & \includegraphics{sistema}  & \multicolumn{1}{p{12cm}}{Consulta todos los proveedores que existen.}\tabularnewline
2.  & \includegraphics{sistema}  & \multicolumn{1}{p{12cm}}{Muestra el resultado.}\tabularnewline
 &  & \multicolumn{1}{p{12cm}}{.... Fin del caso de uso}\tabularnewline
\end{tabular}

\textit{Trayectoria Alternativa B}

Condición: No se pudieron cargar los proveedores registrados

\begin{tabular}{ccl}
 &  & \tabularnewline
1.  & \includegraphics{sistema}  & \multicolumn{1}{p{12cm}}{ Muestra el mensaje de error MSG4 'No se pudo establecer conexión
con la base de datos'}\tabularnewline
 &  & \multicolumn{1}{p{12cm}}{.... Fin del caso de uso}\tabularnewline
\end{tabular}
\\

\textit{Trayectoria Alternativa C}

Condición: No arrojó resultados la búsqueda

\begin{tabular}{ccl}
 &  & \tabularnewline
1.  & \includegraphics{sistema}  & \multicolumn{1}{p{12cm}}{ Muestra el mensaje de error MSG3 'La búsqueda liconsa no muestra ningún resultado'}\tabularnewline
 &  & \multicolumn{1}{p{12cm}}{.... Fin del caso de uso}\tabularnewline
\end{tabular}
\\



% = = = = = = = = = = = = = = = = = = = = = = = = =
%    FIN DE TRAYECTORIA
% = = = = = = = = = = = = = = = = = = = = = = = = = 

% = = = = = = = = = = = = = = = = = = = = = = = = =
%    INICIO DE TRAYECTORIA
% = = = = = = = = = = = = = = = = = = = = = = = = = 



\subsubsection{CUC2.1: Registrar Proveedor}

\textbf{\large Resumen}\\
 {\large {} {} Este caso de uso tiene como objetivo permitir al
usuario registrar un nuevo proveedor dentro del sistema.}\\


% = = = = = = = = = = = = = = = = = = = = = = = = =
%	INICIO DE TABLA
% = = = = = = = = = = = = = = = = = = = = = = = = = 
% Se utiliza para que no se salga la tabla de la hoja, porque sino se autodimensiona
%\multicolumn{1}{p{10cm}||}{Contenido}
\begin{table}[!ht]
\begin{centering}
\begin{tabular}{|c||c|l|}
\hline 
\multicolumn{2}{|c|}{Caso de Uso:} & CUC1.1: Registrar Proveedor\tabularnewline
\hline 
\multicolumn{3}{|>{\columncolor[gray]{0.7}}c}{Resumen de Atributos}\tabularnewline
\hline 
\multicolumn{2}{|c|}{Autor} & \multicolumn{1}{p{10cm}||}{Jiménez López Carlos Adrian}\tabularnewline
\hline 
\multicolumn{2}{|c|}{Actor} & \multicolumn{1}{p{10cm}||}{Usuario de Compras}\tabularnewline
\hline 
\multicolumn{2}{|c|}{Propósito} & \multicolumn{1}{p{10cm}||}{Registrar un nuevo proveedor }\tabularnewline
\hline 
\multicolumn{2}{|c|}{Entradas} & \multicolumn{1}{p{10cm}||}{RFC,
 Telefono, E-mail, Lista de productos.}\tabularnewline
\hline 
\multicolumn{2}{|c|}{Salidas} & \multicolumn{1}{p{10cm}||}{Ninguna}\tabularnewline
\hline 
\multicolumn{2}{|c|}{Pre-condiciones} & \multicolumn{1}{p{10cm}||}{Ninguna}\tabularnewline
\hline 
\multicolumn{2}{|c|}{Pos-condiciones} & \multicolumn{1}{p{10cm}||}{El proveedor debe ser agregado a la base de datos}\tabularnewline
\hline 
\multicolumn{2}{|c|}{Errores} & \multicolumn{1}{p{10cm}||}{MSG4 Error al conectar con la base de datos. MSG7 Formato incorrecto
de los datos. MSG8 Campos obligatorios. MSG11 Datos duplicados.}\tabularnewline
\hline 
\multicolumn{2}{|c|}{Tipo} & \multicolumn{1}{p{10cm}||}{Secundario}\tabularnewline
\hline 
\multicolumn{2}{|c|}{Fuente} & \multicolumn{1}{p{10cm}||}{Proveniente del CUC2.0.}\tabularnewline
\hline 
\end{tabular}
\par\end{centering}

\caption{Caso de Uso Compras 2.1: Registrar Proveedor}


\label{CUC1.1} 
\end{table}


% = = = = = = = = = = = = = = = = = = = = = = = = =
%	FIN DE TABLA
% = = = = = = = = = = = = = = = = = = = = = = = = =


\textbf{\large Trayectorias del CU}{\large \par}

\textit{\large Trayectoria Principal}{\large {} }{\large \par}

%Solo hay que cambiar el nombre de la imagen dependiendo de si es actor o sistema


\begin{tabular}{ccl}
 &  & \tabularnewline
1.  & \includegraphics{actor}  & \multicolumn{1}{p{12cm}}{ Presiona el botón {[}Registrar Proveedor{]} en la pantalla PC1}\tabularnewline
2.  & \includegraphics{sistema}  & \multicolumn{1}{p{12cm}}{ Despliega la pantalla PC5 y pide los datos necesarios.}\tabularnewline
3.  & \includegraphics{actor}  & \multicolumn{1}{p{12cm}}{ Llena el formulario y presiona el botón {[}Registrar{]}{[}Trayectoria
A{]}{[}Trayectoria B{]}{[}Trayectoria C{]}}\tabularnewline
4.  & \includegraphics{sistema}  & \multicolumn{1}{p{12cm}}{ Registra al nuevo proveedor {[}Trayectoria D{]}}\tabularnewline
5.  & \includegraphics{sistema}  & \multicolumn{1}{p{12cm}}{ Despliega el mensaje MSG 1 'El proveedor ha sido registrado exitosamente'}\tabularnewline
6.  & \includegraphics{sistema}  & \multicolumn{1}{p{12cm}}{ Despliega la pantalla PC1}\tabularnewline
 &  & \multicolumn{1}{p{12cm}}{.... Fin del caso de uso}\tabularnewline
\end{tabular}\\


\textit{Trayectoria A}

Condición: Alguno de los campos fue llenado incorrectamente

\begin{tabular}{ccl}
 &  & \tabularnewline
1.  & \includegraphics{sistema}  & \multicolumn{1}{p{12cm}}{ Muestra el mensaje de error MSG7 'El RFC sólo acepta letras y números',
'El teléfono sólo acepta números', 'El email sólo acepta letras, numeros,
punto(.), guiones(-) y guión bajo(\_)' según corresponda}\tabularnewline
2.  & \includegraphics{sistema}  & \multicolumn{1}{p{12cm}}{Regresa a la pantalla PC5}\tabularnewline
 &  & \multicolumn{1}{p{12cm}}{.... Fin del caso de uso}\tabularnewline
\end{tabular}\\


\textit{Trayectoria B}

Condición: Alguno de los campos está vacío y es obligatorio

\begin{tabular}{ccl}
 &  & \tabularnewline
1.  & \includegraphics{sistema}  & \multicolumn{1}{p{12cm}}{ Muestra el mensaje de error MSG8 'El campo RFC es un campo obligatorio',
'El campo teléfono es un campo obligatorio', 'El campo lista de productos
es un campo obligatorio' según corresponda}\tabularnewline
2.  & \includegraphics{sistema}  & \multicolumn{1}{p{12cm}}{Regresa a la pantalla PC5}\tabularnewline
 &  & \multicolumn{1}{p{12cm}}{.... Fin del caso de uso}\tabularnewline
\end{tabular}\\


\textit{Trayectoria C}

Condición: El RFC ya se encuentra registrado

\begin{tabular}{ccl}
 &  & \tabularnewline
1.  & \includegraphics{sistema}  & \multicolumn{1}{p{12cm}}{ Muestra el mensaje de error MSG11 'El RFC ABCD910123 ya existe'}\tabularnewline
2.  & \includegraphics{sistema}  & \multicolumn{1}{p{12cm}}{Regresa a la pantalla PC5}\tabularnewline
 &  & \multicolumn{1}{p{12cm}}{.... Fin del caso de uso}\tabularnewline
\end{tabular}\\


\textit{Trayectoria D}

Condición: No se pudo insertar el proveedor en la base de datos

\begin{tabular}{ccl}
 &  & \tabularnewline
1.  & \includegraphics{sistema}  & \multicolumn{1}{p{12cm}}{Muestra el mensaje de error MSG4 'No se pudo establecer conexion con
la base de datos'}\tabularnewline
2.  & \includegraphics{sistema}  & \multicolumn{1}{p{12cm}}{Regresa a la pantalla PC2}\tabularnewline
 &  & \multicolumn{1}{p{12cm}}{.... Fin del caso de uso}\tabularnewline
\end{tabular}

\newpage{}


\subsubsection{CUC2.2: Modificar Proveedor}

\textbf{\large Resumen}\\
 {\large {} {} Este caso de uso tiene como objetivo permitir al
usuario modificar los datos de un proveedor.}\\


% = = = = = = = = = = = = = = = = = = = = = = = = =
%	INICIO DE TABLA
% = = = = = = = = = = = = = = = = = = = = = = = = = 
% Se utiliza para que no se salga la tabla de la hoja, porque sino se autodimensiona
%\multicolumn{1}{p{10cm}||}{Contenido}
\begin{table}[!ht]
\begin{centering}
\begin{tabular}{|c||c|l|}
\hline 
\multicolumn{2}{|c|}{Caso de Uso:} & CUC2.2: Modificar Proveedor\tabularnewline
\hline 
\multicolumn{3}{|>{\columncolor[gray]{0.7}}c}{Resumen de Atributos}\tabularnewline
\hline 
\multicolumn{2}{|c|}{Autor} & \multicolumn{1}{p{10cm}||}{Jiménez López Carlos Adrian}\tabularnewline
\hline 
\multicolumn{2}{|c|}{Actor} & \multicolumn{1}{p{10cm}||}{Usuario de Compras}\tabularnewline
\hline 
\multicolumn{2}{|c|}{Propósito} & \multicolumn{1}{p{10cm}||}{Modificar los datos de un proveedor registrado}\tabularnewline
\hline 
\multicolumn{2}{|c|}{Entradas} & \multicolumn{1}{p{10cm}||}{RFC, 
 Telefono, E-mail, Lista de productos}\tabularnewline
\hline 
\multicolumn{2}{|c|}{Salidas} & \multicolumn{1}{p{10cm}||}{Ninguna}\tabularnewline
\hline 
\multicolumn{2}{|c|}{Pre-condiciones} & \multicolumn{1}{p{10cm}||}{Debe seleccionar un proveedor de la pantalla PC1}\tabularnewline
\hline 
\multicolumn{2}{|c|}{Pos-condiciones} & \multicolumn{1}{p{10cm}||}{El proveedor debe ser actualizado en la base de datos}\tabularnewline
\hline 
\multicolumn{2}{|c|}{Errores} & \multicolumn{1}{p{10cm}||}{MSG4 Error al conectar con la base de datos. MSG7 Formato incorrecto
de los datos. MSG8 Campos obligatorios. MSG11 Datos duplicados.}\tabularnewline
\hline 
\multicolumn{2}{|c|}{Tipo} & \multicolumn{1}{p{10cm}||}{Secundario}\tabularnewline
\hline 
\multicolumn{2}{|c|}{Fuente} & \multicolumn{1}{p{10cm}||}{Proveniente del CUC2.0.}\tabularnewline
\hline 
\end{tabular}
\par\end{centering}

\caption{Caso de Uso Compras 2.2: Modificar Proveedor}


\label{CUC1.2} 
\end{table}


% = = = = = = = = = = = = = = = = = = = = = = = = =
%	FIN DE TABLA
% = = = = = = = = = = = = = = = = = = = = = = = = =


\textbf{\large Trayectorias del CU}{\large \par}

\textit{\large Trayectoria Principal}{\large {} }{\large \par}

%Solo hay que cambiar el nombre de la imagen dependiendo de si es actor o sistema


\begin{tabular}{ccl}
 &  & \tabularnewline
1.  & \includegraphics{actor}  & \multicolumn{1}{p{12cm}}{ Presiona el ícono {[}Modificar Proveedor{]} en la pantalla PC1}\tabularnewline
2.  & \includegraphics{sistema}  & \multicolumn{1}{p{12cm}}{ Despliega la pantalla PC6 y pide los datos necesarios.}\tabularnewline
3.  & \includegraphics{actor}  & \multicolumn{1}{p{12cm}}{ Llena el formulario y presiona el botón {[}Actualizar{]}{[}Trayectoria
A{]}{[}Trayectoria B{]}{[}Trayectoria C{]}}\tabularnewline
4.  & \includegraphics{sistema}  & \multicolumn{1}{p{12cm}}{ Actualiza los datos del proveedor {[}Trayectoria D{]}}\tabularnewline
5.  & \includegraphics{sistema}  & \multicolumn{1}{p{12cm}}{ Despliega el mensaje MSG 1 'El proveedor ha sido modificado exitosamente'}\tabularnewline
6.  & \includegraphics{sistema}  & \multicolumn{1}{p{12cm}}{ Despliega la pantalla PC1}\tabularnewline
 &  & \multicolumn{1}{p{12cm}}{.... Fin del caso de uso}\tabularnewline
\end{tabular}\\


\textit{Trayectoria A}

Condición: Alguno de los campos fue llenado incorrectamente

\begin{tabular}{ccl}
 &  & \tabularnewline
1.  & \includegraphics{sistema}  & \multicolumn{1}{p{12cm}}{ Muestra el mensaje de error MSG7 'El RFC sólo acepta letras y números',
'El teléfono sólo acepta números', 'El email sólo acepta letras, numeros,
punto(.), guiones(-) y guión bajo(\_)' según corresponda}\tabularnewline
2.  & \includegraphics{sistema}  & \multicolumn{1}{p{12cm}}{Regresa a la pantalla PC6}\tabularnewline
 &  & \multicolumn{1}{p{12cm}}{.... Fin del caso de uso}\tabularnewline
\end{tabular}\\


\textit{Trayectoria B}

Condición: Alguno de los campos está vacío y es obligatorio

\begin{tabular}{ccl}
 &  & \tabularnewline
1.  & \includegraphics{sistema}  & \multicolumn{1}{p{12cm}}{ Muestra el mensaje de error MSG8 'El campo RFC es un campo obligatorio',
'El campo teléfono es un campo obligatorio', 'El campo lista de productos
es un campo obligatorio' según corresponda}\tabularnewline
2.  & \includegraphics{sistema}  & \multicolumn{1}{p{12cm}}{Regresa a la pantalla PC6}\tabularnewline
 &  & \multicolumn{1}{p{12cm}}{.... Fin del caso de uso}\tabularnewline
\end{tabular}\\


\textit{Trayectoria C}

Condición: El RFC ya se encuentra registrado

\begin{tabular}{ccl}
 &  & \tabularnewline
1.  & \includegraphics{sistema}  & \multicolumn{1}{p{12cm}}{ Muestra el mensaje de error MSG11 'El RFC ABCD910123 ya existe'}\tabularnewline
2.  & \includegraphics{sistema}  & \multicolumn{1}{p{12cm}}{Regresa a la pantalla PC6}\tabularnewline
 &  & \multicolumn{1}{p{12cm}}{.... Fin del caso de uso}\tabularnewline
\end{tabular}\\


\textit{Trayectoria D}

Condición: No se pudo insertar el proveedor en la base de datos

\begin{tabular}{ccl}
 &  & \tabularnewline
1.  & \includegraphics{sistema}  & \multicolumn{1}{p{12cm}}{Muestra el mensaje de error MSG4 'No se pudo establecer conexion con
la base de datos'}\tabularnewline
2.  & \includegraphics{sistema}  & \multicolumn{1}{p{12cm}}{Regresa a la pantalla PC6}\tabularnewline
 &  & \multicolumn{1}{p{12cm}}{.... Fin del caso de uso}\tabularnewline
\end{tabular}

%====================
%Nuevo caso de uso
%====================



\subsubsection{CUC2.3: Eliminar Proveedor}

\textbf{\large Resumen}\\
 {\large {} {} Este caso de uso tiene como objetivo permitir al
usuario eliminar los datos de un proveedor.}\\


% = = = = = = = = = = = = = = = = = = = = = = = = =
%	INICIO DE TABLA
% = = = = = = = = = = = = = = = = = = = = = = = = = 
% Se utiliza para que no se salga la tabla de la hoja, porque sino se autodimensiona
%\multicolumn{1}{p{10cm}||}{Contenido}
\begin{table}[!ht]
\begin{centering}
\begin{tabular}{|c||c|l|}
\hline 
\multicolumn{2}{|c|}{Caso de Uso:} & CUC2.3: Eliminar Proveedor\tabularnewline
\hline 
\multicolumn{3}{|>{\columncolor[gray]{0.7}}c}{Resumen de Atributos}\tabularnewline
\hline 
\multicolumn{2}{|c|}{Autor} & \multicolumn{1}{p{10cm}||}{Jiménez López Carlos Adrian}\tabularnewline
\hline 
\multicolumn{2}{|c|}{Actor} & \multicolumn{1}{p{10cm}||}{Usuario de Compras}\tabularnewline
\hline 
\multicolumn{2}{|c|}{Propósito} & \multicolumn{1}{p{10cm}||}{Eliminar un proveedor de manera permanente}\tabularnewline
\hline 
\multicolumn{2}{|c|}{Entradas} & \multicolumn{1}{p{10cm}||}{Ninguna}\tabularnewline
\hline 
\multicolumn{2}{|c|}{Salidas} & \multicolumn{1}{p{10cm}||}{Ninguna}\tabularnewline
\hline 
\multicolumn{2}{|c|}{Pre-condiciones} & \multicolumn{1}{p{10cm}||}{Debe seleccionar un proveedor de la pantalla PC1}\tabularnewline
\hline 
\multicolumn{2}{|c|}{Pos-condiciones} & \multicolumn{1}{p{10cm}||}{El proveedor debe ser eliminado de la base de datos}\tabularnewline
\hline 
\multicolumn{2}{|c|}{Errores} & \multicolumn{1}{p{10cm}||}{MSG4 Error al conectar con la base de datos.}\tabularnewline
\hline 
\multicolumn{2}{|c|}{Tipo} & \multicolumn{1}{p{10cm}||}{Secundario}\tabularnewline
\hline 
\multicolumn{2}{|c|}{Fuente} & \multicolumn{1}{p{10cm}||}{Proveniente del CUC2.0.}\tabularnewline
\hline 
\end{tabular}
\par\end{centering}

\caption{Caso de Uso Compras 2.3: Eliminar Proveedor}


\label{CUC1.3} 
\end{table}


% = = = = = = = = = = = = = = = = = = = = = = = = =
%	FIN DE TABLA
% = = = = = = = = = = = = = = = = = = = = = = = = =


\textbf{\large Trayectorias del CU}{\large \par}

\textit{\large Trayectoria Principal}{\large {} }{\large \par}

%Solo hay que cambiar el nombre de la imagen dependiendo de si es actor o sistema


\begin{tabular}{ccl}
 &  & \tabularnewline
1.  & \includegraphics{actor}  & \multicolumn{1}{p{12cm}}{ Presiona el ícono {[}Eliminar Proveedor{]} en la pantalla PC1}\tabularnewline
2.  & \includegraphics{sistema}  & \multicolumn{1}{p{12cm}}{Despliega el MSG9 '¿Desea eliminar el proveedor?' pidiendo la confirmación
para eliminar.}\tabularnewline
3.  & \includegraphics{actor}  & \multicolumn{1}{p{12cm}}{Presiona el botón {[}Eliminar{]} {[}Trayectoria A{]}}\tabularnewline
4.  & \includegraphics{sistema}  & \multicolumn{1}{p{12cm}}{ Elimina al proveedor {[}Trayectoria B{]}}\tabularnewline
5.  & \includegraphics{sistema}  & \multicolumn{1}{p{12cm}}{ Despliega el mensaje MSG 1 'El proveedor ha sido eliminado exitosamente'}\tabularnewline
6.  & \includegraphics{sistema}  & \multicolumn{1}{p{12cm}}{ Despliega la pantalla PC1}\tabularnewline
 &  & \multicolumn{1}{p{12cm}}{.... Fin del caso de uso}\tabularnewline
\end{tabular}\\


\textit{Trayectoria A}

Condición: El usuario presiona el botón {[}Cancelar{]}

\begin{tabular}{ccl}
 &  & \tabularnewline
1.  & \includegraphics{sistema}  & \multicolumn{1}{p{12cm}}{Regresa a la pantalla PC1}\tabularnewline
 &  & \multicolumn{1}{p{12cm}}{.... Fin del caso de uso}\tabularnewline
\end{tabular}\\


\textit{Trayectoria B}

Condición: No se pudo eliminar el proveedor de la base de datos

\begin{tabular}{ccl}
 &  & \tabularnewline
1.  & \includegraphics{sistema}  & \multicolumn{1}{p{12cm}}{Muestra el mensaje de error MSG4 'No se pudo establecer conexion con
la base de datos'}\tabularnewline
2.  & \includegraphics{sistema}  & \multicolumn{1}{p{12cm}}{Regresa a la pantalla PC1}\tabularnewline
 &  & \multicolumn{1}{p{12cm}}{.... Fin del caso de uso}\tabularnewline
\end{tabular}

%====================
%Nuevo caso de uso
%====================



\subsubsection{CUC2.4: Desbloquear Proveedor}

\textbf{\large Resumen}\\
 {\large {} {} Este caso de uso tiene como objetivo permitir al
usuario desbloquear a un proveedor.}\\


% = = = = = = = = = = = = = = = = = = = = = = = = =
%	INICIO DE TABLA
% = = = = = = = = = = = = = = = = = = = = = = = = = 
% Se utiliza para que no se salga la tabla de la hoja, porque sino se autodimensiona
%\multicolumn{1}{p{10cm}||}{Contenido}
\begin{table}[!ht]
\begin{centering}
\begin{tabular}{|c||c|l|}
\hline 
\multicolumn{2}{|c|}{Caso de Uso:} & CUC2.4: Desbloquear Proveedor\tabularnewline
\hline 
\multicolumn{3}{|>{\columncolor[gray]{0.7}}c}{Resumen de Atributos}\tabularnewline
\hline 
\multicolumn{2}{|c|}{Autor} & \multicolumn{1}{p{10cm}||}{Jiménez López Carlos Adrian}\tabularnewline
\hline 
\multicolumn{2}{|c|}{Actor} & \multicolumn{1}{p{10cm}||}{Usuario de Compras}\tabularnewline
\hline 
\multicolumn{2}{|c|}{Propósito} & \multicolumn{1}{p{10cm}||}{Desbloquear un proveedor}\tabularnewline
\hline 
\multicolumn{2}{|c|}{Entradas} & \multicolumn{1}{p{10cm}||}{Ninguna}\tabularnewline
\hline 
\multicolumn{2}{|c|}{Salidas} & \multicolumn{1}{p{10cm}||}{Ninguna}\tabularnewline
\hline 
\multicolumn{2}{|c|}{Pre-condiciones} & \multicolumn{1}{p{10cm}||}{Debe seleccionar un proveedor de la pantalla PC1.\medskip{}
 El proveedor debe estar bloqueado}\tabularnewline
\hline 
\multicolumn{2}{|c|}{Pos-condiciones} & \multicolumn{1}{p{10cm}||}{El proveedor debe ser desbloqueado de la base de datos}\tabularnewline
\hline 
\multicolumn{2}{|c|}{Errores} & \multicolumn{1}{p{10cm}||}{MSG4 Error al conectar con la base de datos.}\tabularnewline
\hline 
\multicolumn{2}{|c|}{Tipo} & \multicolumn{1}{p{10cm}||}{Secundario}\tabularnewline
\hline 
\multicolumn{2}{|c|}{Fuente} & \multicolumn{1}{p{10cm}||}{Proveniente del CUC2.0.}\tabularnewline
\hline 
\end{tabular}
\par\end{centering}

\caption{Caso de Uso Compras 2.4: Desbloquear Proveedor}


\label{tab:CasosdeUso:nombredecasodeuso-1-1-1} 
\end{table}


% = = = = = = = = = = = = = = = = = = = = = = = = =
%	FIN DE TABLA
% = = = = = = = = = = = = = = = = = = = = = = = = =


\textbf{\large Trayectorias del CU}{\large \par}

\textit{\large Trayectoria Principal}{\large {} }{\large \par}

%Solo hay que cambiar el nombre de la imagen dependiendo de si es actor o sistema


\begin{tabular}{ccl}
 &  & \tabularnewline
1.  & \includegraphics{actor}  & \multicolumn{1}{p{12cm}}{ Presiona el ícono {[}Desbloquear Proveedor{]} en la pantalla PC1}\tabularnewline
2.  & \includegraphics{sistema}  & \multicolumn{1}{p{12cm}}{Despliega el MSG9 '¿Desea desbloquear el proveedor?' pidiendo la confirmación
para desbloquear.}\tabularnewline
3.  & \includegraphics{actor}  & \multicolumn{1}{p{12cm}}{Presiona el botón {[}Desbloquear{]} {[}Trayectoria A{]}}\tabularnewline
4.  & \includegraphics{sistema}  & \multicolumn{1}{p{12cm}}{ Desbloquea al proveedor {[}Trayectoria B{]}}\tabularnewline
5.  & \includegraphics{sistema}  & \multicolumn{1}{p{12cm}}{ Despliega el mensaje MSG 1 'El proveedor ha sido desbloqueado exitosamente'}\tabularnewline
6.  & \includegraphics{sistema}  & \multicolumn{1}{p{12cm}}{ Despliega la pantalla PC1}\tabularnewline
 &  & \multicolumn{1}{p{12cm}}{.... Fin del caso de uso}\tabularnewline
\end{tabular}\\


\textit{Trayectoria A}

Condición: El usuario presiona el botón {[}Cancelar{]}

\begin{tabular}{ccl}
 &  & \tabularnewline
1.  & \includegraphics{sistema}  & \multicolumn{1}{p{12cm}}{Regresa a la pantalla PC1}\tabularnewline
 &  & \multicolumn{1}{p{12cm}}{.... Fin del caso de uso}\tabularnewline
\end{tabular}\\


\textit{Trayectoria B}

Condición: No se pudo desbloquear el proveedor en la base de datos

\begin{tabular}{ccl}
 &  & \tabularnewline
1.  & \includegraphics{sistema}  & \multicolumn{1}{p{12cm}}{Muestra el mensaje de error MSG4 'No se pudo establecer conexion con
la base de datos'}\tabularnewline
2.  & \includegraphics{sistema}  & \multicolumn{1}{p{12cm}}{Regresa a la pantalla PC1}\tabularnewline
 &  & \multicolumn{1}{p{12cm}}{.... Fin del caso de uso}\tabularnewline
\end{tabular}

%====================
%Nuevo caso de uso
%====================



\subsection{CUC3.0: Reportes Compras}

\textbf{\large Resumen}\\
 {\large {} {} {} Este Caso de uso tiene como objetivo permitir
al usuario consultar los reportes referidos al área de Compras.}\\


% = = = = = = = = = = = = = = = = = = = = = = = = =
%    INICIO DE TABLA
% = = = = = = = = = = = = = = = = = = = = = = = = = 


%\multicolumn{1}{p{10cm}||}{Contenido}
\begin{table}[!ht]
\begin{centering}
\begin{tabular}{|c||c|l|}
\hline 
\multicolumn{2}{|c|}{Caso de Uso:} & CUC3.0: Reportes\tabularnewline
\hline 
\multicolumn{3}{|>{\columncolor[gray]{0.7}}c}{Resumen de Atributos}\tabularnewline
\hline 
\multicolumn{2}{|c|}{Autor} & \multicolumn{1}{p{10cm}||}{Ventura Cruz Eduardo}\tabularnewline
\hline 
\multicolumn{2}{|c|}{Actor} & \multicolumn{1}{p{10cm}||}{Usuario de Compras}\tabularnewline
\hline 
\multicolumn{2}{|c|}{Propósito} & \multicolumn{1}{p{10cm}||}{Consultar los reportes referidos a Compras. }\tabularnewline
\hline 
\multicolumn{2}{|c|}{Entradas} & \multicolumn{1}{p{10cm}||}{Ninguna}\tabularnewline
\hline 
\multicolumn{2}{|c|}{Salidas} & \multicolumn{1}{p{10cm}||}{Interfaz Correspondiente a Reportes}\tabularnewline
\hline 
\multicolumn{2}{|c|}{Pre-condiciones} & \multicolumn{1}{p{10cm}||}{El usuario de compras debe estar autitificado.}\tabularnewline
\hline 
\multicolumn{2}{|c|}{Pos-condiciones} & \multicolumn{1}{p{10cm}||}{Deben existir datos referentes a los reportes}\tabularnewline
\hline 
\multicolumn{2}{|c|}{Errores} & \multicolumn{1}{p{10cm}||}{MSG4 Error al conectar con la base de datos.}\tabularnewline
\hline 
\multicolumn{2}{|c|}{Tipo} & \multicolumn{1}{p{10cm}||}{Primario}\tabularnewline
\hline 
\multicolumn{2}{|c|}{Fuente} & \multicolumn{1}{p{10cm}||}{Basado en el funcionamiento}\tabularnewline
\hline 
\end{tabular}
\par\end{centering}

\caption{Caso de Uso CUC3.0: Reportes}


\label{CUC2.0} 
\end{table}


% = = = = = = = = = = = = = = = = = = = = = = = = =
%	FIN DE TABLA
% = = = = = = = = = = = = = = = = = = = = = = = = =
% = = = = = = = = = = = = = = = = = = = = = = = = =
%	INICIO DE TRAYECTORIA
% = = = = = = = = = = = = = = = = = = = = = = = = = 


\textbf{\large Trayectorias del CU}{\large \par}

\textit{\large Trayectoria Principal}{\large {} }{\large \par}

\begin{tabular}{ccl}
 &  & \tabularnewline
1.  & \includegraphics{actor}  & \multicolumn{1}{p{12cm}}{ Presiona el botón {[}Reportes{]} en la pantalla PC3}\tabularnewline
2.  & \includegraphics{sistema}  & \multicolumn{1}{p{12cm}}{Muestra dos opciones a elegir. }\tabularnewline
3.  & \includegraphics{sistema}  & \multicolumn{1}{p{12cm}}{Selecciona la opcion deseada {[}Trayectoria Alternativa A{]}}\tabularnewline
 &  & \multicolumn{1}{p{12cm}}{.... Fin del caso de uso}\tabularnewline
\end{tabular}

\textit{Trayectoria Alternativa A}

Condición: No se cargaron los datos de los Reportes

\begin{tabular}{ccl}
 &  & \tabularnewline
1.  & \includegraphics{sistema}  & \multicolumn{1}{p{12cm}}{ Muestra el mensaje de error MSG4 }\tabularnewline
2.  & \includegraphics{sistema}  & \multicolumn{1}{p{12cm}}{Continua con el paso 3}\tabularnewline
 &  & \multicolumn{1}{p{12cm}}{.... Fin del caso de uso}\tabularnewline
\end{tabular}
\end{document}
